% !TEX root
%
% Copyright (c) 2018-2019  UAVCAN Development Team
%

\documentclass{uavcandoc}

\usepackage{multirow}
\usepackage{tabularx}
\usepackage{amsmath}
\usepackage{amssymb}
\usepackage{amsfonts}
\usepackage{longtable}
\usepackage{diagbox}

\urlstyle{same}

% This macro embeds the selected DSDL definition or the contents of a DSDL namespace into the document.
% It accepts one mandatory argument which is either a full DSDL type name, e.g. uavcan.node.Heartbeat,
% or a full type name glob expression, e.g., uavcan.node.*.
\newcommand{\DSDL}[1]{%
    % Clean up beforehand to ensure clean initial state.
    \immediate\write18{rm -f ../*.tmp}%
    % Invoke the target command and save the useful output into a file; ignore error output
    \immediate\write18{../render_dsdl.py #1 > ../dsdl.tmp}%
    % Now, if the above command has failed, the output file would be empty. We remove empty file to escalate error.
    % Escalation is very important as it allows us to abort compilation on failure instead of generating invalid
    % documents silently.
    \immediate\write18{find .. -type f -name '*.tmp' -size 0 -delete}%
    % Read the file. This command fails if the file was empty, which is exactly want we want.
    \immediate\chapter{Data structure description language}\label{sec:dsdl}

데이터 구조 서술 언어 \emph{DSDL}은 컴포지트 데이터 타입을 정의하기 위해 설계된 간단한 도메인 특화 언어이다.
정의한 데이터 타입은 표준 UAVCAN 전송 프로토콜 중에 하나를 통해서 UAVCAN 노드들 사이에 데이터 교환에 사용된다.\footnote{The standard transport protocols are documented in chapter~\ref{sec:transport}.
UAVCAN은 사용자가 자신의 어플리케이션에 특화된 트랜스포트를 정의할 수 있지만 이 경우 호환성 이슈나 해당 프로토콜에서 성능 이슈가 발생할 수 있다.}.

\section{Architecture}

\subsection{General principles}

UAVCAN 아키텍처에 따라 DSDL에서 사용자는 2가지 종류의 데이터 타입을 정의할 수 있다:
메시지 타입과 서비스 타입
메시지 타입은 publish-subscribe 상에서 데이터 교환에 사용되며 1:다 메시지 링크는 서브젝트-ID로 식별한다. 서비스 타입은 request-response를 수행하는데 사용하며 RPC와 같이 1:1 교환에서 서비스-ID로 식별한다.
서비스 타입은 정확히 2개 내부 데이터 타입으로 구성된다:
그 중에 하나는 request 타입(클라이언트에서 서버로 전송)이고,
나머지 하나는 response 타입(서버에서 클라이언트로 전송)이다.

UAVCAN의 deterministic 속성을 따라서 메시지나 서비스 객체의 크기는 정적으로 이미 알고 있는 크기 이내에 한정된다.
가변길이 엔트티는 항상 데이터 타입 디자이너가 정의한 고정 길이 제한을 가진다.

DSDL정의는 정적 타입이다.

DSDL은 데이터 타입 버전관리를 위해 잘 정의되어 있다. 이를 통해 데이터 타입 유지보수자가 배포한 데이터 타입의 하위 호환을 되도록 변경할 수 있다.

DSDL은 확장 가능한 정적 분석을 지원하도록 설계되었다. 하위 바이너리 호환성과 데이터 필드 레이아웃과 같이 데이터 타입 정의의 중요한 속성은 이를 사용하는 시스템이 운영되기 전에 자동 소프트웨어 도구로 검증 및 검사가 가능하다.

DSDL정의는 타겟 프로그래밍 언어에서 직렬화 소스코드 데이터 타입으로 자동 생성될 수 있다.
DSDL 정의를 기반으로 직렬화 코드를 생성하는 기능을 가진 도구를 \emph{DSDL compiler}라고 부른다.
좀더 일반적으로는 말하자면 DSDL 정의로 동작하도록 설계된 소프트웨어 도구를 \emph{DSDL processing tool}라고 부른다.

\subsection{Data types and namespaces}

모든 데이터 타입은 \emph{namespace} 내부에 위치한다.
네임스페이스는 더 상위레벨 네임스페이스에 포함될 수 있으며 트리구조를 가진다.

트리 구조의 최상단에 있는 네임스페이스를 \emph{root namespace}라고 부른다.
다른 네임스페이스 내부에 위치하고 있는 네임스페이스를 \emph{nested namespace}라고 부른다.

데이터 타입은 네임스페이스와 이름\emph{short name}으로 식별된다.
데이터 타입의 이름은 네임스페이트를 제외한 타입의 이름이다.

데이터 타입의 \emph{full name}은 이름과 네임스페이스 이름들로 구성된다.전체 이름에 포함된 이름과 네임스페이스는를 \emph{name components}라고 부른다.
이름 컴포넌트들은 순서를 가진다: 루트 네임스페이스는 항상 맨 앞에 오고 다음으로 nested 네임스페이스가 온다.
전체 이름에서 이름(short name)은 항상 마지막에 위치한다.
전체 이름은 ASCII dot 문자 ``\verb|.|'' (ASCII code 46)를 통해서 이름 컴포넌트들을 연결시킨다.

\emph{full namespace} 이름은 이름(short name)과 component 구분자를 제외한 전체 이름이다.

\emph{sub-root namespace}은 netsted 네임스페이스로 루트 네임스페이스 바로 밑에 위치한다.
루트 네임스페이스 아래에 위치하고 있는 데이터 타입은 하위루트 네임스페이스를 가지지 않는다.

이름 구조는 figure~\ref{fig:dsdl_data_type_name_structure} 그림으로 표현하였다.

\begin{figure}[H]
    $$
    \overbrace{
        \underbrace{
            \underbrace{\texttt{\huge{uavcan}}}_{\substack{\text{root} \\ \text{namespace}}}%
            \texttt{\huge{.}}%
            \underbrace{\texttt{\huge{node}}}_{\substack{\text{nested, also} \\ \text{sub-root} \\ \text{namespace}}}%
            \texttt{\huge{.}}%
            \underbrace{\texttt{\huge{port}}}_{\substack{\text{nested} \\ \text{namespace}}}%
        }_{\text{full namespace}}%
        \texttt{\huge{.}}%
        \underbrace{\texttt{\huge{GetInfo}}}_{\text{short name}}
    }^{\text{full name}}
    $$
    \caption{Data type name structure\label{fig:dsdl_data_type_name_structure}}
\end{figure}

전체 네임스페이스 이름의 집합과 전체 데이터 타입 이름의 집합은 서로 교차되지 않는다.\footnote{%
    예제로 네임스페이스 ``\texttt{vendor.example}''와 데이터 타입 ``\texttt{vendor.example.1.0}''는 서로 상호배제되는 형태다.
    이 예제에서 보여준 데이터 타입 이름은 이름 규칙에 위배된다. 이는 별도 섹션에서 다룰 예정이다.
}.

데이터 타입 이름과 네임스페이스 이름은 대소문자를 구분한다.
하지만 대소문자가 다른 이름들은 허용하지 않는다.\footnote{%
    대소문자를 구분하지 않는 파일시스템에서는 문제를 야기할 수 있기 때문이다.
}.
다시 말하면 대소문자만 다른 한쌍의 이름은 이름 충돌이 발생할 수 있다.

이름 컴포넌트는 알파벳 ASCII 문자와(\verb|A-Z|, \verb|a-z|, \verb|0-9|) 언더스코어 (``\verb|_|'', ASCII code 95)로 구성되어 있다.
빈 문자열은 이름 컴포넌트로 유효하지 않다.
이름 컴포넌트의 첫번째 문자는 숫자가 올 수 없다.
이름 컴포넌트에는 어떠한 예약 워드 패턴도 패칭되지 않으며 목록은 table~\ref{table:dsdl_reserved_word_patterns}을 참조하자.

전체 데이터 타입 이름의 길이는 255문자를 초과하지 않는다.\footnote{This includes the name component separators, but not the version.}

모든 데이터 타입 정의는 메이저와 마이너 버전 넘버 쌍을 할당한다.
데이터 타입 정의를 식별하기 위해서 버전 넘버가 지정된다.
다음 텍스트에서 majority qualifier가 없는 \emph{version}는 용어는 메이저와 마이너 버전 넘버 쌍을 의미한다.

유효 데이터 타입 버전 넘버는 0에서 255의 범위를 가진다.
메이저와 마이너 컴포넌트 모두 0인 데이터 타입 버전은 허용하지 않는다.

\subsection{File hierarchy}

DSDL 데이터 타입 정의는 파일 이름 확장자가 \verb|.uavcan|이고 UTF-8로 인코딩된 텍스트 파일이다.

하나의 파일은 정확히 하나의 버전의 데이터 타입을 정의하며 메이저와 마이너 버전의 각 조합은 데이터 타입 이름에 대해서 유일하게 된다.
서동일한 데이터 타입에 대해서 서로 동시에 임의의 버전 넘버가 가능하며 각 버전은 기껏해야 한 번 정의될 수 있다.
버전 넘버 순서는 연속되는 수일 필요는 없으며 버전 넘버를 건너 뛰거나 가장 오래되거나 가장 최신인 기존 정의만 아니라면 제거도 가능하다.

데이터 타입 정의는 옵션으로 고정된 port-ID\footnote{Chapter~\ref{sec:basic}.} 지정 값이다.

데이터 타입 정의 파일의 이름은 엔트티와 ASCII dot 문자 ``\verb|.|'' (ASCII code 46)로 지정한 순서로 구성된다. :
\begin{itemize}
    \item 10진수의 고정 port-ID, 고정 port-ID가 이 정의에서 제공되는 경우
    \item 데이터 타입의 short name (필수, 공백 허용 하지 않음).
    \item 10진수의 메이저 버전 넘버 (필수)
    \item 10진수의 마이너 버전 넘버 (필수)
    \item 파일 이름 확장자 ``\verb|uavcan|'' (필수).
\end{itemize}

\begin{figure}[H]
    $$
    \overbrace{%
        \underbrace{\texttt{\huge{432}}}_{\substack{\text{fixed} \\ \text{port-ID}}}%
        \texttt{\huge{.}}%
    }^{\text{optional}}%
    \overbrace{%
        \underbrace{\texttt{\huge{GetInfo}}}_{\substack{\text{short name}}}%
        \texttt{\huge{.}}%
        \underbrace{\texttt{\huge{1.0}}}_{\substack{\text{version} \\ \text{numbers}}}%
        \texttt{\huge{.}}%
        \underbrace{\texttt{\huge{uavcan}}}_{\text{file extension}}%
    }^{\text{mandatory}}
    $$
    \caption{Data type definition file name structure\label{fig:dsdl_definition_file_name_structure}}
\end{figure}

DSDL 네임스페이스는 디렉토리를 나타낸다. 하나의 디렉토리는 정확히 하나의 네임스페이스를 정의하며 내부에 다른 디렉토리를 가질 수 있다.
디렉토리의 이름은 데이터 타입 이름 컴포넌트의 이름을 정의한다.
하나의 네임스페이스를 정의하고 있는 하나의 디렉토리는 전체에서의 사용되는 하나의 네임스페이스를 정의한다. 즉 하나의 네임스페이스의 내용은 동일한 이름을 공유하는 다른 디렉토리와 교차로 사용할 수 없다.
하나의 디렉토리는 
nesting\footnote{%
    For example, ``\texttt{foo.bar}'' is not a valid directory name.
    The valid representation would be ``\texttt{bar}'' nested in ``\texttt{foo}''.
}의 한단계 레벨 이상을 정의할 수 없다.

\begin{remark}
    \begin{figure}[H]
        \begin{tabu}{|l|X|} \hline
            \rowfont{\bfseries}
            Directory tree & Entry description \\\hline

            \texttt{vendor\_x/} &
            Root namespace \texttt{vendor\_x}. \\\cline{2-2}

            \texttt{\qquad{}foo/} &
            Nested namespace (also sub-root) \texttt{vendor\_x.foo}. \\\cline{2-2}

            \texttt{\qquad{}\qquad{}100.Run.1.0.uavcan} &
            Data type definition v1.0 with fixed service-ID 100. \\\cline{2-2}

            \texttt{\qquad{}\qquad{}100.Status.1.0.uavcan} &
            Data type definition v1.0 with fixed subject-ID 100. \\\cline{2-2}

            \texttt{\qquad{}\qquad{}ID.1.0.uavcan} &
            Data type definition v1.0 without fixed port-ID. \\\cline{2-2}

            \texttt{\qquad{}\qquad{}ID.1.1.uavcan} &
            Data type definition v1.1 without fixed port-ID. \\\cline{2-2}

            \texttt{\qquad{}\qquad{}bar\_42/} &
            Nested namespace \texttt{vendor\_x.foo.bar\_42}. \\\cline{2-2}

            \texttt{\qquad{}\qquad{}\qquad{}101.List.1.0.uavcan} &
            Data type definition v1.0 with fixed service-ID 101. \\\cline{2-2}

            \texttt{\qquad{}\qquad{}\qquad{}102.List.2.0.uavcan} &
            Data type definition v2.0 with fixed service-ID 102. \\\cline{2-2}

            \texttt{\qquad{}\qquad{}\qquad{}ID.1.0.uavcan} &
            Data type definition v1.0 without fixed port-ID. \\\hline
        \end{tabu}
        \caption{DSDL directory structure example}\label{fig:dsdl_directory_structure_example}
    \end{figure}
\end{remark}

\subsection{Elements of data type definition}\label{sec:dsdl_elements_of_data_type_definition}

A data type definition file contains an exhaustive description of a particular version of the said data type in the
\emph{data structure description language} (DSDL).

A data type definition contains an ordered, possibly empty collection of \emph{field attributes} and/or
unordered, possibly empty collection of \emph{constant attributes}.

A data type may describe either a \emph{structure object} or a \emph{tagged union object}.
The value of a structure object is a function of the values of all of its field attributes.
A tagged union object is formed from at least two field attributes,
but it is capable of holding exactly one field attribute value at any given time.
The value of a tagged union object is a function of which field attribute value
it is holding at the moment and the value of said field attribute.

A field attribute represents a named dynamically assigned value of a statically defined type
that can be exchanged over the network as a member of its containing object.
A padding field attribute is a special kind of field attribute which is used for data alignment purposes;
such field attributes are not named.

A constant attribute represents a named statically defined value of a statically defined type.
Constants are never exchanged over the network, since they are assumed to be known to all involved nodes
by virtue of them sharing compatible definitions of the data type.

Constant values are defined via \emph{DSDL expressions},
which are evaluated at the time of DSDL definition processing.
There is a special category of types called \emph{expression types},
instances of which are used only during expression evaluation
and cannot be exchanged over the network.

Data type definitions can also contain various auxiliary elements reviewed later,
such as deprecation markers (notifying its users that the data type is no longer recommended for new designs)
or assertions (special statements introduced by data type designers
which are statically validated by DSDL processing tools).

Service type definitions are a special case:
they cannot be instantiated or serialized, they do not contain attributes,
and they are composed of exactly two inner data type definitions\footnote{
    A service type can be thought of as a specialized namespace that contains two types and
    has some of the properties of a type, such as name and version.
}.
These inner types are the service request type and the service response type,
separated by the \emph{service response marker}.
They are otherwise ordinary data types except that they are unutterable\footnote{%
    Cannot be referred to. Another commonly used term is ``Voldemort type''.
}
and they derive some of their properties\footnote{Like version numbers or deprecation status.}
from their \emph{parent service type}.

\subsection{Serialization}

Every object that can be exchanged between UAVCAN nodes has a well-defined \emph{serialized representation}.
The value and meaning of an object can be unambiguously recovered from its serialized representation,
provided that the type of the object is known.
Such recovery process is called \emph{deserialization}.

\label{sec:dsdl_bit_length_set}
A serialized representation is a sequence of binary digits (bits);
the number of bits in a serialized representation is called its \emph{bit length}.
A \emph{bit length set} of a data type refers to the set of bit length values of all possible
serialized representations of objects that are instances of the data type.

A data type whose bit length set contains more than one element is said to be \emph{variable length}.
The opposite case is referred to as \emph{fixed length}.

The data type of a serialized message or service object exchanged over the network
is recovered from its subject-ID or service-ID, respectively,
which is attached to the serialized object, along with other metadata, in a manner dictated by the applicable
transport layer specification (chapter~\ref{sec:transport}).
For more information on port identifiers and data type mapping refer to section~\ref{sec:basic_subjects_and_services}.

The bit length set is not defined on service types (only on their request and response types)
because they cannot be instantiated.

\section{Grammar}\label{sec:dsdl_grammar}

이 섹션은 DSDL 문법의 공식 정의를 다룬다.
표기법에 전에 소개하였다.
문법의 각 엘리멘트와 의미는 다음 섹션에서 설명한다.

\subsection{Notation}

다음 정의는 PEG\footnote{Parsing expression grammar.}
table~\ref{table:dsdl_grammar_definition_notation}%
\footnote{%
    Inspired by Parsimonious -- an MIT-licensed software product authored by Erik Rose;
    its sources are available at \url{https://github.com/erikrose/parsimonious}.
}의 표기법을 따른다.
공식 정의에 대한 내용은 넘버 기호(#)으로 시작하는 코멘트를 포함하며 해당 라인의 끝까지 계속된다.

\begin{UAVCANSimpleTable}{Notation used in the formal grammar definition}{|l X|}
    \label{table:dsdl_grammar_definition_notation}
    Pattern & Description \\

    \texttt{"text"} &
    Denotes a terminal string of ASCII characters.
    The string is case-sensitive. \\

    \emph{(space)} &
    Concatenation.
    E.g., \texttt{korovan paukan excavator} matches a sequence where the specified tokens
    appear in the defined order. \\

    \texttt{abc / ijk / xyz} &
    Alternatives.
    The leftmost matching alternative is accepted. \\

    \texttt{abc?} &
    Optional greedy match. \\

    \texttt{abc*} &
    Zero or more expressions, greedy match. \\

    \texttt{abc+} &
    One or more expressions, greedy match. \\

    \texttt{\textasciitilde{}r"regex"} &
    An IEEE POSIX Extended Regular Expression pattern defined between the double quotes.
    The expression operates on the ASCII character set and is always case-sensitive.
    ASCII escape sequences ``\texttt{\textbackslash{}r}'', ``\texttt{\textbackslash{}n}'', and
    ``\texttt{\textbackslash{}t}'' are used to denote ASCII carriage return (code 13),
    line feed (code 10), and tabulation (code 9) characters, respectively. \\

    \texttt{\textasciitilde{}r'regex'} &
    As above, with single quotes instead of double quotes. \\

    \texttt{(abc xyz)} &
    Parentheses are used for grouping. \\
\end{UAVCANSimpleTable}

\subsection{Definition}

At the top level, a DSDL definition file is an ordered collection of statements;
the order is determined by the relative placement of statements inside the DSDL source file:
statements located closer the beginning of the file precede those that are located closer to the end of the file.

From the top level down to the expression rule, the grammar is a valid regular grammar,
meaning that it can be parsed using standard regular expressions.

The grammar definition provided here assumes lexerless parsing;
that is, it applies directly to the unprocessed source text of the definition.

All characters used in the definition belong to the ASCII character set.

\clearpage\inputminted[fontsize=\scriptsize]{python}{dsdl/grammar.parsimonious}

\subsection{Expressions}

Symbols representing operators belong to the ASCII (basic Latin) character set.

Operators of the same precedence level are evaluated from left to right.

The attribute reference operator is a special case: it is defined for an instance of any type
on its left side and an attribute identifier on its right side.
The concept of ``attribute identifier'' is not otherwise manifested in the type system.
The attribute reference operator is not explicitly documented for any data type;
instead, the documentation specifies the set of available attributes for instances of said type,
if there are any.

\begin{UAVCANSimpleTable}{Unary operators}{|l l X|}
    Symbol                             & Precedence & Description \\
    \texttt{\textbf{+}}                         & 3 & Unary plus \\
    \texttt{\textbf{-}} (hyphen-minus)          & 3 & Unary minus \\
    \texttt{\textbf{!}}                         & 8 & Logical not \\
\end{UAVCANSimpleTable}

\begin{UAVCANSimpleTable}{Binary operators}{|l l X|}
    Symbol                                          & Precedence & Description \\
    \texttt{\textbf{.}} (full stop)                          & 1 & Attribute reference
                                                                   (parent object on the left side,
                                                                   attribute identifier on the right side) \\

    \texttt{\textbf{**}}                                     & 2 & Exponentiation
                                                                   (base on the left side, power on the right side) \\

    \texttt{\textbf{*}}                                      & 4 & Multiplication \\
    \texttt{\textbf{/}}                                      & 4 & Division \\
    \texttt{\textbf{\%}}                                     & 4 & Modulo \\

    \texttt{\textbf{+}}                                      & 5 & Addition \\
    \texttt{\textbf{-}} (hyphen-minus)                       & 5 & Subtraction \\

    \texttt{\textbf{|}} (vertical line)                      & 6 & Bitwise or \\
    \texttt{\textbf{\textasciicircum{}}} (circumflex accent) & 6 & Bitwise xor \\
    \texttt{\textbf{\&}}                                     & 6 & Bitwise and \\

    \texttt{\textbf{==}} (dual equals sign)                  & 7 & Equality \\
    \texttt{\textbf{!=}}                                     & 7 & Inequality \\
    \texttt{\textbf{<=}}                                     & 7 & Less or equal \\
    \texttt{\textbf{>=}}                                     & 7 & Greater or equal \\
    \texttt{\textbf{<}}                                      & 7 & Less \\
    \texttt{\textbf{>}}                                      & 7 & Greater \\

    \texttt{\textbf{||}} (dual vertical line)                & 9 & Logical or \\
    \texttt{\textbf{\&\&}}                                   & 9 & Logical and \\
\end{UAVCANSimpleTable}

\subsection{Literals}

Upon its evaluation, a literal yields an object of a particular type depending on the syntax of the literal,
as specified in this section.

\subsubsection{Boolean literals}

A boolean literal is denoted by the keyword ``\verb|true|'' or ``\verb|false|''
represented by an instance of primitive type ``\verb|bool|'' (section~\ref{sec:dsdl_primitive_types})
with an appropriate value.

\subsubsection{Numeric literals}

Integer and real literals are represented as instances of type ``\verb|rational|'' (section~\ref{sec:dsdl_rational}).

The digit separator character ``\verb|_|'' (underscore) does not affect the interpretation of numeric literals.

The significand of a real literal is formed by the integer part, the optional decimal point,
and the optional fraction part;
either the integer part or the fraction part (not both) can be omitted.
The exponent is optionally specified after the letter ``\verb|e|'' or ``\verb|E|'';
it indicates the power of 10 by which the significand is to be scaled.
Either the decimal point or the letter ``\verb|e|''/``\verb|E|'' with the following exponent
(not both) can be omitted from a real literal.

\begin{remark}
    An integer literal \verb|0x123| is represented internally as $\frac{291}{1}$.

    A real literal \verb|.3141592653589793e+1| is represented internally as
    $\frac{3141592653589793}{1000000000000000}$.
\end{remark}

\subsubsection{String literals}

String literals are represented as instances of type ``\verb|string|'' (section~\ref{sec:dsdl_string}).

A string literal is allowed to contain an arbitrary sequence of Unicode characters,
excepting escape sequences defined in table~\ref{table:dsdl_string_literal_escape}
which shall follow one of the specified therein forms.
An escape sequence begins with the ASCII backslash character ``\verb|\|''.

\begin{UAVCANSimpleTable}{String literal escape sequences}{|l X|}
    Sequence & Interpretation
    \label{table:dsdl_string_literal_escape} \\

    \texttt{\textbackslash{}\textbackslash{}}   & Backslash, ASCII code 92. Same as the escape character. \\
    \texttt{\textbackslash{}r}                  & Carriage return, ASCII code 13.               \\
    \texttt{\textbackslash{}n}                  & Line feed, ASCII code 10.                     \\
    \texttt{\textbackslash{}t}                  & Horizontal tabulation, ASCII code 9.          \\

    \texttt{\textbackslash{}\textquotesingle{}} &
    Apostrophe (single quote), ASCII code 39. Regardless of the type of quotes around the literal. \\

    \texttt{\textbackslash{}\textquotedbl{}}    &
    Quotation mark (double quote), ASCII code 34. Regardless of the type of quotes around the literal. \\

    \texttt{\textbackslash{}u????} &
    Unicode symbol with the code point specified by a four-digit hexadecimal number.
    The placeholder ``\texttt{?}'' represents a hexadecimal character \texttt{[0-9a-fA-F]}. \\

    \texttt{\textbackslash{}U????????} &
    Like above, the code point is specified by an eight-digit hexadecimal number. \\

\end{UAVCANSimpleTable}

\begin{remark}
    \begin{minted}{python}
        @assert "oh,\u0020hi\U0000000aMark" == 'oh, hi\nMark'
    \end{minted}
\end{remark}

\subsubsection{Set literals}

Set literals are represented as instances of type ``\verb|set|'' (section~\ref{sec:dsdl_set})
parameterized by the type of the contained elements which is determined automatically.

A set literal declaration shall specify at least one element,
which is used to determine the element type of the set.

The elements of a set literal are defined as DSDL expressions which are evaluated before a set is constructed
from the corresponding literal.

\begin{remark}
    \begin{minted}{python}
        @assert {"cells", 'interlinked'} == {"inter" + "linked", 'cells'}
    \end{minted}
\end{remark}

\subsection{Reserved identifiers}\label{sec:dsdl_reserved_identifiers}

DSDL identifiers and data type name components that match any of the
case-insensitive patterns specified in table~\ref{table:dsdl_reserved_word_patterns}
cannot be used to name new entities.
The semantics of such identifiers is predefined by the DSDL specification,
and as such, they cannot be used for other purposes.
Some of the reserved identifiers do not have any functions associated with them
in this version of the DSDL specification, but this may change in the future.

\begin{UAVCANSimpleTable}{Reserved identifier patterns (POSIX ERE notation, ASCII character set, case-insensitive)}%
    {|l l X|}%
    \label{table:dsdl_reserved_word_patterns}%
    POSIX ERE ASCII pattern                            & Example            & Special meaning \\
    \texttt{truncated}                                 &                    & Cast mode specifier \\
    \texttt{saturated}                                 &                    & Cast mode specifier \\
    \texttt{true}                                      &                    & Boolean literal \\
    \texttt{false}                                     &                    & Boolean literal \\
    \texttt{bool}                                      &                    & Primitive type category \\
    \texttt{u?int\textbackslash{}d*}                   & \texttt{uint8}     & Primitive type category \\
    \texttt{float\textbackslash{}d*}                   & \texttt{float}     & Primitive type category \\
    \texttt{u?q\textbackslash{}d+\_\textbackslash{}d+} & \texttt{q16\_8}    & Primitive type category (future) \\
    \texttt{void\textbackslash{}d*}                    & \texttt{void}      & Void type category \\
    \texttt{optional}                                  &                    & Reserved for future use \\
    \texttt{aligned}                                   &                    & Reserved for future use \\
    \texttt{const}                                     &                    & Reserved for future use \\
    \texttt{struct}                                    &                    & Reserved for future use \\
    \texttt{super}                                     &                    & Reserved for future use \\
    \texttt{template}                                  &                    & Reserved for future use \\
    \texttt{enum}                                      &                    & Reserved for future use \\
    \texttt{self}                                      &                    & Reserved for future use \\
    \texttt{and}                                       &                    & Reserved for future use \\
    \texttt{or}                                        &                    & Reserved for future use \\
    \texttt{not}                                       &                    & Reserved for future use \\
    \texttt{auto}                                      &                    & Reserved for future use \\
    \texttt{type}                                      &                    & Reserved for future use \\
    \texttt{con}                                       &                    & Compatibility with Microsoft Windows \\
    \texttt{prn}                                       &                    & Compatibility with Microsoft Windows \\
    \texttt{aux}                                       &                    & Compatibility with Microsoft Windows \\
    \texttt{nul}                                       &                    & Compatibility with Microsoft Windows \\
    \texttt{com\textbackslash{}d}                      & \texttt{com1}      & Compatibility with Microsoft Windows \\
    \texttt{lpt\textbackslash{}d}                      & \texttt{lpt9}      & Compatibility with Microsoft Windows \\
    \texttt{\_.*\_}                                    & \texttt{\_offset\_}& Special-purpose intrinsic entities \\
\end{UAVCANSimpleTable}

\input{dsdl/expression_types.tex}
\input{dsdl/serializable_types.tex}
\section{Attributes}\label{sec:dsdl_attributes}

\emph{attribute}는 특정 객체나 타입과 관련된 이름이 붙여진 엔트티이다.(패딩 필드는 제외)

\subsection{Composite type attributes}

데이터 타입 정의 시간에 할당되는 값을 가지는 컴포지트 타입 속성을 \emph{constant attribute}라고 부른다.
데이터 타입 정의 시간에 할당되는 않는 컴포지트 타입 속성을 \emph{field attribute}라고 부른다.

컴포지트 타입 속성의 이름은 이를 포함하는 데이터 타입 정의 내에서는 유일한 이름이 되며,
table
\ref{table:dsdl_reserved_word_patterns}에서 정의한 예약 이름 패턴과 매칭되지 않는다.
이런 요구사항은 패딩 필드에 적용되지 않는다.

\subsubsection{Field attributes}

필드 속성은 정적으로 정의된 타입의 동적으로 할당된 값을 나타내며 포함하고 있는 객체의 하나의 멤버로 네트워크 상에서 교환이 가능하다.
필드 속성의 데이터 타입은 직렬화 가능한 타입의 카테고리(section~\ref{sec:dsdl_serializable_types})가 될 수 있으며,
void 타입 카테고리는 허용하지 않아서 제외된다.

필드 속성의 특수한 종류--- \emph{padding fields}에 예외가 적용된다.
패딩 필드 속성의 타입은 void 카테고리의 타입이 된다.
패딩 필드 속성은 이름을 갖지 않을 것이다.

필드 속성 한쌍이 동일하다고 판단되는 경우는 양쪽 모두 동일한 속성이고 모두 동일한 이름 이거나 패딩 필드 속성인 경우이다.

\begin{remark}
    Example:
    \begin{minted}{python}
        uint8[<=10] regular_field   # A field named "regular field"
        void16                      # A padding field; no name is permitted
    \end{minted}
\end{remark}

\subsubsection{Constant attributes}

상수 속성은 정적으로 정의된 타입에 이름을 갖는 정적으로 할당된 값을 표현한다.
상수 속성의 값은 네트워크 상에서 절대로 교환할 수 없다.
왜냐하면 데이터 타입의 동일한 정의를 공유하는 방식으로 관련된 모든 노드들이 모두 사전에 알고 있다고 가정한다.

상수 속성의 데이터 타입은 원시 타입이된다.
카테고리
(section~\ref{sec:dsdl_serializable_types})

상수 속성의 값은 \emph{initialization expression}를 evaluation으로 DSDL 정의 처리 시점에 결정된다.

구문은 상수 속성의 값을 초기화하기 위해서 evaluation에 따라서 호환되는 타입을 생성한다.
호환 타입의 집합은 초기화되는 상수 속성의 타입에 의존하며 table~\ref{table:dsdl_constant_init_pattern}에서 지정되어 있다.

\begin{UAVCANSimpleTable}[wide]{Permitted constant attribute value initialization patterns}{|l | X | X[2] | X[2]|}
    \diagbox[font=\footnotesize]{Constant\\type\\category}{Expression\\type} &
    \texttt{bool} & \texttt{rational} & \texttt{string} \\

    \textbf{Boolean} &
    Allowed. &
    Not allowed. &
    Not allowed. \\

    \textbf{Integer} &
    Not allowed. &
    Allowed if the denominator equals one and the numerator value is within the range of the constant type. &
    Allowed if the target type is \texttt{uint8} and the source string contains one symbol whose code point falls
    into the range $[0, 127]$. \\

    \textbf{Floating point} &
    Not allowed. &
    Allowed if the source value does not exceed the finite range of the constant type.
    The final value is computed as the quotient of the numerator and the denominator
    with implementation-defined accuracy. &
    Not allowed. \label{table:dsdl_constant_init_pattern}\\

\end{UAVCANSimpleTable}

데이터 타입 정의 시점에 정의한 상수 속성의 값으로 인해,
원시 타입으로 된 상수의 캐스트 모드는 관찰 영향을 가지고 있지 않다.

\begin{remark}
    A real literal \verb|1234.5678| is represented internally as
    $\frac{6172839}{5000}$, which can be used to initialize a \verb|float16| value,
    resulting in $1235.0$.

    The specification states that the value of a floating-point constant should be computed
    with an implementation-defined accuracy. UAVCAN avoids strict accuracy requirements in order to
    ensure compatibility with implementations that rely on non-standard floating point formats.
    Such laxity in the specification is considered acceptable since the uncertainty is always
    confined to a single division expression per constant; all preceding computations, if any,
    are always performed by the DSDL compiler using exact rational arithmetic.
\end{remark}

\subsection{Local attributes}\label{sec:dsdl_local_attributes}

로컬 속성은 DSDL 정의 처리 시간에 유효하다.
section~\ref{sec:dsdl_grammar}에 정의된 것과 같이,
DSDL 정의는 순서를 가지는 구문들을 모아둔 것이다.
구문 넘버 $E$ 에 포함된 expression은 이름으로 구문 넘버 $A$ 에서 말한 컴포지트 타입 속성을 뜻한다. 여기서 $A < E$ 와 양쪽 구문이 동일한 데이터 타입 definition\footnote{
    Per \ref{sec:dsdl_elements_of_data_type_definition},
    in case of services, this applies only to their request and response types.
}에 속한다.
참조한 DSDL expression의 컨텍스트에서 참조한 속성의 표현은 table~\ref{table:dsdl_local_attribute_representation}에 지정되어 있다.

\begin{UAVCANSimpleTable}{Local attribute representation}{|l X X|}\label{table:dsdl_local_attribute_representation}%
    Attribute category & Value type & Value \\

    Constant attribute &
    Type of the constant attribute &
    Value of the constant attribute \\

%    Field attribute &
%    \texttt{metaserializable} &
%    Type of the field attribute \\
    Field attribute &
    Illegal &
    Illegal \\

\end{UAVCANSimpleTable}

\begin{remark}
    \begin{minted}{python}
        uint8 FOO = 123
        uint16 BAR = FOO ** 2
        @assert BAR == 15129
        ---  # The request type ends here; its attributes are no longer accessible.
        #uint16 BAZ = BAR  # Would fail - BAR is not accessible here.
        float64 FOO = 3.14
        @assert FOO == 3.14
    \end{minted}
\end{remark}

\subsection{Intrinsic attributes}

고유한 속성은 어떤 expression에서 든 유효하다.
이 값들은 컨텍스트에 따라서 DSDL 처리 툴에 의해서 생성되며,
이번 섹션에서 다룬다.

\subsubsection{Offset attribute}

오프셋 속성은 식별자 ``\verb|_offset_|''를 뜻한다.
이 값은 type $\texttt{set}_\texttt{<rational>}$의 값이다.

다음 텍스트에서 용어 \emph{referring statement}은 오프셋 속성을 참조하는 expression을 포함하는 구문을 뜻한다.
용어 \emph{bit length set}은 section~\ref{sec:dsdl_bit_length_set}에 정의되어 있다.

속성의 값은 필드 속성 선언의 기능으로 참조 구문 앞에 위치하고 포함하는 정의의 카테고리이다.

현재 정의가 태그된 union 카테고리에 속한다면,
참조 구문은 최종 필드 속성 정의 뒤에 위치하게 된다.
참조 구문에 이후에 있는 필드 속성 정의는 현재 정의가 유효하지 않다고 판단하게 된다.
태그 unions에 대해서 오프셋 속성의 값은 unions 필드의 누적 bit length set\footnote{Section \ref{sec:dsdl_composite_alignment_cumulative_bls}}로 정의한다. 여기서 해당 집합의 각 엘리멘트는 내포된 union 태그 필드의(section \ref{sec:dsdl_serialization_composite}) bit length만큼 증가시킨다.

현재 데이터 정의는 태그된 union 카테고리에 속하지 않지만,
참조 구문은 현재 정의 내부에 어떤 곳이든 위치할 수 있다.
오프셋 속성의 값은 참조 구문(see section~\ref{sec:dsdl_grammar} on statement ordering) 앞에 있는 구문에서 필드의 누적 bit length set\footnote{Section \ref{sec:dsdl_composite_alignment_cumulative_bls}.}로 정의된다.

\begin{remark}
    \begin{minted}{python}
        @union
        uint8 a
        #@print _offset_  # Would fail: it's a tagged union, _offset_ is undefined until after the last field
        uint16 b
        @assert _offset_ == {8 + 8,  8 + 16}
        ---
        @assert _offset_ == {0}
        float16 a
        @assert _offset_ == {16}
        void4
        @assert _offset_ == {20}
        int4 b
        @assert _offset_ == {24}
        uint8[<4] c
        @assert _offset_ == 8 + {24,  32,  40,  48}
        @assert _offset_ % 8 == {0}
        # One of the main usages for _offset_ is statically proving that the following field is byte-aligned
        # for all possible valid serialized representations of the preceding fields. It is done by computing
        # a remainder as shown above. If the field is aligned, the remainder set will equal {0}. If the
        # remainder set contains other elements, the field may be misaligned under some circumstances.
        # If the remainder set does not contain zero, the field is never aligned.
        uint8 well_aligned   # Proven to be byte-aligned.
    \end{minted}
\end{remark}

\input{dsdl/directives.tex}
\section{Data serialization}\label{sec:dsdl_data_serialization}

\newcommand{\hugett}[1]{\texttt{\huge{#1}}}

\subsection{General principles}

\subsubsection{Design goals}

이 섹션에 설명하는 직렬화 표현에 대한 주요 디자인 원칙은 현재와 향후 컴퓨터 아키텍쳐에서 사용할 네이티브 표현과 최대한 호환되도록 하는 것이다.
DSDL에서 정의하는 직렬화 표현이 현대 컴퓨터의 내부 데이터 표현과 매칭되도록 하는 것이 목표이다. 따라서 이상적으로는 UAVCAN 네트워크 상에서 데이터 교환하는 동안 데이터 변환 수행이 필요없도록 하는 것이다.

이 섹션에서 소개하는 절삭 및 zero 확장 원칙은 구조적 서브타이핑을 촉진하고 하위 호환을 위해서 데이터 타입의 확장성을 가능하게 하기 위해서 설계되었다.
이는 런타임 타입 체크와 장기 안정성 보장 사이의 교환이 발생한다.
이 모델에서는 데이터 타입 호환성이 런타임이 아닌 정적으로 결정된다고 가정한다.

\subsubsection{Bit and byte ordering}

가장 작은 아토믹 데이터 엔티트는 비트(bit)이다.
8개 비트가 하나의 바이트를 형성한다:
바이트 내부에서 비트들은 순서를 가지는데 LSB(least significant bit)가 첫번째(0-th index)오고 MSB(most significant bit)가 마지막에(7-th index)에 온다.

여러 바이트들로 구성되는 숫자 값은 LSB가 먼저 오도록 인코딩하여 정렬한다. 
이런 포맷을 리틀-인디언(little-endian)이라고 한다.

\begin{figure}[H]
    $$
    \overset{\text{bit index}}{%
        \underbrace{%
            \overset{\text{M}}{\overset{7}{\hugett{0}}}
            \overset{6}{\hugett{1}}
            \overset{5}{\hugett{0}}
            \overset{4}{\hugett{1}}
            \overset{3}{\hugett{0}}
            \overset{2}{\hugett{1}}
            \overset{1}{\hugett{0}}
            \overset{\text{L}}{\overset{0}{\hugett{1}}}
        }_\text{least significant byte}%
    }
    \hugett{\ldots}
    \overset{\text{bit index}}{%
        \underbrace{%
            \overset{\text{M}}{\overset{7}{\hugett{0}}}
            \overset{6}{\hugett{1}}
            \overset{5}{\hugett{0}}
            \overset{4}{\hugett{1}}
            \overset{3}{\hugett{0}}
            \overset{2}{\hugett{1}}
            \overset{1}{\hugett{0}}
            \overset{\text{L}}{\overset{0}{\hugett{1}}}
        }_\text{most significant byte}%
    }
    $$
    \caption{Bit and byte ordering\label{fig:dsdl_serialization_bit_ordering}}
\end{figure}

\subsubsection{Implicit truncation of excessive data}\label{sec:dsdl_serialization_implicit_truncation}

직렬화 시킨 표현을 반직렬화할때에 구현에서 사용하지 않는 데이터나 패딩 비트들은 반직렬화에\footnote{%
The presence of unused data should not be considered an error.
} 따라서 그대로 남겨둔다.
직렬화 표현의 전체 크기는 기반 트랜스포트 계층이나 네스티드 객체의 경우에는 \emph{delimiter header}
(section \ref{sec:dsdl_serialization_composite_non_sealed})가 알려준다.

위 요구사항에 따라서 트랜스포트 계층은 직렬화 표현의 마지막에 데이터 크기 제약을 만족시키기 위해서 0 패딩 비트를 추가할 수 있게 된다.
0이 아닌 패딩 비트는 허용하지 않는다. \footnote{%
    Because padding bits may be misinterpreted as part of the serialized representation.
}

\begin{remark}
    Because of implicit truncation a serialized representation constructed from an instance of type $B$ can be
    deserialized into an instance of type $A$ as long as $B$ is a structural subtype of $A$.

    Let $x$ be an instance of data type $B$, which is defined as follows:

    \begin{minted}{python}
        float32 parameter
        float32 variance
    \end{minted}

    Let $A$ be a structural supertype of $B$, being defined as follows:

    \begin{minted}{python}
        float32 parameter
    \end{minted}

    Then the serialized representation of $x$ can be deserialized into an instance of $A$.
    The topic of data type compatibility is explored in detail in section~\ref{sec:dsdl_versioning}.
\end{remark}

\subsubsection{Implicit zero extension of missing data}\label{sec:dsdl_serialization_implicit_zero_extension}

역직렬화 루틴의 목적을 위해
데이터 타입의 인스턴스의 직렬화된 표현은 (0).\footnote{%
    This can be implemented by checking for out-of-bounds access during deserialization and returning zeros
    if an out-of-bounds access is detected. This is where the name ``implicit zero extension rule'' is derived
    from.
}으로 된 무한 연속 비트로 암묵적으로 끝이 난다.
For the purposes of deserialization routines,
the serialized representation of any instance of a data type shall \emph{implicitly} end with an
infinite sequence of bits with a value of zero (0).\footnote{%
    This can be implemented by checking for out-of-bounds access during deserialization and returning zeros
    if an out-of-bounds access is detected. This is where the name ``implicit zero extension rule'' is derived
    from.
}.

Despite this rule, implementations are not allowed to intentionally truncate trailing zeros
upon construction of a serialized representation of an object\footnote{%
    Intentional truncation is prohibited because a future revision of the specification may remove the implicit zero
    extension rule.
    If intentional truncation were allowed, removal of this rule would break backward compatibility.
}.

The total size of the serialized representation is reported either by the underlying transport layer, or,
in the case of nested objects, by the \emph{delimiter header}
(section \ref{sec:dsdl_serialization_composite_non_sealed}).

\begin{remark}
    The implicit zero extension rule enables extension of data types by introducing additional fields
    without breaking backward compatibility with existing deployments.
    The topic of data type compatibility is explored in detail in section~\ref{sec:dsdl_versioning}.

    The following example assumes that the reader is familiar with the variable-length array serialization rules,
    explained in section~\ref{sec:dsdl_serialized_variable_length_array}.

    Let the data type $A$ be defined as follows:

    \begin{minted}{python}
        uint8 scalar
    \end{minted}

    Let $x$ be an instance of $A$, where the value of \verb|scalar| is 4.
    Let the data type $B$ be defined as follows:

    \begin{minted}{python}
        uint8[<256] array
    \end{minted}

    Then the serialized representation of $x$ can be deserialized into an instance of $B$ where the field
    \verb|array| contains a sequence of four zeros: $0, 0, 0, 0$.
\end{remark}

\subsubsection{Error handling}\label{sec:dsdl_serialized_error}

In this section and further, an object that nests other objects is referred to as an \emph{outer object}
in relation to the nested object.

Correct UAVCAN types shall have no serialization error states.

A deserialization process may encounter a serialized representation that does not belong to the
set of serialized representations of the data type at hand.
In such case, the invalid serialized representation shall be discarded and the implementation
shall explicitly report its inability to complete the deserialization process for the given input.
Correct UAVCAN types shall have no other deserialization error states.

Failure to deserialize a nested object renders the outer object invalid\footnote{%
    Therefore, failure in a single deeply nested object propagates upward, rendering the entire structure invalid.
    The motivation for such behavior is that it is likely that if an inner object cannot be deserialized,
    then the outer object is likely to be also invalid.
}.

\subsection{Void types}\label{sec:dsdl_serialized_void}

The serialized representation of a void-typed field attribute is constructed as a sequence of zero bits.
The length of the sequence equals the numeric suffix of the type name.

When a void-typed field attribute is deserialized, the values of respective bits are ignored;
in other words, any bit sequence of correct length is a valid serialized representation
of a void-typed field attribute.
This behavior facilitates usage of void fields as placeholders for non-void fields
introduced in newer versions of the data type (section~\ref{sec:dsdl_versioning}).

\begin{remark}
    The following data type will be serialized as a sequence of three zero bits $000_2$:
    \begin{minted}{python}
        void3
    \end{minted}
    The following bit sequences are valid serialized representations of the type:
    $000_2$,
    $001_2$,
    $010_2$,
    $011_2$,
    $100_2$,
    $101_2$,
    $110_2$,
    $111_2$.

    Shall the padding field be replaced with a non-void-typed field in a future version of the data type,
    nodes utilizing the newer definition may be able to retain compatibility with nodes using older types,
    since the specification guarantees that padding fields are always initialized with zeros:

    \begin{minted}{python}
        # Version 1.1
        float64 a
        void64
    \end{minted}

    \begin{minted}{python}
        # Version 1.2
        float64 a
        float32 b  # Messages v1.1 will be interpreted such that b = 0.0
        void32
    \end{minted}
\end{remark}

\subsection{Primitive types}

\subsubsection{General principles}

Implementations where native data formats are incompatible with those adopted by UAVCAN shall perform
conversions between the native formats and the corresponding UAVCAN formats during
serialization and deserialization.
Implementations shall avoid or minimize information loss and/or distortion caused by such conversions.

Serialized representations of instances of the primitive type category that are longer than one byte (8 bits)
are constructed as follows.
First, only the least significant bytes that contain the used bits of the value are preserved;
the rest are discarded following the lossy assignment policy selected by the specified cast mode.
Then the bytes are arranged in the least-significant-byte-first order\footnote{Also known as ``little endian''.}.
If the bit width of the value is not an integer multiple of eight (8) then the next value in the type will begin
starting with the next bit in the current byte. If there are no further values then the remaining bits
shall be zero (0).

\begin{remark}
    The value $1110\,1101\,1010_2$ (3802 in base-10) of type \verb|uint12| is encoded as follows.
    The bit sequence is shown in the base-2 system, where bytes (octets) are comma-separated:
    $$
        \overset{\text{byte 0}}{%
            \underbrace{%
                \overset{7}{\hugett{1}}
                \overset{6}{\hugett{1}}
                \overset{5}{\hugett{0}}
                \overset{4}{\hugett{1}}
                \overset{3}{\hugett{1}}
                \overset{2}{\hugett{0}}
                \overset{1}{\hugett{1}}
                \overset{0}{\hugett{0}}
            }_{\substack{\text{Least significant 8} \\ \text{bits of }3802_{10}}}%
        }%
        \hugett{,}%
        \overset{\text{byte 1}}{%
            \underbrace{
                \overset{7}{\hugett{?}}
                \overset{6}{\hugett{?}}
                \overset{5}{\hugett{?}}
                \overset{4}{\hugett{?}}
            }_{\substack{\text{Next object} \\ \text{or zero} \\ \text{padding bits}}}%
            \underbrace{
                \overset{3}{\hugett{1}}
                \overset{2}{\hugett{1}}
                \overset{1}{\hugett{1}}
                \overset{0}{\hugett{0}}
            }_{\substack{\text{Most} \\ \text{significant} \\ \text{4 bits of} \\ \text{3802}_{10}}}%
        }
    $$
\end{remark}

\subsubsection{Boolean types}\label{sec:dsdl_serialized_bool}

The serialized representation of a value of type \verb|bool| is a single bit.
If the value represents falsity, the value of the bit is zero (0); otherwise, the value of the bit is one (1).

\subsubsection{Unsigned integer types}\label{sec:dsdl_serialized_unsigned_integer}

The serialized representation of an unsigned integer value of length $n$ bits
(which is reflected in the numerical suffix of the data type name)
is constructed as if the number were to be written in base-2 numerical system
with leading zeros preserved so that the total number of binary digits would equal $n$.

\begin{remark}
    The serialized representation of integer 42 of type \verb|uint7| is $0101010_2$.
\end{remark}

\subsubsection{Signed integer types}

The serialized representation of a non-negative value of a signed integer type is constructed as described
in section~\ref{sec:dsdl_serialized_unsigned_integer}.

The serialized representation of a negative value of a signed integer type is computed by
applying the following transformation:
$$2^n + x$$
where $n$ is the bit length of the serialized representation
(which is reflected in the numerical suffix of the data type name)
and $x$ is the value whose serialized representation is being constructed.
The result of the transformation is a positive number,
whose serialized representation is then constructed as described in section~\ref{sec:dsdl_serialized_unsigned_integer}.

The representation described here is widely known as \emph{two's complement}.

\begin{remark}
    The serialized representation of integer -42 of type \verb|int7| is $1010110_2$.
\end{remark}

\subsubsection{Floating point types}

The serialized representation of floating point types follows the IEEE 754 series of standards as follows:

\begin{itemize}
    \item \verb|float16| --- IEEE 754 binary16;
    \item \verb|float32| --- IEEE 754 binary32;
    \item \verb|float64| --- IEEE 754 binary64.
\end{itemize}

Implementations that model real numbers using any method other than IEEE 754 shall be able to model
positive infinity, negative infinity, signaling NaN\footnote{%
    Per the IEEE 754 standard, NaN stands for
    ``not-a-number'' -- a set of special bit patterns that represent lack of a meaningful value.
}, and quiet NaN.

\subsection{Array types}

\subsubsection{Fixed-length array types}

Serialized representations of a fixed-length array of $n$ elements of type $T$ and
a sequence of $n$ field attributes of type $T$ are equivalent.

\begin{remark}
    Serialized representations of the following two data type definitions are equivalent:

    \begin{minted}{python}
        AnyType[3] array
    \end{minted}

    \begin{minted}{python}
        AnyType item_0
        AnyType item_1
        AnyType item_2
    \end{minted}
\end{remark}

\subsubsection{Variable-length array types}\label{sec:dsdl_serialized_variable_length_array}

A serialized representation of a variable-length array consists of two segments:
the implicit length field immediately followed by the array elements.

The implicit length field is of an unsigned integer type.
The serialized representation of the implicit length field
is injected in the beginning of the serialized representation of its array.
The bit length of the unsigned integer value is first determined as follows:

$$b=\lceil{}\log_2 (c + 1)\rceil{}$$

where $c$ is the capacity (i.e., the maximum number of elements) of the variable-length array and
$b$ is the minimum number of bits needed to encode $c$ as an unsigned integer. An additional transformation
of $b$ ensures byte alignment of this implicit field when serialized\footnote{Future updates to the specification
may allow this second step to be modified but the default action will always be to byte-align the implicit
length field.}:

$$2^{\lceil{}\log_2 (\text{max}(8, b))\rceil{}}$$

The number of elements $n$ contained in the variable-length array is encoded
in the serialized representation of the implicit length field
as described in section~\ref{sec:dsdl_serialized_unsigned_integer}.
By definition, $n \leq c$; therefore, bit sequences where the implicit length field contains values
greater than $c$ do not belong to the set of serialized representations of the array.

The rest of the serialized representation is constructed as if the variable-length array was
a fixed-length array of $n$ elements\footnote{%
    Observe that the implicit array length field, per its definition,
    is guaranteed to never break the alignment of the following array elements.
    There may be no padding between the implicit array length field and its elements.
}.

\begin{remark}
    Data type authors must take into account that variable-length arrays with a capacity of $\leq{}255$ elements will
    consume an additional 8 bits of the serialized representation
    (where a capacity of $\leq 65535$ will consume 16 bits and so on).
    For example:

    \begin{minted}{python}
        uint8 first
        uint8[<=6] second              # The implicit length field is 8 bits wide
        @assert _offset_.max / 8 <= 7  # This would fail.
    \end{minted}

    In the above example the author attempted to fit the message into a single Classic CAN frame but
    did not account for the implicit length field. The correct version would be:

    \begin{minted}{python}
        uint8 first
        uint8[<=5] second              # The implicit length field is 8 bits wide
        @assert _offset_.max / 8 <= 7  # This would pass.
    \end{minted}

    If the array contained three elements, the resulting set of its serialized representations would
    be equivalent to that of the following definition:

    \begin{minted}{python}
        uint8 first
        uint8 implicit_length_field  # Set to 3, because the array contains three elements
        uint8 item_0
        uint8 item_1
        uint8 item_2
    \end{minted}
\end{remark}

\subsection{Composite types}\label{sec:dsdl_serialization_composite}

\subsubsection{Sealed structure}

A serialized representation of an object of a sealed composite type that is not a tagged union
is a sequence of serialized representations of its field attribute values joined into a bit sequence,
separated by padding if such is necessary to satisfy the alignment requirements.
The ordering of the serialized representations of the field attribute values follows the order
of field attribute declaration.

\begin{remark}
    Consider the following definition,
    where the fields are assigned runtime values shown in the comments:

    \begin{minted}{python}
        #                          decimal           bit sequence   comment
        truncated uint12 first   # +48858     1011_1110_1101_1010   overflow, MSB truncated
        saturated  int3  second  #     -1                     111   two's complement
        saturated  int4  third   #     -5                    1011   two's complement
        saturated  int2  fourth  #     -1                      11   two's complement
        truncated uint4  fifth   #   +136               1000_1000   overflow, MSB truncated
        @sealed
    \end{minted}

    It can be seen that the bit layout is rather complicated because the field boundaries do not align with byte
    boundaries, which makes it a good case study.
    The resulting serialized byte sequence is shown below in the base-2 system:
    $$
        \underbrace{%
            \overbrace{%
                \underset{7}{\overset{7}{\hugett{1}}}%
                \underset{6}{\overset{6}{\hugett{1}}}%
                \underset{5}{\overset{5}{\hugett{0}}}%
                \underset{4}{\overset{4}{\hugett{1}}}%
                \underset{3}{\overset{3}{\hugett{1}}}%
                \underset{2}{\overset{2}{\hugett{0}}}%
                \underset{1}{\overset{1}{\hugett{1}}}%
                \underset{0}{\overset{0}{\hugett{0}}}%
            }^{\texttt{first}}%
        }_{\texttt{byte 0}}%
        \hugett{,}%
        \underbrace{%
            \overbrace{%
                \underset{7}{\overset{0}{\hugett{1}}}%
            }^{\texttt{third}}%
            \overbrace{%
                \underset{6}{\overset{2}{\hugett{1}}}%
                \underset{5}{\overset{1}{\hugett{1}}}%
                \underset{4}{\overset{0}{\hugett{1}}}%
            }^{\texttt{second}}%
            \overbrace{%
                \underset{3}{\overset{11}{\hugett{1}}}%
                \underset{2}{\overset{10}{\hugett{1}}}%
                \underset{1}{\overset{9}{\hugett{1}}}%
                \underset{0}{\overset{8}{\hugett{0}}}%
            }^{\texttt{first}}%
        }_{\texttt{byte 1}}%
        \hugett{,}%
        \underbrace{%
            \overbrace{%
                \underset{7}{\overset{2}{\hugett{0}}}%
                \underset{6}{\overset{1}{\hugett{0}}}%
                \underset{5}{\overset{0}{\hugett{0}}}%
            }^{\texttt{fifth}}%
            \overbrace{%
                \underset{4}{\overset{1}{\hugett{1}}}%
                \underset{3}{\overset{0}{\hugett{1}}}%
            }^{\texttt{fourth}}%
            \overbrace{%
                \underset{2}{\overset{3}{\hugett{1}}}%
                \underset{1}{\overset{2}{\hugett{0}}}%
                \underset{0}{\overset{1}{\hugett{1}}}%
            }^{\texttt{third}}%
        }_{\texttt{byte 2}}%
        \hugett{,}%
        \underbrace{%
            \overbrace{%
                \underset{7}{\overset{?}{\hugett{?}}}%
                \underset{6}{\overset{?}{\hugett{?}}}%
                \underset{5}{\overset{?}{\hugett{?}}}%
                \underset{4}{\overset{?}{\hugett{?}}}%
                \underset{3}{\overset{?}{\hugett{?}}}%
                \underset{2}{\overset{?}{\hugett{?}}}%
                \underset{1}{\overset{?}{\hugett{?}}}%
            }^{\substack{\text{Next object or} \\ \text{zero padding bits}}}
            \overbrace{%
                \underset{0}{\overset{3}{\hugett{1}}}%
            }^{\texttt{fifth}}%
        }_{\texttt{byte 3}}%
    $$

    Note that some of the complexity of the above illustration stems from the modern convention of representing
    numbers with the most significant components on the left moving to the least significant component of the
    number of the right. If you were to reverse this convention the bit sequences for each type in the composite
    would seem to be continuous as they crossed byte boundaries. Using this reversed representation, however, is
    not recommended because the convention is deeply ingrained in most readers, tools, and technologies.
\end{remark}

\subsubsection{Sealed tagged union}

Similar to variable-length arrays, a serialized representation of a sealed tagged union consists of two segments:
the implicit \emph{union tag} value followed by the selected field attribute value.

The implicit union tag is an unsigned integer value whose serialized representation
is implicitly injected in the beginning of the serialized representation of its tagged union.
The bit length of the implicit union tag is determined as follows:
$$b=\lceil{}\log_2 n\rceil{}$$
where $n$ is the number of field attributes in the union, $n \geq 2$ and $b$ is the minimum number of bits needed
to encode $n$ as an unsigned integer. An additional transformation of $b$ ensures byte alignment of this implicit
field when serialized\footnote{Future updates to the specification may allow this second step to be modified but
the default action will always be to byte-align the implicit length field.}:

$$2^{\lceil{}\log_2 (\text{max}(8, b))\rceil{}}$$

Each of the tagged union field attributes is assigned an index according to the order of their definition;
the order follows that of the DSDL statements (see section~\ref{sec:dsdl_grammar} on statement ordering).
The first defined field attribute is assigned the index 0 (zero),
the index of each following field attribute is incremented by one.

The index of the field attribute whose value is currently held by the tagged union is encoded
in the serialized representation of the implicit union tag as described in section
\ref{sec:dsdl_serialized_unsigned_integer}.
By definition, $i < n$, where $i$ is the index of the current field attribute;
therefore, bit sequences where the implicit union tag field contains values
that are greater than or equal $n$ do not belong to the set of serialized representations of the tagged union.

The serialized representation of the implicit union tag is immediately followed by
the serialized representation of the currently selected field attribute value\footnote{%
    Observe that the implicit union tag field, per its definition,
    is guaranteed to never break the alignment of the following field.
    There may be no padding between the implicit union tag field and the selected field.
}.

\begin{remark}
    Consider the following example:

    \begin{minted}{python}
        @sealed
        @union           # In this case, the implicit union tag is one byte wide
        uint16 FOO = 42  # A regular constant attribute
        uint16 a         # Field index 0
        uint8 b          # Field index 1
        uint32 BAR = 42  # Another regular constant attribute
        float64 c        # Field index 2
    \end{minted}

    In order to serialize the field \verb|b|, the implicit union tag shall be assigned the value 1.
    The following type will have an identical layout:

    \begin{minted}{python}
        @sealed
        uint8 implicit_union_tag  # Set to 1
        uint8 b                   # The actual value
    \end{minted}

    Suppose that the value of \verb|b| is 7.
    The resulting serialized representation is shown below in the base-2 system:
    $$%
    \overset{\text{byte 0}}{%
        \underbrace{\hugett{00000001}}_{\substack{\text{union} \\ \text{tag}}}%
    }%
    \hugett{,}%
    \overset{\text{byte 1}}{%
        \underbrace{\hugett{00000111}}_{\text{field }\texttt{b}}%
    }
    $$

\end{remark}

\begin{remark}
    Let the following data type be defined under the short name \verb|Empty| and version 1.0:

    \begin{minted}{python}
        # Empty. The only valid serialized representation is an empty bit sequence.
        @sealed
    \end{minted}

    Consider the following union:

    \begin{minted}{python}
        @sealed
        @union
        Empty.1.0 none
        AnyType.1.0 some
    \end{minted}

    The set of serialized representations of the union given above is equivalent to
    that of the following variable-length array:

    \begin{minted}{python}
        @sealed
        AnyType.1.0[<=1] maybe_some
    \end{minted}
\end{remark}

\subsubsection{Delimited types}\label{sec:dsdl_serialization_composite_non_sealed}

Objects of delimited (non-sealed) composite types that are nested inside other objects\footnote{%
    Of any type, not necessarily composite; e.g., arrays.
}
are serialized into opaque containers that consist of two parts:
the fixed-length \emph{delimiter header},
immediately followed by the serialized representation of the object as if it was of a sealed type.

Objects of delimited composite types that are \emph{not} nested inside other objects (i.e., top-level objects)
are serialized as if they were of a sealed type (without the delimiter header).
The delimiter header, therefore, logically belongs to the container object rather than the contained one.

\begin{remark}
    Top-level objects do not require the delimiter header because the change in their length does not necessarily
    affect the backward compatibility thanks to the implicit truncation rule
    (section \ref{sec:dsdl_serialization_implicit_truncation}) and the implicit zero extension rule
    (section \ref{sec:dsdl_serialization_implicit_zero_extension}).
\end{remark}

The delimiter header is an implicit field of type \verb|uint32| that encodes the length of the
serialized representation it precedes in bytes\footnote{%
    Remember that by virtue of the padding requirement (section \ref{sec:dsdl_composite_alignment_cumulative_bls}),
    the length of the serialized representation of a composite type is always an integer number of bytes.
}.
During deserialization, if the length of the serialized representation reported by its delimiter header
does not match the expectation of the deserializer,
the implicit truncation (section \ref{sec:dsdl_serialization_implicit_truncation})
and the implicit zero extension (section \ref{sec:dsdl_serialization_implicit_zero_extension})
rules apply.

The length encoded in a delimiter header cannot exceed the number of bytes remaining between the delimiter header
and the end of the serialized representation of the outer object.
Otherwise, the serialized representation of the outer object is invalid and is to be discarded
(section \ref{sec:dsdl_serialized_error}).

It is allowed for a sealed composite type to nest non-sealed composite types, and vice versa.
No special rules apply in such cases.

\begin{remark}
    The resulting serialized representation of a delimited composite is identical to \verb|uint8[<2**32]|
    (sans the higher alignment requirement).
    The implicit array length field is like the delimiter header,
    and the array content is the serialized representation of the composite as if it was sealed.

    The following illustrates why this is necessary for robust extensibility.
    Suppose that some composite $C$ contains two fields whose types are $A$ and $B$.
    The fields of $A$ are $a_0,\ a_1$;
    likewise, $B$ contains $b_0,\ b_1$.

    Suppose that $C^\prime$ is modified such that $A^\prime$ contains an extra field $a_2$.
    If $A$ (and $A^\prime$) were sealed, this would result in the breakage of compatibility between $C$ and $C^\prime$
    as illustrated in figure \ref{fig:dsdl_sealed_non_extensibility} because the positions of the fields of $B$
    (which is sealed) would be shifted by the size of $a_2$.

    The use of opaque containers allows the implicit truncation and the implicit zero extension rules to apply
    at any level of nesting, enabling agents expecting $C$ to truncate $a_2$ away,
    and enabling agents expecting $C^\prime$ to zero-extend $a_2$
    if it is not present, as shown in figure \ref{fig:dsdl_non_sealed_extensibility},
    where $H_A$ is the delimiter header of $A$.
    Observe that it is irrelevant whether $C$ (same as $C^\prime$) is sealed or not.

    \begin{figure}[H]
        \centering
        \begin{tabular}{r c c c c c}
            \cline{2-5}
            $C$ &
            \multicolumn{1}{|c|}{$a_0$} & \multicolumn{1}{c|}{$a_1$}
            &\multicolumn{1}{c|}{$b_0$} & \multicolumn{1}{c|}{$b_1$} &
            \\\cline{2-5}
            & $\checkmark$ & $\checkmark$ & $\times$ & $\times$ & $\times$ \\
            \cline{2-6}
            $C^\prime$ &
            \multicolumn{1}{|c|}{$a_0$} & \multicolumn{1}{c|}{$a_1$} & \multicolumn{1}{c|}{$a_2$}
            &\multicolumn{1}{c|}{$b_0$} & \multicolumn{1}{c|}{$b_1$}
            \\\cline{2-6}
        \end{tabular}
        \caption{Non-extensibility of sealed types}
        \label{fig:dsdl_sealed_non_extensibility}
    \end{figure}

    \begin{figure}[H]
        \centering
        \begin{tabular}{r c c c c c c}
            \cline{2-7}
            $C$ &
            \multicolumn{1}{|c|}{$H_A$} & \multicolumn{1}{c|}{$a_0$} & \multicolumn{1}{c|}{$a_1$}
            &\multicolumn{1}{c|}{\footnotesize{$\ldots$}}
            &\multicolumn{1}{c|}{$b_0$} & \multicolumn{1}{c|}{$b_1$}
            \\\cline{2-7}
            & $\checkmark$ & $\checkmark$ & $\checkmark$ & $\checkmark$ & $\checkmark$ & $\checkmark$ \\
            \cline{2-7}
            $C^\prime$ &
            \multicolumn{1}{|c|}{$H_A$} & \multicolumn{1}{c|}{$a_0$} & \multicolumn{1}{c|}{$a_1$} &
            \multicolumn{1}{c|}{$a_2$}
            &\multicolumn{1}{c|}{$b_0$} & \multicolumn{1}{c|}{$b_1$}
            \\\cline{2-7}
        \end{tabular}
        \caption{Extensibility of delimited types with the help of the delimiter header}
        \label{fig:dsdl_non_sealed_extensibility}
    \end{figure}

    This example also illustrates why the extent is necessary.
    Per the rules set forth in \ref{sec:dsdl_composite_extent_and_sealing},
    it is required that the extent (i.e., the buffer memory requirement) of $A$ shall be large enough to accommodate
    serialized representations of $A^\prime$, and, therefore,
    the extent of $C$ is large enough to accommodate serialized representations of $C^\prime$.
    If that were not the case, then an implementation expecting $C$ would be unable to correctly process $C^\prime$
    because the implicit truncation rule would have cut off $b_1$, which is unexpected.

    The design decision to make the delimiter header of a fixed width may not be obvious so it's worth explaining.
    There are two alternatives: making it variable-length and making the length a function of the extent
    (section \ref{sec:dsdl_composite_extent_and_sealing}).
    The first option does not align with the rest of the specification because DSDL does not make use of
    variable-length integers (unlike some other formats, like Google Protobuf, for example),
    and because a variable-length length {\footnotesize{(sic!)}} prefix would have somewhat complicated the
    bit length set computation.
    The second option would make nested hierarchies (composites that nest other composites) possibly highly fragile
    because the change of the extent of a deeply nested type may inadvertently move the delimiter header of an
    outer type into a different length category, which would be disastrous for compatibility and hard to spot.
    There is an in-depth discussion of this issue (and other related matters) on the forum.

    The fixed-length delimiter header may be considered large,
    but delimited types tend to also be complex, which makes the overhead comparatively insignificant,
    whereas sealed types that tend to be compact and overhead-sensitive do not contain the delimiter header.
\end{remark}

\begin{remark}
    In order to efficiently serialize an object of a delimited type,
    the implementation may need to perform a second pass to reach the delimiter header
    after the object is serialized, because before that, the value of the delimiter header cannot be known
    unless the object is of a fixed-size (i.e., the cardinality of the bit length set is one).

    Consider:
    \begin{minted}{python}
        uint8[<=4] x
    \end{minted}
    Let $\texttt{x} = \left[ 4, 2 \right]$,
    then the nested serialized representation would be constructed as:
    \begin{enumerate}
        \item Memorize the current memory address $M_\text{origin}$.
        \item Skip 32 bits.
        \item Encode the length: 2 elements.
        \item Encode $x_0 = 4$.
        \item Encode $x_1 = 2$.
        \item Memorize the current memory address $M_\text{current}$.
        \item Go back to $M_\text{origin}$.
        \item Encode a 32-bit wide value of $(M_\text{current} - M_\text{origin})$.
        \item Go back to $M_\text{current}$.
    \end{enumerate}

    However, if the object is known to be of a constant size, the above can be simplified,
    because there may be only one possible value of the delimiter header.
    Automatic code generation tools should take advantage of this knowledge.
\end{remark}

\section{Compatibility and versioning}\label{sec:dsdl_versioning}

\subsection{Rationale}

데이터 타입 정의는 어플리케이션의 니즈에 더 잘 맞도록 개선되면서 시간이 지날수록 진화하게 된다.
UAVCAN은 데이터 타입 디자이너가 하위 호환이나 기능 안정성을 보장하기 위해서 데이터 타입 정의를 수정하고 이용하도록 규칙들을 정의할 수 있다.

\subsection{Semantic compatibility}\label{sec:dsdl_semantic_compatibility}

데이터 타입 $A$은 데이터 타입 $B$와 \emph{semantically compatible}하다고 하는 경우는 만약에 관련 어플리케이션의 동작 속성이 항상 $A$ 를 $B$로 대체 가능한 경우이다.
의미 호환이 가능한 속성은 교환법칙이 성립된다.

\begin{remark}[breakable]
    The following two definitions are semantically compatible and can be used interchangeably:

    \begin{minted}{python}
        uint16 FLAG_A = 1
        uint16 FLAG_B = 256
        uint16 flags
        @extent 16
    \end{minted}

    \begin{minted}{python}
        uint8 FLAG_A = 1
        uint8 FLAG_B = 1
        uint8 flags_a
        uint8 flags_b
        @extent 16
    \end{minted}

    필드와 일정한 속성이 달라지므로 주의를 주는 것이 필요하다.
    제공된 정의로부터 자동생성된 소스 코드는 교체하기 위해서 어플리케이션에서 변경이 필요할 수 있다.;
    하지만 소스 코드 레벨 어플리케이션 호환성은 데이터 타입 호환성과 관련이 없다.


    다음 수퍼타입은 의미상 제거된 필드의 의미에 따라서 위에 있는 것과 의미상 호환이 될 수도 안될 수도 있다.:

    \begin{minted}{python}
        uint8 FLAG_A = 1
        uint8 flags_a
        @extent 16
    \end{minted}
\end{remark}

\begin{remark}
    Let node $A$ publish messages of the following type:

    \begin{minted}{python}
        float32 foo
        float64 bar
        @extent 128
    \end{minted}

    Let node $B$ subscribe to the same subject using the following data type definition:

    \begin{minted}{python}
        float32 foo
        float64 bar
        int16   baz  # Extra field; implicit zero extension rule applies.
        @extent 128
    \end{minted}

    Let node $C$ subscribe to the same subject using the following data type definition:

    \begin{minted}{python}
        float32 foo
        # The field 'bar' is missing; implicit truncation rule applies.
        @extent 128
    \end{minted}

    Provided that the semantics of the added and omitted fields allow it,
    the nodes will be able to interoperate successfully despite using different data type definitions.
\end{remark}

\subsection{Versioning}

\subsubsection{General assumptions}

버전닝의 개념은 컴포지트 데이터 타입에만 적용된다.
특별히 언급하지 않으면 이 섹션에 ``data type''에 대한 모든 레퍼런스는 컴포지트 데이터 타입을 의미한다.

데이터 타입은 전체 이름으로 식별되고 모든 루트 네임스페이스는 유일한 이름을 갖는다고 가정한다.
모든 데이터 타입의 1개 이상의 버전이 있다.

데이터 타입 정의는 전체 이름과 버전 넘버의 쌍으로 식별된다.
달리 말하면, 버전 넘버가 다른 데이터 타입의 다양한 정의가 있을 수 있다.

\subsubsection{Versioning principles}

모든 데이터 타입 정의는 버전 넘버의 쌍을 가진다. ---
메이저 버전 넘버와 마이너 버전 넘버은 의미있는 버전을 매기는 원칙을 따른다.

다음 정의의 목적을 위해서 데이터 타입 정의의 \emph{release}는 의도된 사용자나 일반 대중에게 데이터 타입의 정의를 공개하는 것을 의미하거나 제품 시스템에서 데이터 타입의 정의의 사용을 개시하는 목적이다.

어플리케이션 동작을 보장하고 데이터 타입 정의와 관련된 강건한 마이그레이션 경로를 보장하기 위해서 동일한 전체 이름과 동일한 메이저 버전 넘버를 공유하는 모든 데이터 타입 정의는 의미적으로 서로 호환된다.

버전닝을 매기는 원칙은 데이터 타입의 이름이 변경되는 지점의 시나리오로 확장되지 않는데,
왜냐하면 이렇게 하면 본질적으로 새로운 데이터 타입의 릴리즈로 해석되기 때문이다.
이는 모든 호환 요구사항으로부터 디자이너를 안심시킬 수 있다.
새로운 데이터 타입이 처음으로 릴리즈되는 때에,
처음 정의의 버전 넘버는 ``1.0'' (major 1, minor 0)이 할당된다.

UAVCAN을 활용하여 어플리케이션의 예측가능성과 기능정 안정성을 확보하기 위해서,
일단 데이터 타입 정의가 릴리즈시키는 것을 추천하고 DSDL 소스 텍스트, 이름, 버전 넘버, 고정 port-ID, 확장, 실링 및 기타 속성은 다음과 같은 예외를 제외하고는 수정이 되지 않는다.:
\begin{itemize}
    \item Whitespace changes of the DSDL source text are allowed,
          excepting string literals and other semantically sensitive contexts.

    \item Comment changes of the DSDL source text are allowed as long as such changes
          do not affect semantic compatibility of the definition.

    \item A deprecation marker directive (section~\ref{sec:dsdl_directives}) can be added or removed\footnote{%
              Removal is useful when a decision to deprecate a data type definition is withdrawn.
          }.
\end{itemize}
특정 버전의 데이터 타입 정의가 릴리즈 되고 난 이후에는 고정 port-id를 추가 혹은 제거는 허용되지 않는다.

따라서 큰 변화는 동일한 데이터 타입의 새로운 정의를 릴리즈하는 경우에만 적용된다.(예) 새로운 버전)
데이터 타입의 새로운 정의를 위해서 동일한 메이저 넘버를 유지하는 것이 바람직하고 가능하다면,
새로운 정의의 마이너 버전 넘버는 새로운 정의를 사용하기 전에 가장 최신의 기존 마이너 버전 넘버보다 큰 수로 정할 수 있다.
반면에 메이저 버전 넘버는 1씩 증가하고 마이너 버전은 0으로 설정된다.

메이저 버전 넘버가 0이 되면 위에 원칙에 대한 예외가 적용된다.
메이저 버전 넘버가 0인 데이터 타입 정의는 어떠한 호환성 요구사항도 적용되지 않는다.
메이저 버전 넘버가 0으로 릴리즈된 데이터 타입 정의는 호환성에 대한 고려없이 임의의 방식으로 변경될 수 있다.
하지만 불변성의 원칙을 따르기 위해서 가장 최신의 기존 정의보다 마이너 버전 넘버가 1이 증가된 차기 정의를 릴리즈하는 것을 추천한다.

어떤 데이터 타입에 대해서 버전마다 많아야 하나의 정의가 있다.
달리 말하면 정의는 데이터 타입 이름과 버전 넘버 쌍의 조합마다 하나이거나 없거나 둘 중에 하나이다.

동일한 이름 아래에서 모든 데이터 타입은 동일한 종류가 된다.
달리 말하면 데이터 타입의 첫음 릴리즈된 정의가 메시지 종류 중에 하나라면 모든 버전도 메시지 종류 중에 하나여야 한다.

동일한 이름과 메이저 버전 넘버 아래에서 모든 데이터 타입은 동일한 확장과 동일한 실링 상태를 공유해야만 한다.
따라서 다음과 같이 다음과 같이 권고한다:
\begin{itemize}
    \item Avoid marking types sealed, especially complex types,
    because it is likely to render their evolution impossible.

    \item When the first version is released, its extent should be sufficiently large
    to permit addition of new fields in the future.
    Since the value of extent does not affect the network traffic, it is safe to pick a large value
    without compromising the temporal properties of the system.
\end{itemize}

\subsubsection{Fixed port identifier assignment constraints}

The following constraints apply to fixed port-ID assignments:
\begin{align*}
    \exists P(x_{a.b})                          &\rightarrow \exists P(x_{a.c})
    &\mid&\ b < c;\ x \in (M \cup S)
    \\
    \exists P(x_{a.b})                          &\rightarrow         P(x_{a.b}) =    P(x_{a.c})
    &\mid&\ b < c;\ x \in (M \cup S)
    \\
    \exists P(x_{a.b}) \land \exists P(x_{c.d}) &\rightarrow         P(x_{a.b}) \neq P(x_{c.d})
    &\mid&\ a \neq c;\ x \in (M \cup S)
    \\
    \exists P(x_{a.b}) \land \exists P(y_{c.d}) &\rightarrow         P(x_{a.b}) \neq P(y_{c.d})
    &\mid&\ x \neq y;\ x \in T;\ y \in T;\ T = \left\{ M, S \right\}
\end{align*}
where $t_{a.b}$ denotes a data type $t$ version $a.b$ ($a$ major, $b$ minor);
$P(t)$ denotes the fixed port-ID (whose existence is optional) of data type $t$;
$M$ is the set of message types, and $S$ is the set of service types.

\subsubsection{Data type version selection}

DSDL 컴파일러는 모든 유효한 데이터 타입 버전을 따로따로 컴파일해야만 한다.
이렇게 해야 어플리케이션이 모든 메이저와 마이너 버전 조합으로부터 선택할 수 있다.

전송이 시작되면 데이터 타입의 메이저 버전은 어플리케이션의 재량으로 선택된다.
마이너 버전은 선택된 메이저 버전 중에 하나
When emitting a transfer, the major version of the data type is chosen at the discretion of the application.
The minor version should be the newest available one under the chosen major version.

When receiving a transfer, the node deduces which major version of the data type to use
from its port identifier (either fixed or non-fixed).
The minor version should be the newest available one under the deduced major version\footnote{%
    Such liberal minor version selection policy poses no compatibility risks since all definitions under the same
    major version are compatible with each other.
}.

It follows from the above two rules that when a node is responding to a service request,
the major data type version used for the response transfer shall be the same that is used for the request transfer.
The minor versions may differ, which is acceptable due to the major version compatibility requirements.

\begin{remark}[breakable]
    A simple usage example is provided in this intermission.

    Suppose a vendor named ``Sirius Cybernetics Corporation'' is contracted to design a
    cryopod management data bus for a colonial spaceship ``Golgafrincham B-Ark''.
    Having consulted with applicable specifications and standards, an engineer came up with the following
    definition of a cryopod status message type (named \verb|sirius_cyber_corp.b_ark.cryopod.Status|):

    \begin{minted}{python}
        # sirius_cyber_corp.b_ark.cryopod.Status.0.1

        float16 internal_temperature    # [kelvin]
        float16 coolant_temperature     # [kelvin]

        uint8 FLAG_COOLING_SYSTEM_A_ACTIVE = 1
        uint8 FLAG_COOLING_SYSTEM_B_ACTIVE = 2
        # Status flags in the lower bits.
        uint8 FLAG_PSU_MALFUNCTION = 32
        uint8 FLAG_OVERHEATING     = 64
        uint8 FLAG_CRYOBOX_BREACH  = 128
        # Error flags in the higher bits.
        uint8 flags  # Storage for the above defined flags (this is not the recommended practice).

        @extent 1024 * 8  # Pick a large extent to allow evolution. Does not affect network traffic.
    \end{minted}

    The definition is then deployed to the first prototype for initial laboratory testing.
    Since the definition is experimental, the major version number is set to zero in order to signify the
    tentative nature of the definition.
    Suppose that upon completion of the first trials it is identified that the units should track their
    power consumption in real time for each of the three redundant power supplies independently.

    It is easy to see that the amended definition shown below is not semantically compatible
    with the original definition; however, it shares the same major version number of zero, because the backward
    compatibility rules do not apply to zero-versioned data types to allow for low-overhead experimentation
    before the system is deployed and fielded.

    \begin{minted}{python}
        # sirius_cyber_corp.b_ark.cryopod.Status.0.2

        truncated float16 internal_temperature    # [kelvin]
        truncated float16 coolant_temperature     # [kelvin]

        saturated float32 power_consumption_0     # [watt] Power consumption by the redundant PSU 0
        saturated float32 power_consumption_1     # [watt] likewise for PSU 1
        saturated float32 power_consumption_2     # [watt] likewise for PSU 2
        # breaking compatibility with Status.0.1 is okay because the major version is 0

        uint8 FLAG_COOLING_SYSTEM_A_ACTIVE = 1
        uint8 FLAG_COOLING_SYSTEM_B_ACTIVE = 2
        # Status flags in the lower bits.
        uint8 FLAG_PSU_MALFUNCTION = 32
        uint8 FLAG_OVERHEATING     = 64
        uint8 FLAG_CRYOBOX_BREACH  = 128
        # Error flags in the higher bits.
        uint8 flags  # Storage for the above defined flags (this is not the recommended practice).

        @extent 512 * 8  # Extent can be changed freely because v0.x does not guarantee compatibility.
    \end{minted}

    The last definition is deemed sufficient and is deployed to the production system
    under the version number of 1.0: \verb|sirius_cyber_corp.b_ark.cryopod.Status.1.0|.

    Having collected empirical data from the fielded systems, the Sirius Cybernetics Corporation has
    identified a shortcoming in the v1.0 definition, which is corrected in an updated definition.
    Since the updated definition, which is shown below, is semantically compatible\footnote{%
        The topic of data serialization is explored in detail in section~\ref{sec:dsdl_data_serialization}.
    } with v1.0, the major version number is kept the same and the minor version number is incremented by one:

    \begin{minted}{python}
        # sirius_cyber_corp.b_ark.cryopod.Status.1.1

        saturated float16 internal_temperature    # [kelvin]
        saturated float16 coolant_temperature     # [kelvin]

        float32[3] power_consumption    # [watt] Power consumption by the PSU

        bool flag_cooling_system_a_active
        bool flag_cooling_system_b_active
        # Status flags (this is the recommended practice).

        void3   # Reserved for other flags

        bool flag_psu_malfunction
        bool flag_overheating
        bool flag_cryobox_breach
        # Error flags (this is the recommended practice).

        @extent 512 * 8  # Extent is to be kept unchanged now to avoid breaking compatibility.
    \end{minted}

    Since the definitions v1.0 and v1.1 are semantically compatible,
    UAVCAN nodes using either of them can successfully interoperate on the same bus.

    Suppose further that at some point a newer version of the cryopod module,
    equipped with better temperature sensors, is released.
    The definition is updated accordingly to use \verb|float32| for the temperature fields instead of \verb|float16|.
    Seeing as that change breaks the compatibility, the major version number has to be incremented by one,
    and the minor version number has to be reset back to zero:

    \begin{minted}{python}
        # sirius_cyber_corp.b_ark.cryopod.Status.2.0

        float32 internal_temperature    # [kelvin]
        float32 coolant_temperature     # [kelvin]

        float32[3] power_consumption    # [watt] Power consumption by the PSU

        bool flag_cooling_system_a_active
        bool flag_cooling_system_b_active
        void3
        bool flag_psu_malfunction
        bool flag_overheating
        bool flag_cryobox_breach

        @extent 768 * 8  # Since the major version number is different, extent can be changed.
    \end{minted}

    Imagine that later it was determined that the module should report additional status information
    relating to the coolant pump.
    Thanks to the implicit truncation (section \ref{sec:dsdl_serialization_implicit_truncation}),
    implicit zero extension (section \ref{sec:dsdl_serialization_implicit_zero_extension}),
    and the delimited serialization (section \ref{sec:dsdl_serialization_composite_non_sealed}),
    the new fields can be introduced in a semantically-compatible way without releasing
    a new major version of the data type:

    \begin{minted}{python}
        # sirius_cyber_corp.b_ark.cryopod.Status.2.1

        float32 internal_temperature    # [kelvin]
        float32 coolant_temperature     # [kelvin]

        float32[3] power_consumption    # [watt] Power consumption by the PSU

        bool flag_cooling_system_a_active
        bool flag_cooling_system_b_active
        void3
        bool flag_psu_malfunction
        bool flag_overheating
        bool flag_cryobox_breach

        float32 rotor_angular_velocity  # [radian/second] (usage of RPM would be non-compliant)
        float32 volumetric_flow_rate    # [meter^3/second]
        # Coolant pump fields (extension over v2.0; implicit truncation/extension rules apply)
        # If zero, assume that the values are unavailable.

        @extent 768 * 8
    \end{minted}

    It is also possible to add an optional field at the end wrapped into a variable-length
    array of up to one element, or a tagged union where the first field is empty
    and the second field is the wrapped value.
    In this way, the implicit truncation/extension rules would automatically make such optional field
    appear/disappear depending on whether it is supported by the receiving node.

    Nodes using v1.0, v1.1, v2.0, and v2.1 definitions can coexist on the same network,
    and they can interoperate successfully as long as they all support at least v1.x or v2.x.
    The correct version can be determined at runtime from the port identifier assignment as described in
    section~\ref{sec:basic_subjects_and_services}.

    In general, nodes that need to maximize their compatibility are likely to employ all existing major versions of
    each used data type.
    If there are more than one minor versions available, the highest minor version within the major version should
    be used in order to take advantage of the latest changes in the data type definition.
    It is also expected that in certain scenarios some nodes may resort to publishing the same message type
    using different major versions concurrently to circumvent compatibility issues
    (in the example reviewed here that would be v1.1 and v2.1).

    The examples shown above rely on the primitive scalar types for reasons of simplicity.
    Real applications should use the type-safe physical unit definitions available in the SI namespace instead.
    This is covered in section~\ref{sec:application_functions_si}.
\end{remark}

\section{Conventions and recommendations}

이번 섹션에서는 데이터 타입 설계자가 일관성 있는 스타일을 유지하는데 도움을 얻을 수 있는 추천 규칙과 일반적인 실수를 피하는 방법에 대해서 다룬다.
이 섹션에서 제공하는 모든 추천 규칙은 반드시 따라야하는 강제성은 없으면 선택할 수 있다.

\subsection{Naming recommendations}

일반 SW 개발 분야에서 널리 사용되는 DSDL 네이밍 추천은 아래와 같다.

\begin{itemize}
    \item Namespaces and field attributes should be named in the \verb|snake_case|.
    \item Constant attributes should be named in the \verb|SCREAMING_SNAKE_CASE|.
    \item Data types (excluding their namespaces) should be named in the \verb|PascalCase|.
    \item Names of message types should form a declarative phrase or a noun. For example,
          \verb|BatteryStatus| or \verb|OutgoingPacket|.
    \item Names of service types should form an imperative phrase or a verb. For example,
          \verb|GetInfo| or \verb|HandleIncomingPacket|.
    \item Short names, unnecessary abbreviations, and uncommon acronyms should be avoided.
\end{itemize}

\subsection{Comments}

모든 데이터 타입 정의 파일에서 시작은 해당 데이터 타입에 대한 상세한 설명, 목적, 사용 패턴, 관련 데어터 교환 패턴, 제약사항 등과 같이 데이터 타입을 사용하는데 유용한 정보를 제공하는 헤더 코멘트로 시작한다.

데이터 타입 정의에 대한 모든 속성과 각 필드의 특별한 속성은 상세한 설명, 목적, 사용 패턴, 관련 데어터 교환 패턴, 제약사항 등을 가지고 있다.
속성 중에 충분한 설명이 제공되고 이름이 의도를 명확하게 나타내는 경우에는 예외가 적용된다.

코멘트는 설명하려는 엔트티 뒤에 위치해야만 한다:
동일한 라인에 있던가 다음 라인에 위치한다.
이런 추천은 파일 헤더 코멘트에는 적용되지 않는다.

% Field comment placement https://forum.uavcan.org/t/dsdl-documentation-comments/407

\subsection{Optional value representation}

자료구조에는 항상 존재하지 않을 수 있는 옵션 필드 속성을 포함할 수도 있다.

옵션 필드 속성을 표현하는 추천 접근법은 하나의 엘리멘트로 가변 길이 배열을 사용하는 것이다.

다른 방법은 이런 1개짜리 엘리멘트 가변 길이 배열은 2개짜리 필드 unions로 대체가 가능하다. 여기서 첫번째 필드는 비어 있고 두번째 필드는 원하는 옵션 값을 포함한다.
원하는 레이아웃은 위에서 설명한 의미상 1개짜리 엘리멘트 배열과 호환되고 필드 속성이 스왑되지 않았다는 것을 제공한다.

Floating-point-typed 필드 속성은 IEEE 754 NaN의 값이 할당될 수 있다. 이것은 해당 값이 지정되지 않았다는 것을 나타낸다.:
하지만 이런 패턴은 해당 값이 존재하지 않더라도 버스 상에서 전송되어야 하는데 못하고 있으므로 사용하지 않는다. 그리고 이런 특별한 값은 타입 안전성을 헤친다.

\begin{remark}[breakable]
    Array-based optional field:

    \begin{minted}{python}
        MyType[<=1] optional_field
    \end{minted}

    Union-based optional field:

    \begin{minted}{python}
        @sealed                         # Sic!
        @union                          # The implicit tag is one byte long.
        uavcan.primitive.Empty none     # Represents lack of value, unpopulated field.
        MyType some                     # The field of interest; field ordering is important.
    \end{minted}

    The defined above union can be used as follows (suppose it is named \verb|MaybeMyType|):

    \begin{minted}{python}
        MaybeMyType optional_field
    \end{minted}

    The shown approaches are semantically compatible.
\end{remark}

\begin{remark}[breakable]
    The implicit truncation and the implicit zero extension rules allow one to freely add such optional fields
    at the end of a definition while retaining semantic compatibility.
    The implicit truncation rule will render them invisible to nodes that utilize older data type definitions
    which do not contain them, whereas nodes that utilize newer definitions will be able to correctly process
    objects serialized using older definitions because the implicit zero extension rule guarantees
    that the optional fields will appear unpopulated.

    For example, let the following be the old message definition:

    \begin{minted}{python}
        float64 foo
        float32 bar
    \end{minted}

    The new message definition with the new field is as follows:

    \begin{minted}{python}
        float64 foo
        float32 bar
        MyType[<=1] my_new_field
    \end{minted}

    Suppose that one node is publishing a message using the old definition,
    and another node is receiving it using the new definition.
    The implicit zero extension rule guarantees that the optional field array will
    appear empty to the receiving node because the implicit length field will be read as zero.
    Same is true if the message was nested inside another one, thanks to the delimiter header.
\end{remark}

\subsection{Bit flag representation}

비트 플래그들의 집합을 정의하는 추천 방법은 각각에 대해서 \verb|bool|-타입 필드 속성을 명시하는 것이다.
2의 자승의 정수 합을\footnote{Which are popular in programming.} 기반으로 하는 표현은 의도를 명확히 드러내는데 모호하므로 사용하지 않는다.

\begin{remark}
    Recommended approach:

    \begin{minted}{python}
        void5
        bool flag_foo
        bool flag_bar
        bool flag_baz
    \end{minted}

    Not recommended:

    \begin{minted}{python}
        uint8 flags             # Not recommended
        uint8 FLAG_BAZ = 1
        uint8 FLAG_BAR = 2
        uint8 FLAG_FOO = 4
    \end{minted}
\end{remark}

%
    % Clean up afterwards to prevent accidental reuse if the command fails the next time we invoke it.
    \immediate\write18{rm -f ../*.tmp}%
}

\newcommand{\DSDLReference}[1]{% We use detokenize to permit underscores
    \mbox{\texttt{\detokenize{#1}}} (section~\ref{sec:dsdl:#1} on page~\pageref{sec:dsdl:#1})%
}

\title{Specification v1.0-beta}

\hbadness=10000

\begin{document}
\frontmatter

\begin{titlepage}

\section*{Overview}

UAVCAN is an open lightweight protocol designed for reliable intravehicular communication
in aerospace and robotic applications over robust transports.
It is created to address the challenge of deterministic on-board data exchange
between systems and components of advanced intelligent vehicles.

UAVCAN은 \emph{Uncomplicated Application-level Vehicular Communication And Networking}를 의미한다.

특징:

\begin{itemize}
    \item 평등한 네트워크 -- 버스 마스터가 없고 단일 장애로 전체 시스템이 불능이 발생하지 않음
    \item Publish/subscribe과 request/response (RPC\footnote{Remote procedure call.}) 통신 시맨틱스
    \item 큰 자료 구조를 효과적으로 교환하기 위한 자동 분해/결합
    \item 경량, deterministic, 구현 및 검증 용이성
    \item 임베디드, 자원 제약, 실시간 시스템에 적합
    \item 2 ~ 3중화 트랜스포트 지원
    \item 높은 정확도의 네트워크내 시간 동기화 지원
    \item 풍부한 데이터 타입 및 인터페이스 추상화 제공 - 인테페이스 서술 언어 Provides rich data type and interface abstractions -- an interface description language is a core part of
    the technology which allows deeply embedded sub-systems to interface with higher-level systems directly and
    in a maintainable manner while enabling simulation and functional testing.
    \item 스펙 및 높은 품질의 구현이 가능하며 대중 프로그래밍 언어에서 무료로 오픈소스로 가능하며 MIT 라이센스로 상업적으로 사용이 가능하다.
\end{itemize}

\BeginRightColumn

\section*{License}

UAVCAN은 모두에게 오픈된 표준이며 항상 이 상태를 유지할 것이다. 
UAVCAN is a standard open to everyone, and it will always remain this way.
No authorization or approval of any kind is necessary for its implementation, distribution, or use.

% The following statement looks a bit archaic, but it is the recommended form according to
% https://creativecommons.org/choose/results-one?license_code=by&amp;jurisdiction=&amp;version=4.0&amp;lang=en
This work is licensed under the Creative Commons Attribution 4.0 International License.
To view a copy of this license, visit
\href{http://creativecommons.org/licenses/by/4.0/}{creativecommons.org/licenses/by/4.0}
or send a letter to Creative Commons, PO Box 1866, Mountain View, CA 94042, USA.

\hspace*{\fill}\includegraphics[height=1.75\baselineskip]{cc-by}\hspace*{\fill}

\section*{Disclaimer of warranty}

Note well: this Specification is provided on an ``as is'' basis, without warranties or conditions of any kind,
express or implied, including, without limitation, any warranties or conditions of
title, non-infringement, merchantability, or fitness for a particular purpose.

\section*{Limitation of liability}

In no event and under no legal theory, whether in tort (including negligence), contract, or otherwise,
unless required by applicable law (such as deliberate and grossly negligent acts) or agreed to in writing,
shall any author of this Specification be liable for damages,
including any direct, indirect, special, incidental, or consequential damages of any character arising
from, out of, or in connection with the Specification or the implementation, deployment,
or other use of the Specification (including but not limited to damages for loss of goodwill,
work stoppage, equipment failure or malfunction, injuries to persons, death,
or any and all other commercial damages or losses),
even if such author has been made aware of the possibility of such damages.

\end{titlepage}

\tableofcontents
\clearpage\onecolumn\listoftables
\clearpage\onecolumn\listoffigures

\mainmatter

\chapter{Introduction}\label{sec:introduction}

This is a non-normative chapter covering the basic concepts that govern development and maintenance of
the specification.

\section{Overview}

UAVCAN은 경량 프로토콜로서 publish-subscribe과 원격 프로시저 호출 방식의 높은 신뢰성을 가지는 통신을 제공하도록 설계되었다. 장치내에 있는 강건한 버스 네트워크를 통해서 항공기와 로보틱스 응용 장치에 적합하다.
차세대 지능 운송 장치의 온보드 상에 시스템과 컴포넌트 사이의 데이터 교환의 문제점을 해결하기 위해 고안되었다.: 유인/무인 비행장치, 우주선, 로봇, 차량


UAVCAN can be approximated as a highly deterministic decentralized object request broker
with a specialized interface description language and a highly efficient data serialization format
suitable for use in real-time safety-critical systems with optional modular redundancy.

UAVCAN의 \emph{Uncomplicated Application-level Vehicular Computing And Networking}를 의미한다.

UAVCAN은 모두에게 오픈된 표준이며 향후에도 계속 오픈된 상태를 유지할 것이다.
구현, 배포, 사용에 있어서 허가나 승인이 필요하지 않다.

UAVCAN 스펙의 개발 및 유지보수는 공개된 토론 포럼, 소프트웨어 저장소, 공식 사이트 \href{http://uavcan.org}{uavcan.org}를 통해서 이뤄진다.

UAVCAN을 활용을 모색하는 엔지니어는 공식 웹사이트에서 별도로 제공하는 \emph{The UAVCAN Guide} 를 참고하도록 한다.

\section{Document conventions}

프로토콜의 정의에 직접 들어가지 않는 비공식 내용, 예제, 추천, 퇴고는 footnotes\footnote{This is a footnote.}에 포함되거나 아래와 같이 강조 섹션에 포함된다.

\begin{remark}
    예제와 같이 비공식 내용은 여기와 같이 어두운 박스에 표현된다.
\end{remark}

코드는 아래와 같은 형태로 표현된다.
이런 형태의 모든 코드는 특별한 언급이 없다면 이 스펙과 동일한 라이센스로 배포된다.

\begin{minted}{rust}
    // This is a source code listing.
    fn main() {
        println!("Hello World!");
    }
\end{minted}

하나의 바이트(byte)는 8개 비트(bit)의 그룹이다.
A byte is a group of eight (8) bits.

Textual patterns are specified using the standard
POSIX Extended Regular Expression (ERE) syntax;
the character set is ASCII and patterns are case sensitive, unless explicitly specified otherwise.

Type parameterization expressions use subscript notation,
where the parameter is specified in the subscript enclosed in angle brackets:
$\texttt{type}_\texttt{<parameter>}$.

Numbers are represented in base-10 by default.
If a different base is used, it is specified after the number in the subscript\footnote{%
    E.g., $\text{BADC0FFEE}_{16} = 50159747054$, $10101_2 = 21$.
}.

DSDL definition examples provided in the document are illustrative and may be incomplete or invalid.
This is to ensure that the examples are not cluttered by irrelevant details.
For example, \verb|@extent| or \verb|@sealed| directives may be omitted if not relevant.

\section{Design principles}

\begin{description}
    \item[평등한 네트워크] --- 마스터 노드가 존재하지 않는다.
    네트워크 내에 있는 모든 노드들은 동일한 통신 권한을 가진다. 단일 장애로 네트워크 전체가 불능상태로 빠지지 않는다.

    \item[기능 안정성 촉진] --- A system designer relying on UAVCAN will have the necessary
    guarantees and tools at their disposal to analyze the system and ensure its correct behavior.

    \item[High-level 통신 추상화] --- The protocol will support publish/subscribe and remote procedure
    call communication semantics with statically defined and statically verified data types (schema).
    The data types used for communication will be defined in a clear, platform-agnostic way
    that can be easily understood by machines, including humans.  % I hope you are ok with this, my dear fellow robots.

    \item[벤더간 상호운용성 촉진] --- UAVCAN will be a common foundation that
    different vendors can build upon to maximize interoperability of their equipment.
    UAVCAN will provide a generic set of standard application-agnostic communication data types.

    \item[잘 정의된 일반 high-level 기능] --- UAVCAN will define standard services
    and messages for common high-level functions, such as network discovery, node configuration,
    node software update, node status monitoring, network-wide time synchronization, plug-and-play node support, etc.

    \item[Atomic data abstractions] --- Nodes shall be provided with a simple way of exchanging large
    data structures that exceed the capacity of a single transport frame\footnote{%
        A \emph{transport frame} is an atomic transmission unit defined by the underlying transport protocol.
        For example, a CAN frame.
    }.
    UAVCAN should perform automatic data decomposition and reassembly at the protocol level,
    hiding the related complexity from the application.

    \item[High throughput, low latency, determinism] --- UAVCAN will add a very low overhead to the underlying
    transport protocol, which will ensure high throughput and low latency, rendering the protocol well-suited
    for hard real-time applications.

    \item[Support for redundant interfaces and redundant nodes] --- UAVCAN shall be suitable for use in
    applications that require modular redundancy.

    \item[Simple logic, low computational requirements] --- UAVCAN targets a wide variety of embedded systems,
    from high-performance on-board computers to extremely resource-constrained microcontrollers.
    It will be inexpensive to support in terms of computing power and engineering hours,
    and advanced features can be implemented incrementally as needed.

    \item[Rich data type and interface abstractions] --- An interface description language will be a core part of
    the technology which will allow deeply embedded sub-systems to interface with higher-level systems directly and
    in a maintainable manner while enabling simulation and functional testing.

    \item[Support for various transport protocols] --- UAVCAN will be usable with different transports.
    The standard shall be capable of accommodating other transport protocols in the future.

    \item[API-agnostic standard] --- Unlike some other networking standards, UAVCAN will not attempt to describe
    the application program interface (API). Any details that do not affect the behavior of an implementation
    observable by other participants of the network will be outside of the scope of this specification.

    \item[Open specification and reference implementations] --- The UAVCAN specification will always be open and
    free to use for everyone; the reference implementations will be distributed under the terms of
    the permissive MIT License or released into the public domain.
\end{description}

\section{Capabilities}

The maximum number of nodes per logical network is dependent on the transport protocol in use,
but it is guaranteed to be not less than 128.

UAVCAN supports an unlimited number of composite data types,
which can be defined by the specification (such definitions are called \emph{standard data types})
or by others for private use or for public release
(in which case they are said to be \emph{application-specific} or \emph{vendor-specific}; these terms are equivalent).
There can be up to 256 major versions of a data type, and up to 256 minor versions per major version.

UAVCAN supports 8192 message subject identifiers for publish/subscribe exchanges and
512 service identifiers for remote procedure call exchanges.
A small subset of these identifiers is reserved for the core standard and for publicly released vendor-specific types
(chapter~\ref{sec:application}).

Depending on the transport protocol, UAVCAN supports at least eight distinct communication priority levels
(section~\ref{sec:transport_transfer_priority}).

The list of transport protocols supported by UAVCAN is provided in chapter~\ref{sec:transport}.
Non-redundant, doubly-redundant and triply-redundant transports are supported.
Additional transport layers may be added in future revisions of the protocol.

Application-level capabilities of the protocol (such as time synchronization, file transfer,
node software update, diagnostics, schemaless named registers, diagnostics, plug-and-play node insertion, etc.)
are listed in section~\ref{sec:application_functions}.

The core specification does not define nor explicitly limit any physical layers for a given transport, however;
properties required by UAVCAN may imply or impose constraints and/or minimum performance requirements on physical
networks. Because of this, the core standard does not control compatibility below a supported transport layer between
compliant nodes on a physical network (i.e. there are no, anticipated, compatibility concerns between compliant nodes
connected to a virtual network where hardware constraints are not enforced nor emulated).
Additional standards specifying physical-layer requirements, including connectors,
may be required to utilize this standard in a vehicle system.

The capabilities of the protocol will never be reduced within a major version of the specification but may be expanded.

\section{Management policy}

The UAVCAN maintainers are tasked with maintaining and advancing this specification and
the set of public regulated data types\footnote{%
    The related technical aspects are covered in chapters~\ref{sec:basic} and~\ref{sec:dsdl}.
} based on their research and the input from adopters.
The maintainers will be committed to ensuring long-term stability and backward compatibility of
existing and new deployments.
The maintainers will publish relevant announcements and solicit inputs from adopters
via the discussion forum whenever a decision that may potentially affect existing deployments is being made.

The set of standard data types is a subset of public regulated data types and is an integral part of the specification;
however, there is only a very small subset of required standard data types needed to implement the protocol.
A larger set of optional data types are defined to create a standardized data exchange environment
supporting the interoperability of COTS\footnote{Commercial off-the-shelf equipment.}
equipment manufactured by different vendors.
Adopters are invited to take part in the advancement and maintenance of the public regulated data types
under the management and coordination of the UAVCAN maintainers.

\section{Referenced sources}

The UAVCAN specification contains references to the following sources:

% Please keep the list sorted alphabetically.
\begin{itemize}
    \item CiA 103 --- Intrinsically safe capable physical layer.
    \item CiA 801 --- Application note --- Automatic bit rate detection.

    \item IEEE 754 --- Standard for binary floating-point arithmetic.
    \item IEEE Std 1003.1 --- IEEE Standard for Information Technology --
          Portable Operating System Interface (POSIX) Base Specifications.

    \item IETF RFC2119 --- Key words for use in RFCs to Indicate Requirement Levels.

    \item ISO 11898-1 --- Controller area network (CAN) --- Part 1: Data link layer and physical signaling.
    \item ISO 11898-2 --- Controller area network (CAN) --- Part 2: High-speed medium access unit.
    \item ISO/IEC 10646 --- Universal Coded Character Set (UCS).
    \item ISO/IEC 14882 --- Programming Language C++.

    \item \href{http://semver.org}{semver.org} --- Semantic versioning specification.

    \item ``A Passive Solution to the Sensor Synchronization Problem'', Edwin Olson.
    \item ``Implementing a Distributed High-Resolution Real-Time Clock using the CAN-Bus'', M. Gergeleit and H. Streich.
    \item ``In Search of an Understandable Consensus Algorithm (Extended Version)'', Diego Ongaro and John Ousterhout.
\end{itemize}

\section{Revision history}

\subsection{v1.0 -- work in progress}

\begin{itemize}
    \item The maximum data type name length has been increased from 50 to 255 characters.

    \item The default extent function has been removed (section \ref{sec:dsdl_composite_extent_and_sealing}).
    The extent now has to be specified explicitly always unless the data type is sealed.
\end{itemize}

\subsection{v1.0-beta -- Sep 2020}

Compared to v1.0-alpha, the differences are as follows (the motivation is provided on the forum):

\begin{itemize}
    \item The physical layer specification has been removed.
    It is now up to the domain-specific UAVCAN-based standards to define the physical layer.

    \item The subject-ID range reduced from $[0, 32767]$ down to $[0, 8191]$.
    This change may be reverted in a future edition of the standard, if found practical.

    \item Added support for delimited serialization; introduced related concepts of \emph{extent} and \emph{sealing}
    (section \ref{sec:dsdl_composite_extent_and_sealing}).
    This change enables one to easily evolve networked services in a backward-compatible way.

    \item Enabled the automatic runtime adjustment of the transfer-ID timeout on a per-subject basis
    as a function of the transfer reception rate (section \ref{sec:transport_transfer_reception}).
\end{itemize}

\subsection{v1.0-alpha -- Jan 2020}

This is the initial version of the document.
The discussions that shaped the initial version are available on the public UAVCAN discussion forum.

\input{basic/basic.tex}
\chapter{Data structure description language}\label{sec:dsdl}

데이터 구조 서술 언어 \emph{DSDL}은 컴포지트 데이터 타입을 정의하기 위해 설계된 간단한 도메인 특화 언어이다.
정의한 데이터 타입은 표준 UAVCAN 전송 프로토콜 중에 하나를 통해서 UAVCAN 노드들 사이에 데이터 교환에 사용된다.\footnote{The standard transport protocols are documented in chapter~\ref{sec:transport}.
UAVCAN은 사용자가 자신의 어플리케이션에 특화된 트랜스포트를 정의할 수 있지만 이 경우 호환성 이슈나 해당 프로토콜에서 성능 이슈가 발생할 수 있다.}.

\section{Architecture}

\subsection{General principles}

UAVCAN 아키텍처에 따라 DSDL에서 사용자는 2가지 종류의 데이터 타입을 정의할 수 있다:
메시지 타입과 서비스 타입
메시지 타입은 publish-subscribe 상에서 데이터 교환에 사용되며 1:다 메시지 링크는 서브젝트-ID로 식별한다. 서비스 타입은 request-response를 수행하는데 사용하며 RPC와 같이 1:1 교환에서 서비스-ID로 식별한다.
서비스 타입은 정확히 2개 내부 데이터 타입으로 구성된다:
그 중에 하나는 request 타입(클라이언트에서 서버로 전송)이고,
나머지 하나는 response 타입(서버에서 클라이언트로 전송)이다.

UAVCAN의 deterministic 속성을 따라서 메시지나 서비스 객체의 크기는 정적으로 이미 알고 있는 크기 이내에 한정된다.
가변길이 엔트티는 항상 데이터 타입 디자이너가 정의한 고정 길이 제한을 가진다.

DSDL정의는 정적 타입이다.

DSDL은 데이터 타입 버전관리를 위해 잘 정의되어 있다. 이를 통해 데이터 타입 유지보수자가 배포한 데이터 타입의 하위 호환을 되도록 변경할 수 있다.

DSDL은 확장 가능한 정적 분석을 지원하도록 설계되었다. 하위 바이너리 호환성과 데이터 필드 레이아웃과 같이 데이터 타입 정의의 중요한 속성은 이를 사용하는 시스템이 운영되기 전에 자동 소프트웨어 도구로 검증 및 검사가 가능하다.

DSDL정의는 타겟 프로그래밍 언어에서 직렬화 소스코드 데이터 타입으로 자동 생성될 수 있다.
DSDL 정의를 기반으로 직렬화 코드를 생성하는 기능을 가진 도구를 \emph{DSDL compiler}라고 부른다.
좀더 일반적으로는 말하자면 DSDL 정의로 동작하도록 설계된 소프트웨어 도구를 \emph{DSDL processing tool}라고 부른다.

\subsection{Data types and namespaces}

모든 데이터 타입은 \emph{namespace} 내부에 위치한다.
네임스페이스는 더 상위레벨 네임스페이스에 포함될 수 있으며 트리구조를 가진다.

트리 구조의 최상단에 있는 네임스페이스를 \emph{root namespace}라고 부른다.
다른 네임스페이스 내부에 위치하고 있는 네임스페이스를 \emph{nested namespace}라고 부른다.

데이터 타입은 네임스페이스와 이름\emph{short name}으로 식별된다.
데이터 타입의 이름은 네임스페이트를 제외한 타입의 이름이다.

데이터 타입의 \emph{full name}은 이름과 네임스페이스 이름들로 구성된다.전체 이름에 포함된 이름과 네임스페이스는를 \emph{name components}라고 부른다.
이름 컴포넌트들은 순서를 가진다: 루트 네임스페이스는 항상 맨 앞에 오고 다음으로 nested 네임스페이스가 온다.
전체 이름에서 이름(short name)은 항상 마지막에 위치한다.
전체 이름은 ASCII dot 문자 ``\verb|.|'' (ASCII code 46)를 통해서 이름 컴포넌트들을 연결시킨다.

\emph{full namespace} 이름은 이름(short name)과 component 구분자를 제외한 전체 이름이다.

\emph{sub-root namespace}은 netsted 네임스페이스로 루트 네임스페이스 바로 밑에 위치한다.
루트 네임스페이스 아래에 위치하고 있는 데이터 타입은 하위루트 네임스페이스를 가지지 않는다.

이름 구조는 figure~\ref{fig:dsdl_data_type_name_structure} 그림으로 표현하였다.

\begin{figure}[H]
    $$
    \overbrace{
        \underbrace{
            \underbrace{\texttt{\huge{uavcan}}}_{\substack{\text{root} \\ \text{namespace}}}%
            \texttt{\huge{.}}%
            \underbrace{\texttt{\huge{node}}}_{\substack{\text{nested, also} \\ \text{sub-root} \\ \text{namespace}}}%
            \texttt{\huge{.}}%
            \underbrace{\texttt{\huge{port}}}_{\substack{\text{nested} \\ \text{namespace}}}%
        }_{\text{full namespace}}%
        \texttt{\huge{.}}%
        \underbrace{\texttt{\huge{GetInfo}}}_{\text{short name}}
    }^{\text{full name}}
    $$
    \caption{Data type name structure\label{fig:dsdl_data_type_name_structure}}
\end{figure}

전체 네임스페이스 이름의 집합과 전체 데이터 타입 이름의 집합은 서로 교차되지 않는다.\footnote{%
    예제로 네임스페이스 ``\texttt{vendor.example}''와 데이터 타입 ``\texttt{vendor.example.1.0}''는 서로 상호배제되는 형태다.
    이 예제에서 보여준 데이터 타입 이름은 이름 규칙에 위배된다. 이는 별도 섹션에서 다룰 예정이다.
}.

데이터 타입 이름과 네임스페이스 이름은 대소문자를 구분한다.
하지만 대소문자가 다른 이름들은 허용하지 않는다.\footnote{%
    대소문자를 구분하지 않는 파일시스템에서는 문제를 야기할 수 있기 때문이다.
}.
다시 말하면 대소문자만 다른 한쌍의 이름은 이름 충돌이 발생할 수 있다.

이름 컴포넌트는 알파벳 ASCII 문자와(\verb|A-Z|, \verb|a-z|, \verb|0-9|) 언더스코어 (``\verb|_|'', ASCII code 95)로 구성되어 있다.
빈 문자열은 이름 컴포넌트로 유효하지 않다.
이름 컴포넌트의 첫번째 문자는 숫자가 올 수 없다.
이름 컴포넌트에는 어떠한 예약 워드 패턴도 패칭되지 않으며 목록은 table~\ref{table:dsdl_reserved_word_patterns}을 참조하자.

전체 데이터 타입 이름의 길이는 255문자를 초과하지 않는다.\footnote{This includes the name component separators, but not the version.}

모든 데이터 타입 정의는 메이저와 마이너 버전 넘버 쌍을 할당한다.
데이터 타입 정의를 식별하기 위해서 버전 넘버가 지정된다.
다음 텍스트에서 majority qualifier가 없는 \emph{version}는 용어는 메이저와 마이너 버전 넘버 쌍을 의미한다.

유효 데이터 타입 버전 넘버는 0에서 255의 범위를 가진다.
메이저와 마이너 컴포넌트 모두 0인 데이터 타입 버전은 허용하지 않는다.

\subsection{File hierarchy}

DSDL 데이터 타입 정의는 파일 이름 확장자가 \verb|.uavcan|이고 UTF-8로 인코딩된 텍스트 파일이다.

하나의 파일은 정확히 하나의 버전의 데이터 타입을 정의하며 메이저와 마이너 버전의 각 조합은 데이터 타입 이름에 대해서 유일하게 된다.
서동일한 데이터 타입에 대해서 서로 동시에 임의의 버전 넘버가 가능하며 각 버전은 기껏해야 한 번 정의될 수 있다.
버전 넘버 순서는 연속되는 수일 필요는 없으며 버전 넘버를 건너 뛰거나 가장 오래되거나 가장 최신인 기존 정의만 아니라면 제거도 가능하다.

데이터 타입 정의는 옵션으로 고정된 port-ID\footnote{Chapter~\ref{sec:basic}.} 지정 값이다.

데이터 타입 정의 파일의 이름은 엔트티와 ASCII dot 문자 ``\verb|.|'' (ASCII code 46)로 지정한 순서로 구성된다. :
\begin{itemize}
    \item 10진수의 고정 port-ID, 고정 port-ID가 이 정의에서 제공되는 경우
    \item 데이터 타입의 short name (필수, 공백 허용 하지 않음).
    \item 10진수의 메이저 버전 넘버 (필수)
    \item 10진수의 마이너 버전 넘버 (필수)
    \item 파일 이름 확장자 ``\verb|uavcan|'' (필수).
\end{itemize}

\begin{figure}[H]
    $$
    \overbrace{%
        \underbrace{\texttt{\huge{432}}}_{\substack{\text{fixed} \\ \text{port-ID}}}%
        \texttt{\huge{.}}%
    }^{\text{optional}}%
    \overbrace{%
        \underbrace{\texttt{\huge{GetInfo}}}_{\substack{\text{short name}}}%
        \texttt{\huge{.}}%
        \underbrace{\texttt{\huge{1.0}}}_{\substack{\text{version} \\ \text{numbers}}}%
        \texttt{\huge{.}}%
        \underbrace{\texttt{\huge{uavcan}}}_{\text{file extension}}%
    }^{\text{mandatory}}
    $$
    \caption{Data type definition file name structure\label{fig:dsdl_definition_file_name_structure}}
\end{figure}

DSDL 네임스페이스는 디렉토리를 나타낸다. 하나의 디렉토리는 정확히 하나의 네임스페이스를 정의하며 내부에 다른 디렉토리를 가질 수 있다.
디렉토리의 이름은 데이터 타입 이름 컴포넌트의 이름을 정의한다.
하나의 네임스페이스를 정의하고 있는 하나의 디렉토리는 전체에서의 사용되는 하나의 네임스페이스를 정의한다. 즉 하나의 네임스페이스의 내용은 동일한 이름을 공유하는 다른 디렉토리와 교차로 사용할 수 없다.
하나의 디렉토리는 
nesting\footnote{%
    For example, ``\texttt{foo.bar}'' is not a valid directory name.
    The valid representation would be ``\texttt{bar}'' nested in ``\texttt{foo}''.
}의 한단계 레벨 이상을 정의할 수 없다.

\begin{remark}
    \begin{figure}[H]
        \begin{tabu}{|l|X|} \hline
            \rowfont{\bfseries}
            Directory tree & Entry description \\\hline

            \texttt{vendor\_x/} &
            Root namespace \texttt{vendor\_x}. \\\cline{2-2}

            \texttt{\qquad{}foo/} &
            Nested namespace (also sub-root) \texttt{vendor\_x.foo}. \\\cline{2-2}

            \texttt{\qquad{}\qquad{}100.Run.1.0.uavcan} &
            Data type definition v1.0 with fixed service-ID 100. \\\cline{2-2}

            \texttt{\qquad{}\qquad{}100.Status.1.0.uavcan} &
            Data type definition v1.0 with fixed subject-ID 100. \\\cline{2-2}

            \texttt{\qquad{}\qquad{}ID.1.0.uavcan} &
            Data type definition v1.0 without fixed port-ID. \\\cline{2-2}

            \texttt{\qquad{}\qquad{}ID.1.1.uavcan} &
            Data type definition v1.1 without fixed port-ID. \\\cline{2-2}

            \texttt{\qquad{}\qquad{}bar\_42/} &
            Nested namespace \texttt{vendor\_x.foo.bar\_42}. \\\cline{2-2}

            \texttt{\qquad{}\qquad{}\qquad{}101.List.1.0.uavcan} &
            Data type definition v1.0 with fixed service-ID 101. \\\cline{2-2}

            \texttt{\qquad{}\qquad{}\qquad{}102.List.2.0.uavcan} &
            Data type definition v2.0 with fixed service-ID 102. \\\cline{2-2}

            \texttt{\qquad{}\qquad{}\qquad{}ID.1.0.uavcan} &
            Data type definition v1.0 without fixed port-ID. \\\hline
        \end{tabu}
        \caption{DSDL directory structure example}\label{fig:dsdl_directory_structure_example}
    \end{figure}
\end{remark}

\subsection{Elements of data type definition}\label{sec:dsdl_elements_of_data_type_definition}

A data type definition file contains an exhaustive description of a particular version of the said data type in the
\emph{data structure description language} (DSDL).

A data type definition contains an ordered, possibly empty collection of \emph{field attributes} and/or
unordered, possibly empty collection of \emph{constant attributes}.

A data type may describe either a \emph{structure object} or a \emph{tagged union object}.
The value of a structure object is a function of the values of all of its field attributes.
A tagged union object is formed from at least two field attributes,
but it is capable of holding exactly one field attribute value at any given time.
The value of a tagged union object is a function of which field attribute value
it is holding at the moment and the value of said field attribute.

A field attribute represents a named dynamically assigned value of a statically defined type
that can be exchanged over the network as a member of its containing object.
A padding field attribute is a special kind of field attribute which is used for data alignment purposes;
such field attributes are not named.

A constant attribute represents a named statically defined value of a statically defined type.
Constants are never exchanged over the network, since they are assumed to be known to all involved nodes
by virtue of them sharing compatible definitions of the data type.

Constant values are defined via \emph{DSDL expressions},
which are evaluated at the time of DSDL definition processing.
There is a special category of types called \emph{expression types},
instances of which are used only during expression evaluation
and cannot be exchanged over the network.

Data type definitions can also contain various auxiliary elements reviewed later,
such as deprecation markers (notifying its users that the data type is no longer recommended for new designs)
or assertions (special statements introduced by data type designers
which are statically validated by DSDL processing tools).

Service type definitions are a special case:
they cannot be instantiated or serialized, they do not contain attributes,
and they are composed of exactly two inner data type definitions\footnote{
    A service type can be thought of as a specialized namespace that contains two types and
    has some of the properties of a type, such as name and version.
}.
These inner types are the service request type and the service response type,
separated by the \emph{service response marker}.
They are otherwise ordinary data types except that they are unutterable\footnote{%
    Cannot be referred to. Another commonly used term is ``Voldemort type''.
}
and they derive some of their properties\footnote{Like version numbers or deprecation status.}
from their \emph{parent service type}.

\subsection{Serialization}

Every object that can be exchanged between UAVCAN nodes has a well-defined \emph{serialized representation}.
The value and meaning of an object can be unambiguously recovered from its serialized representation,
provided that the type of the object is known.
Such recovery process is called \emph{deserialization}.

\label{sec:dsdl_bit_length_set}
A serialized representation is a sequence of binary digits (bits);
the number of bits in a serialized representation is called its \emph{bit length}.
A \emph{bit length set} of a data type refers to the set of bit length values of all possible
serialized representations of objects that are instances of the data type.

A data type whose bit length set contains more than one element is said to be \emph{variable length}.
The opposite case is referred to as \emph{fixed length}.

The data type of a serialized message or service object exchanged over the network
is recovered from its subject-ID or service-ID, respectively,
which is attached to the serialized object, along with other metadata, in a manner dictated by the applicable
transport layer specification (chapter~\ref{sec:transport}).
For more information on port identifiers and data type mapping refer to section~\ref{sec:basic_subjects_and_services}.

The bit length set is not defined on service types (only on their request and response types)
because they cannot be instantiated.

\section{Grammar}\label{sec:dsdl_grammar}

이 섹션은 DSDL 문법의 공식 정의를 다룬다.
표기법에 전에 소개하였다.
문법의 각 엘리멘트와 의미는 다음 섹션에서 설명한다.

\subsection{Notation}

다음 정의는 PEG\footnote{Parsing expression grammar.}
table~\ref{table:dsdl_grammar_definition_notation}%
\footnote{%
    Inspired by Parsimonious -- an MIT-licensed software product authored by Erik Rose;
    its sources are available at \url{https://github.com/erikrose/parsimonious}.
}의 표기법을 따른다.
공식 정의에 대한 내용은 넘버 기호(#)으로 시작하는 코멘트를 포함하며 해당 라인의 끝까지 계속된다.

\begin{UAVCANSimpleTable}{Notation used in the formal grammar definition}{|l X|}
    \label{table:dsdl_grammar_definition_notation}
    Pattern & Description \\

    \texttt{"text"} &
    Denotes a terminal string of ASCII characters.
    The string is case-sensitive. \\

    \emph{(space)} &
    Concatenation.
    E.g., \texttt{korovan paukan excavator} matches a sequence where the specified tokens
    appear in the defined order. \\

    \texttt{abc / ijk / xyz} &
    Alternatives.
    The leftmost matching alternative is accepted. \\

    \texttt{abc?} &
    Optional greedy match. \\

    \texttt{abc*} &
    Zero or more expressions, greedy match. \\

    \texttt{abc+} &
    One or more expressions, greedy match. \\

    \texttt{\textasciitilde{}r"regex"} &
    An IEEE POSIX Extended Regular Expression pattern defined between the double quotes.
    The expression operates on the ASCII character set and is always case-sensitive.
    ASCII escape sequences ``\texttt{\textbackslash{}r}'', ``\texttt{\textbackslash{}n}'', and
    ``\texttt{\textbackslash{}t}'' are used to denote ASCII carriage return (code 13),
    line feed (code 10), and tabulation (code 9) characters, respectively. \\

    \texttt{\textasciitilde{}r'regex'} &
    As above, with single quotes instead of double quotes. \\

    \texttt{(abc xyz)} &
    Parentheses are used for grouping. \\
\end{UAVCANSimpleTable}

\subsection{Definition}

At the top level, a DSDL definition file is an ordered collection of statements;
the order is determined by the relative placement of statements inside the DSDL source file:
statements located closer the beginning of the file precede those that are located closer to the end of the file.

From the top level down to the expression rule, the grammar is a valid regular grammar,
meaning that it can be parsed using standard regular expressions.

The grammar definition provided here assumes lexerless parsing;
that is, it applies directly to the unprocessed source text of the definition.

All characters used in the definition belong to the ASCII character set.

\clearpage\inputminted[fontsize=\scriptsize]{python}{dsdl/grammar.parsimonious}

\subsection{Expressions}

Symbols representing operators belong to the ASCII (basic Latin) character set.

Operators of the same precedence level are evaluated from left to right.

The attribute reference operator is a special case: it is defined for an instance of any type
on its left side and an attribute identifier on its right side.
The concept of ``attribute identifier'' is not otherwise manifested in the type system.
The attribute reference operator is not explicitly documented for any data type;
instead, the documentation specifies the set of available attributes for instances of said type,
if there are any.

\begin{UAVCANSimpleTable}{Unary operators}{|l l X|}
    Symbol                             & Precedence & Description \\
    \texttt{\textbf{+}}                         & 3 & Unary plus \\
    \texttt{\textbf{-}} (hyphen-minus)          & 3 & Unary minus \\
    \texttt{\textbf{!}}                         & 8 & Logical not \\
\end{UAVCANSimpleTable}

\begin{UAVCANSimpleTable}{Binary operators}{|l l X|}
    Symbol                                          & Precedence & Description \\
    \texttt{\textbf{.}} (full stop)                          & 1 & Attribute reference
                                                                   (parent object on the left side,
                                                                   attribute identifier on the right side) \\

    \texttt{\textbf{**}}                                     & 2 & Exponentiation
                                                                   (base on the left side, power on the right side) \\

    \texttt{\textbf{*}}                                      & 4 & Multiplication \\
    \texttt{\textbf{/}}                                      & 4 & Division \\
    \texttt{\textbf{\%}}                                     & 4 & Modulo \\

    \texttt{\textbf{+}}                                      & 5 & Addition \\
    \texttt{\textbf{-}} (hyphen-minus)                       & 5 & Subtraction \\

    \texttt{\textbf{|}} (vertical line)                      & 6 & Bitwise or \\
    \texttt{\textbf{\textasciicircum{}}} (circumflex accent) & 6 & Bitwise xor \\
    \texttt{\textbf{\&}}                                     & 6 & Bitwise and \\

    \texttt{\textbf{==}} (dual equals sign)                  & 7 & Equality \\
    \texttt{\textbf{!=}}                                     & 7 & Inequality \\
    \texttt{\textbf{<=}}                                     & 7 & Less or equal \\
    \texttt{\textbf{>=}}                                     & 7 & Greater or equal \\
    \texttt{\textbf{<}}                                      & 7 & Less \\
    \texttt{\textbf{>}}                                      & 7 & Greater \\

    \texttt{\textbf{||}} (dual vertical line)                & 9 & Logical or \\
    \texttt{\textbf{\&\&}}                                   & 9 & Logical and \\
\end{UAVCANSimpleTable}

\subsection{Literals}

Upon its evaluation, a literal yields an object of a particular type depending on the syntax of the literal,
as specified in this section.

\subsubsection{Boolean literals}

A boolean literal is denoted by the keyword ``\verb|true|'' or ``\verb|false|''
represented by an instance of primitive type ``\verb|bool|'' (section~\ref{sec:dsdl_primitive_types})
with an appropriate value.

\subsubsection{Numeric literals}

Integer and real literals are represented as instances of type ``\verb|rational|'' (section~\ref{sec:dsdl_rational}).

The digit separator character ``\verb|_|'' (underscore) does not affect the interpretation of numeric literals.

The significand of a real literal is formed by the integer part, the optional decimal point,
and the optional fraction part;
either the integer part or the fraction part (not both) can be omitted.
The exponent is optionally specified after the letter ``\verb|e|'' or ``\verb|E|'';
it indicates the power of 10 by which the significand is to be scaled.
Either the decimal point or the letter ``\verb|e|''/``\verb|E|'' with the following exponent
(not both) can be omitted from a real literal.

\begin{remark}
    An integer literal \verb|0x123| is represented internally as $\frac{291}{1}$.

    A real literal \verb|.3141592653589793e+1| is represented internally as
    $\frac{3141592653589793}{1000000000000000}$.
\end{remark}

\subsubsection{String literals}

String literals are represented as instances of type ``\verb|string|'' (section~\ref{sec:dsdl_string}).

A string literal is allowed to contain an arbitrary sequence of Unicode characters,
excepting escape sequences defined in table~\ref{table:dsdl_string_literal_escape}
which shall follow one of the specified therein forms.
An escape sequence begins with the ASCII backslash character ``\verb|\|''.

\begin{UAVCANSimpleTable}{String literal escape sequences}{|l X|}
    Sequence & Interpretation
    \label{table:dsdl_string_literal_escape} \\

    \texttt{\textbackslash{}\textbackslash{}}   & Backslash, ASCII code 92. Same as the escape character. \\
    \texttt{\textbackslash{}r}                  & Carriage return, ASCII code 13.               \\
    \texttt{\textbackslash{}n}                  & Line feed, ASCII code 10.                     \\
    \texttt{\textbackslash{}t}                  & Horizontal tabulation, ASCII code 9.          \\

    \texttt{\textbackslash{}\textquotesingle{}} &
    Apostrophe (single quote), ASCII code 39. Regardless of the type of quotes around the literal. \\

    \texttt{\textbackslash{}\textquotedbl{}}    &
    Quotation mark (double quote), ASCII code 34. Regardless of the type of quotes around the literal. \\

    \texttt{\textbackslash{}u????} &
    Unicode symbol with the code point specified by a four-digit hexadecimal number.
    The placeholder ``\texttt{?}'' represents a hexadecimal character \texttt{[0-9a-fA-F]}. \\

    \texttt{\textbackslash{}U????????} &
    Like above, the code point is specified by an eight-digit hexadecimal number. \\

\end{UAVCANSimpleTable}

\begin{remark}
    \begin{minted}{python}
        @assert "oh,\u0020hi\U0000000aMark" == 'oh, hi\nMark'
    \end{minted}
\end{remark}

\subsubsection{Set literals}

Set literals are represented as instances of type ``\verb|set|'' (section~\ref{sec:dsdl_set})
parameterized by the type of the contained elements which is determined automatically.

A set literal declaration shall specify at least one element,
which is used to determine the element type of the set.

The elements of a set literal are defined as DSDL expressions which are evaluated before a set is constructed
from the corresponding literal.

\begin{remark}
    \begin{minted}{python}
        @assert {"cells", 'interlinked'} == {"inter" + "linked", 'cells'}
    \end{minted}
\end{remark}

\subsection{Reserved identifiers}\label{sec:dsdl_reserved_identifiers}

DSDL identifiers and data type name components that match any of the
case-insensitive patterns specified in table~\ref{table:dsdl_reserved_word_patterns}
cannot be used to name new entities.
The semantics of such identifiers is predefined by the DSDL specification,
and as such, they cannot be used for other purposes.
Some of the reserved identifiers do not have any functions associated with them
in this version of the DSDL specification, but this may change in the future.

\begin{UAVCANSimpleTable}{Reserved identifier patterns (POSIX ERE notation, ASCII character set, case-insensitive)}%
    {|l l X|}%
    \label{table:dsdl_reserved_word_patterns}%
    POSIX ERE ASCII pattern                            & Example            & Special meaning \\
    \texttt{truncated}                                 &                    & Cast mode specifier \\
    \texttt{saturated}                                 &                    & Cast mode specifier \\
    \texttt{true}                                      &                    & Boolean literal \\
    \texttt{false}                                     &                    & Boolean literal \\
    \texttt{bool}                                      &                    & Primitive type category \\
    \texttt{u?int\textbackslash{}d*}                   & \texttt{uint8}     & Primitive type category \\
    \texttt{float\textbackslash{}d*}                   & \texttt{float}     & Primitive type category \\
    \texttt{u?q\textbackslash{}d+\_\textbackslash{}d+} & \texttt{q16\_8}    & Primitive type category (future) \\
    \texttt{void\textbackslash{}d*}                    & \texttt{void}      & Void type category \\
    \texttt{optional}                                  &                    & Reserved for future use \\
    \texttt{aligned}                                   &                    & Reserved for future use \\
    \texttt{const}                                     &                    & Reserved for future use \\
    \texttt{struct}                                    &                    & Reserved for future use \\
    \texttt{super}                                     &                    & Reserved for future use \\
    \texttt{template}                                  &                    & Reserved for future use \\
    \texttt{enum}                                      &                    & Reserved for future use \\
    \texttt{self}                                      &                    & Reserved for future use \\
    \texttt{and}                                       &                    & Reserved for future use \\
    \texttt{or}                                        &                    & Reserved for future use \\
    \texttt{not}                                       &                    & Reserved for future use \\
    \texttt{auto}                                      &                    & Reserved for future use \\
    \texttt{type}                                      &                    & Reserved for future use \\
    \texttt{con}                                       &                    & Compatibility with Microsoft Windows \\
    \texttt{prn}                                       &                    & Compatibility with Microsoft Windows \\
    \texttt{aux}                                       &                    & Compatibility with Microsoft Windows \\
    \texttt{nul}                                       &                    & Compatibility with Microsoft Windows \\
    \texttt{com\textbackslash{}d}                      & \texttt{com1}      & Compatibility with Microsoft Windows \\
    \texttt{lpt\textbackslash{}d}                      & \texttt{lpt9}      & Compatibility with Microsoft Windows \\
    \texttt{\_.*\_}                                    & \texttt{\_offset\_}& Special-purpose intrinsic entities \\
\end{UAVCANSimpleTable}

\input{dsdl/expression_types.tex}
\input{dsdl/serializable_types.tex}
\section{Attributes}\label{sec:dsdl_attributes}

\emph{attribute}는 특정 객체나 타입과 관련된 이름이 붙여진 엔트티이다.(패딩 필드는 제외)

\subsection{Composite type attributes}

데이터 타입 정의 시간에 할당되는 값을 가지는 컴포지트 타입 속성을 \emph{constant attribute}라고 부른다.
데이터 타입 정의 시간에 할당되는 않는 컴포지트 타입 속성을 \emph{field attribute}라고 부른다.

컴포지트 타입 속성의 이름은 이를 포함하는 데이터 타입 정의 내에서는 유일한 이름이 되며,
table
\ref{table:dsdl_reserved_word_patterns}에서 정의한 예약 이름 패턴과 매칭되지 않는다.
이런 요구사항은 패딩 필드에 적용되지 않는다.

\subsubsection{Field attributes}

필드 속성은 정적으로 정의된 타입의 동적으로 할당된 값을 나타내며 포함하고 있는 객체의 하나의 멤버로 네트워크 상에서 교환이 가능하다.
필드 속성의 데이터 타입은 직렬화 가능한 타입의 카테고리(section~\ref{sec:dsdl_serializable_types})가 될 수 있으며,
void 타입 카테고리는 허용하지 않아서 제외된다.

필드 속성의 특수한 종류--- \emph{padding fields}에 예외가 적용된다.
패딩 필드 속성의 타입은 void 카테고리의 타입이 된다.
패딩 필드 속성은 이름을 갖지 않을 것이다.

필드 속성 한쌍이 동일하다고 판단되는 경우는 양쪽 모두 동일한 속성이고 모두 동일한 이름 이거나 패딩 필드 속성인 경우이다.

\begin{remark}
    Example:
    \begin{minted}{python}
        uint8[<=10] regular_field   # A field named "regular field"
        void16                      # A padding field; no name is permitted
    \end{minted}
\end{remark}

\subsubsection{Constant attributes}

상수 속성은 정적으로 정의된 타입에 이름을 갖는 정적으로 할당된 값을 표현한다.
상수 속성의 값은 네트워크 상에서 절대로 교환할 수 없다.
왜냐하면 데이터 타입의 동일한 정의를 공유하는 방식으로 관련된 모든 노드들이 모두 사전에 알고 있다고 가정한다.

상수 속성의 데이터 타입은 원시 타입이된다.
카테고리
(section~\ref{sec:dsdl_serializable_types})

상수 속성의 값은 \emph{initialization expression}를 evaluation으로 DSDL 정의 처리 시점에 결정된다.

구문은 상수 속성의 값을 초기화하기 위해서 evaluation에 따라서 호환되는 타입을 생성한다.
호환 타입의 집합은 초기화되는 상수 속성의 타입에 의존하며 table~\ref{table:dsdl_constant_init_pattern}에서 지정되어 있다.

\begin{UAVCANSimpleTable}[wide]{Permitted constant attribute value initialization patterns}{|l | X | X[2] | X[2]|}
    \diagbox[font=\footnotesize]{Constant\\type\\category}{Expression\\type} &
    \texttt{bool} & \texttt{rational} & \texttt{string} \\

    \textbf{Boolean} &
    Allowed. &
    Not allowed. &
    Not allowed. \\

    \textbf{Integer} &
    Not allowed. &
    Allowed if the denominator equals one and the numerator value is within the range of the constant type. &
    Allowed if the target type is \texttt{uint8} and the source string contains one symbol whose code point falls
    into the range $[0, 127]$. \\

    \textbf{Floating point} &
    Not allowed. &
    Allowed if the source value does not exceed the finite range of the constant type.
    The final value is computed as the quotient of the numerator and the denominator
    with implementation-defined accuracy. &
    Not allowed. \label{table:dsdl_constant_init_pattern}\\

\end{UAVCANSimpleTable}

데이터 타입 정의 시점에 정의한 상수 속성의 값으로 인해,
원시 타입으로 된 상수의 캐스트 모드는 관찰 영향을 가지고 있지 않다.

\begin{remark}
    A real literal \verb|1234.5678| is represented internally as
    $\frac{6172839}{5000}$, which can be used to initialize a \verb|float16| value,
    resulting in $1235.0$.

    The specification states that the value of a floating-point constant should be computed
    with an implementation-defined accuracy. UAVCAN avoids strict accuracy requirements in order to
    ensure compatibility with implementations that rely on non-standard floating point formats.
    Such laxity in the specification is considered acceptable since the uncertainty is always
    confined to a single division expression per constant; all preceding computations, if any,
    are always performed by the DSDL compiler using exact rational arithmetic.
\end{remark}

\subsection{Local attributes}\label{sec:dsdl_local_attributes}

로컬 속성은 DSDL 정의 처리 시간에 유효하다.
section~\ref{sec:dsdl_grammar}에 정의된 것과 같이,
DSDL 정의는 순서를 가지는 구문들을 모아둔 것이다.
구문 넘버 $E$ 에 포함된 expression은 이름으로 구문 넘버 $A$ 에서 말한 컴포지트 타입 속성을 뜻한다. 여기서 $A < E$ 와 양쪽 구문이 동일한 데이터 타입 definition\footnote{
    Per \ref{sec:dsdl_elements_of_data_type_definition},
    in case of services, this applies only to their request and response types.
}에 속한다.
참조한 DSDL expression의 컨텍스트에서 참조한 속성의 표현은 table~\ref{table:dsdl_local_attribute_representation}에 지정되어 있다.

\begin{UAVCANSimpleTable}{Local attribute representation}{|l X X|}\label{table:dsdl_local_attribute_representation}%
    Attribute category & Value type & Value \\

    Constant attribute &
    Type of the constant attribute &
    Value of the constant attribute \\

%    Field attribute &
%    \texttt{metaserializable} &
%    Type of the field attribute \\
    Field attribute &
    Illegal &
    Illegal \\

\end{UAVCANSimpleTable}

\begin{remark}
    \begin{minted}{python}
        uint8 FOO = 123
        uint16 BAR = FOO ** 2
        @assert BAR == 15129
        ---  # The request type ends here; its attributes are no longer accessible.
        #uint16 BAZ = BAR  # Would fail - BAR is not accessible here.
        float64 FOO = 3.14
        @assert FOO == 3.14
    \end{minted}
\end{remark}

\subsection{Intrinsic attributes}

고유한 속성은 어떤 expression에서 든 유효하다.
이 값들은 컨텍스트에 따라서 DSDL 처리 툴에 의해서 생성되며,
이번 섹션에서 다룬다.

\subsubsection{Offset attribute}

오프셋 속성은 식별자 ``\verb|_offset_|''를 뜻한다.
이 값은 type $\texttt{set}_\texttt{<rational>}$의 값이다.

다음 텍스트에서 용어 \emph{referring statement}은 오프셋 속성을 참조하는 expression을 포함하는 구문을 뜻한다.
용어 \emph{bit length set}은 section~\ref{sec:dsdl_bit_length_set}에 정의되어 있다.

속성의 값은 필드 속성 선언의 기능으로 참조 구문 앞에 위치하고 포함하는 정의의 카테고리이다.

현재 정의가 태그된 union 카테고리에 속한다면,
참조 구문은 최종 필드 속성 정의 뒤에 위치하게 된다.
참조 구문에 이후에 있는 필드 속성 정의는 현재 정의가 유효하지 않다고 판단하게 된다.
태그 unions에 대해서 오프셋 속성의 값은 unions 필드의 누적 bit length set\footnote{Section \ref{sec:dsdl_composite_alignment_cumulative_bls}}로 정의한다. 여기서 해당 집합의 각 엘리멘트는 내포된 union 태그 필드의(section \ref{sec:dsdl_serialization_composite}) bit length만큼 증가시킨다.

현재 데이터 정의는 태그된 union 카테고리에 속하지 않지만,
참조 구문은 현재 정의 내부에 어떤 곳이든 위치할 수 있다.
오프셋 속성의 값은 참조 구문(see section~\ref{sec:dsdl_grammar} on statement ordering) 앞에 있는 구문에서 필드의 누적 bit length set\footnote{Section \ref{sec:dsdl_composite_alignment_cumulative_bls}.}로 정의된다.

\begin{remark}
    \begin{minted}{python}
        @union
        uint8 a
        #@print _offset_  # Would fail: it's a tagged union, _offset_ is undefined until after the last field
        uint16 b
        @assert _offset_ == {8 + 8,  8 + 16}
        ---
        @assert _offset_ == {0}
        float16 a
        @assert _offset_ == {16}
        void4
        @assert _offset_ == {20}
        int4 b
        @assert _offset_ == {24}
        uint8[<4] c
        @assert _offset_ == 8 + {24,  32,  40,  48}
        @assert _offset_ % 8 == {0}
        # One of the main usages for _offset_ is statically proving that the following field is byte-aligned
        # for all possible valid serialized representations of the preceding fields. It is done by computing
        # a remainder as shown above. If the field is aligned, the remainder set will equal {0}. If the
        # remainder set contains other elements, the field may be misaligned under some circumstances.
        # If the remainder set does not contain zero, the field is never aligned.
        uint8 well_aligned   # Proven to be byte-aligned.
    \end{minted}
\end{remark}

\input{dsdl/directives.tex}
\section{Data serialization}\label{sec:dsdl_data_serialization}

\newcommand{\hugett}[1]{\texttt{\huge{#1}}}

\subsection{General principles}

\subsubsection{Design goals}

이 섹션에 설명하는 직렬화 표현에 대한 주요 디자인 원칙은 현재와 향후 컴퓨터 아키텍쳐에서 사용할 네이티브 표현과 최대한 호환되도록 하는 것이다.
DSDL에서 정의하는 직렬화 표현이 현대 컴퓨터의 내부 데이터 표현과 매칭되도록 하는 것이 목표이다. 따라서 이상적으로는 UAVCAN 네트워크 상에서 데이터 교환하는 동안 데이터 변환 수행이 필요없도록 하는 것이다.

이 섹션에서 소개하는 절삭 및 zero 확장 원칙은 구조적 서브타이핑을 촉진하고 하위 호환을 위해서 데이터 타입의 확장성을 가능하게 하기 위해서 설계되었다.
이는 런타임 타입 체크와 장기 안정성 보장 사이의 교환이 발생한다.
이 모델에서는 데이터 타입 호환성이 런타임이 아닌 정적으로 결정된다고 가정한다.

\subsubsection{Bit and byte ordering}

가장 작은 아토믹 데이터 엔티트는 비트(bit)이다.
8개 비트가 하나의 바이트를 형성한다:
바이트 내부에서 비트들은 순서를 가지는데 LSB(least significant bit)가 첫번째(0-th index)오고 MSB(most significant bit)가 마지막에(7-th index)에 온다.

여러 바이트들로 구성되는 숫자 값은 LSB가 먼저 오도록 인코딩하여 정렬한다. 
이런 포맷을 리틀-인디언(little-endian)이라고 한다.

\begin{figure}[H]
    $$
    \overset{\text{bit index}}{%
        \underbrace{%
            \overset{\text{M}}{\overset{7}{\hugett{0}}}
            \overset{6}{\hugett{1}}
            \overset{5}{\hugett{0}}
            \overset{4}{\hugett{1}}
            \overset{3}{\hugett{0}}
            \overset{2}{\hugett{1}}
            \overset{1}{\hugett{0}}
            \overset{\text{L}}{\overset{0}{\hugett{1}}}
        }_\text{least significant byte}%
    }
    \hugett{\ldots}
    \overset{\text{bit index}}{%
        \underbrace{%
            \overset{\text{M}}{\overset{7}{\hugett{0}}}
            \overset{6}{\hugett{1}}
            \overset{5}{\hugett{0}}
            \overset{4}{\hugett{1}}
            \overset{3}{\hugett{0}}
            \overset{2}{\hugett{1}}
            \overset{1}{\hugett{0}}
            \overset{\text{L}}{\overset{0}{\hugett{1}}}
        }_\text{most significant byte}%
    }
    $$
    \caption{Bit and byte ordering\label{fig:dsdl_serialization_bit_ordering}}
\end{figure}

\subsubsection{Implicit truncation of excessive data}\label{sec:dsdl_serialization_implicit_truncation}

직렬화 시킨 표현을 반직렬화할때에 구현에서 사용하지 않는 데이터나 패딩 비트들은 반직렬화에\footnote{%
The presence of unused data should not be considered an error.
} 따라서 그대로 남겨둔다.
직렬화 표현의 전체 크기는 기반 트랜스포트 계층이나 네스티드 객체의 경우에는 \emph{delimiter header}
(section \ref{sec:dsdl_serialization_composite_non_sealed})가 알려준다.

위 요구사항에 따라서 트랜스포트 계층은 직렬화 표현의 마지막에 데이터 크기 제약을 만족시키기 위해서 0 패딩 비트를 추가할 수 있게 된다.
0이 아닌 패딩 비트는 허용하지 않는다. \footnote{%
    Because padding bits may be misinterpreted as part of the serialized representation.
}

\begin{remark}
    Because of implicit truncation a serialized representation constructed from an instance of type $B$ can be
    deserialized into an instance of type $A$ as long as $B$ is a structural subtype of $A$.

    Let $x$ be an instance of data type $B$, which is defined as follows:

    \begin{minted}{python}
        float32 parameter
        float32 variance
    \end{minted}

    Let $A$ be a structural supertype of $B$, being defined as follows:

    \begin{minted}{python}
        float32 parameter
    \end{minted}

    Then the serialized representation of $x$ can be deserialized into an instance of $A$.
    The topic of data type compatibility is explored in detail in section~\ref{sec:dsdl_versioning}.
\end{remark}

\subsubsection{Implicit zero extension of missing data}\label{sec:dsdl_serialization_implicit_zero_extension}

역직렬화 루틴의 목적을 위해
데이터 타입의 인스턴스의 직렬화된 표현은 (0).\footnote{%
    This can be implemented by checking for out-of-bounds access during deserialization and returning zeros
    if an out-of-bounds access is detected. This is where the name ``implicit zero extension rule'' is derived
    from.
}으로 된 무한 연속 비트로 암묵적으로 끝이 난다.
For the purposes of deserialization routines,
the serialized representation of any instance of a data type shall \emph{implicitly} end with an
infinite sequence of bits with a value of zero (0).\footnote{%
    This can be implemented by checking for out-of-bounds access during deserialization and returning zeros
    if an out-of-bounds access is detected. This is where the name ``implicit zero extension rule'' is derived
    from.
}.

Despite this rule, implementations are not allowed to intentionally truncate trailing zeros
upon construction of a serialized representation of an object\footnote{%
    Intentional truncation is prohibited because a future revision of the specification may remove the implicit zero
    extension rule.
    If intentional truncation were allowed, removal of this rule would break backward compatibility.
}.

The total size of the serialized representation is reported either by the underlying transport layer, or,
in the case of nested objects, by the \emph{delimiter header}
(section \ref{sec:dsdl_serialization_composite_non_sealed}).

\begin{remark}
    The implicit zero extension rule enables extension of data types by introducing additional fields
    without breaking backward compatibility with existing deployments.
    The topic of data type compatibility is explored in detail in section~\ref{sec:dsdl_versioning}.

    The following example assumes that the reader is familiar with the variable-length array serialization rules,
    explained in section~\ref{sec:dsdl_serialized_variable_length_array}.

    Let the data type $A$ be defined as follows:

    \begin{minted}{python}
        uint8 scalar
    \end{minted}

    Let $x$ be an instance of $A$, where the value of \verb|scalar| is 4.
    Let the data type $B$ be defined as follows:

    \begin{minted}{python}
        uint8[<256] array
    \end{minted}

    Then the serialized representation of $x$ can be deserialized into an instance of $B$ where the field
    \verb|array| contains a sequence of four zeros: $0, 0, 0, 0$.
\end{remark}

\subsubsection{Error handling}\label{sec:dsdl_serialized_error}

In this section and further, an object that nests other objects is referred to as an \emph{outer object}
in relation to the nested object.

Correct UAVCAN types shall have no serialization error states.

A deserialization process may encounter a serialized representation that does not belong to the
set of serialized representations of the data type at hand.
In such case, the invalid serialized representation shall be discarded and the implementation
shall explicitly report its inability to complete the deserialization process for the given input.
Correct UAVCAN types shall have no other deserialization error states.

Failure to deserialize a nested object renders the outer object invalid\footnote{%
    Therefore, failure in a single deeply nested object propagates upward, rendering the entire structure invalid.
    The motivation for such behavior is that it is likely that if an inner object cannot be deserialized,
    then the outer object is likely to be also invalid.
}.

\subsection{Void types}\label{sec:dsdl_serialized_void}

The serialized representation of a void-typed field attribute is constructed as a sequence of zero bits.
The length of the sequence equals the numeric suffix of the type name.

When a void-typed field attribute is deserialized, the values of respective bits are ignored;
in other words, any bit sequence of correct length is a valid serialized representation
of a void-typed field attribute.
This behavior facilitates usage of void fields as placeholders for non-void fields
introduced in newer versions of the data type (section~\ref{sec:dsdl_versioning}).

\begin{remark}
    The following data type will be serialized as a sequence of three zero bits $000_2$:
    \begin{minted}{python}
        void3
    \end{minted}
    The following bit sequences are valid serialized representations of the type:
    $000_2$,
    $001_2$,
    $010_2$,
    $011_2$,
    $100_2$,
    $101_2$,
    $110_2$,
    $111_2$.

    Shall the padding field be replaced with a non-void-typed field in a future version of the data type,
    nodes utilizing the newer definition may be able to retain compatibility with nodes using older types,
    since the specification guarantees that padding fields are always initialized with zeros:

    \begin{minted}{python}
        # Version 1.1
        float64 a
        void64
    \end{minted}

    \begin{minted}{python}
        # Version 1.2
        float64 a
        float32 b  # Messages v1.1 will be interpreted such that b = 0.0
        void32
    \end{minted}
\end{remark}

\subsection{Primitive types}

\subsubsection{General principles}

Implementations where native data formats are incompatible with those adopted by UAVCAN shall perform
conversions between the native formats and the corresponding UAVCAN formats during
serialization and deserialization.
Implementations shall avoid or minimize information loss and/or distortion caused by such conversions.

Serialized representations of instances of the primitive type category that are longer than one byte (8 bits)
are constructed as follows.
First, only the least significant bytes that contain the used bits of the value are preserved;
the rest are discarded following the lossy assignment policy selected by the specified cast mode.
Then the bytes are arranged in the least-significant-byte-first order\footnote{Also known as ``little endian''.}.
If the bit width of the value is not an integer multiple of eight (8) then the next value in the type will begin
starting with the next bit in the current byte. If there are no further values then the remaining bits
shall be zero (0).

\begin{remark}
    The value $1110\,1101\,1010_2$ (3802 in base-10) of type \verb|uint12| is encoded as follows.
    The bit sequence is shown in the base-2 system, where bytes (octets) are comma-separated:
    $$
        \overset{\text{byte 0}}{%
            \underbrace{%
                \overset{7}{\hugett{1}}
                \overset{6}{\hugett{1}}
                \overset{5}{\hugett{0}}
                \overset{4}{\hugett{1}}
                \overset{3}{\hugett{1}}
                \overset{2}{\hugett{0}}
                \overset{1}{\hugett{1}}
                \overset{0}{\hugett{0}}
            }_{\substack{\text{Least significant 8} \\ \text{bits of }3802_{10}}}%
        }%
        \hugett{,}%
        \overset{\text{byte 1}}{%
            \underbrace{
                \overset{7}{\hugett{?}}
                \overset{6}{\hugett{?}}
                \overset{5}{\hugett{?}}
                \overset{4}{\hugett{?}}
            }_{\substack{\text{Next object} \\ \text{or zero} \\ \text{padding bits}}}%
            \underbrace{
                \overset{3}{\hugett{1}}
                \overset{2}{\hugett{1}}
                \overset{1}{\hugett{1}}
                \overset{0}{\hugett{0}}
            }_{\substack{\text{Most} \\ \text{significant} \\ \text{4 bits of} \\ \text{3802}_{10}}}%
        }
    $$
\end{remark}

\subsubsection{Boolean types}\label{sec:dsdl_serialized_bool}

The serialized representation of a value of type \verb|bool| is a single bit.
If the value represents falsity, the value of the bit is zero (0); otherwise, the value of the bit is one (1).

\subsubsection{Unsigned integer types}\label{sec:dsdl_serialized_unsigned_integer}

The serialized representation of an unsigned integer value of length $n$ bits
(which is reflected in the numerical suffix of the data type name)
is constructed as if the number were to be written in base-2 numerical system
with leading zeros preserved so that the total number of binary digits would equal $n$.

\begin{remark}
    The serialized representation of integer 42 of type \verb|uint7| is $0101010_2$.
\end{remark}

\subsubsection{Signed integer types}

The serialized representation of a non-negative value of a signed integer type is constructed as described
in section~\ref{sec:dsdl_serialized_unsigned_integer}.

The serialized representation of a negative value of a signed integer type is computed by
applying the following transformation:
$$2^n + x$$
where $n$ is the bit length of the serialized representation
(which is reflected in the numerical suffix of the data type name)
and $x$ is the value whose serialized representation is being constructed.
The result of the transformation is a positive number,
whose serialized representation is then constructed as described in section~\ref{sec:dsdl_serialized_unsigned_integer}.

The representation described here is widely known as \emph{two's complement}.

\begin{remark}
    The serialized representation of integer -42 of type \verb|int7| is $1010110_2$.
\end{remark}

\subsubsection{Floating point types}

The serialized representation of floating point types follows the IEEE 754 series of standards as follows:

\begin{itemize}
    \item \verb|float16| --- IEEE 754 binary16;
    \item \verb|float32| --- IEEE 754 binary32;
    \item \verb|float64| --- IEEE 754 binary64.
\end{itemize}

Implementations that model real numbers using any method other than IEEE 754 shall be able to model
positive infinity, negative infinity, signaling NaN\footnote{%
    Per the IEEE 754 standard, NaN stands for
    ``not-a-number'' -- a set of special bit patterns that represent lack of a meaningful value.
}, and quiet NaN.

\subsection{Array types}

\subsubsection{Fixed-length array types}

Serialized representations of a fixed-length array of $n$ elements of type $T$ and
a sequence of $n$ field attributes of type $T$ are equivalent.

\begin{remark}
    Serialized representations of the following two data type definitions are equivalent:

    \begin{minted}{python}
        AnyType[3] array
    \end{minted}

    \begin{minted}{python}
        AnyType item_0
        AnyType item_1
        AnyType item_2
    \end{minted}
\end{remark}

\subsubsection{Variable-length array types}\label{sec:dsdl_serialized_variable_length_array}

A serialized representation of a variable-length array consists of two segments:
the implicit length field immediately followed by the array elements.

The implicit length field is of an unsigned integer type.
The serialized representation of the implicit length field
is injected in the beginning of the serialized representation of its array.
The bit length of the unsigned integer value is first determined as follows:

$$b=\lceil{}\log_2 (c + 1)\rceil{}$$

where $c$ is the capacity (i.e., the maximum number of elements) of the variable-length array and
$b$ is the minimum number of bits needed to encode $c$ as an unsigned integer. An additional transformation
of $b$ ensures byte alignment of this implicit field when serialized\footnote{Future updates to the specification
may allow this second step to be modified but the default action will always be to byte-align the implicit
length field.}:

$$2^{\lceil{}\log_2 (\text{max}(8, b))\rceil{}}$$

The number of elements $n$ contained in the variable-length array is encoded
in the serialized representation of the implicit length field
as described in section~\ref{sec:dsdl_serialized_unsigned_integer}.
By definition, $n \leq c$; therefore, bit sequences where the implicit length field contains values
greater than $c$ do not belong to the set of serialized representations of the array.

The rest of the serialized representation is constructed as if the variable-length array was
a fixed-length array of $n$ elements\footnote{%
    Observe that the implicit array length field, per its definition,
    is guaranteed to never break the alignment of the following array elements.
    There may be no padding between the implicit array length field and its elements.
}.

\begin{remark}
    Data type authors must take into account that variable-length arrays with a capacity of $\leq{}255$ elements will
    consume an additional 8 bits of the serialized representation
    (where a capacity of $\leq 65535$ will consume 16 bits and so on).
    For example:

    \begin{minted}{python}
        uint8 first
        uint8[<=6] second              # The implicit length field is 8 bits wide
        @assert _offset_.max / 8 <= 7  # This would fail.
    \end{minted}

    In the above example the author attempted to fit the message into a single Classic CAN frame but
    did not account for the implicit length field. The correct version would be:

    \begin{minted}{python}
        uint8 first
        uint8[<=5] second              # The implicit length field is 8 bits wide
        @assert _offset_.max / 8 <= 7  # This would pass.
    \end{minted}

    If the array contained three elements, the resulting set of its serialized representations would
    be equivalent to that of the following definition:

    \begin{minted}{python}
        uint8 first
        uint8 implicit_length_field  # Set to 3, because the array contains three elements
        uint8 item_0
        uint8 item_1
        uint8 item_2
    \end{minted}
\end{remark}

\subsection{Composite types}\label{sec:dsdl_serialization_composite}

\subsubsection{Sealed structure}

A serialized representation of an object of a sealed composite type that is not a tagged union
is a sequence of serialized representations of its field attribute values joined into a bit sequence,
separated by padding if such is necessary to satisfy the alignment requirements.
The ordering of the serialized representations of the field attribute values follows the order
of field attribute declaration.

\begin{remark}
    Consider the following definition,
    where the fields are assigned runtime values shown in the comments:

    \begin{minted}{python}
        #                          decimal           bit sequence   comment
        truncated uint12 first   # +48858     1011_1110_1101_1010   overflow, MSB truncated
        saturated  int3  second  #     -1                     111   two's complement
        saturated  int4  third   #     -5                    1011   two's complement
        saturated  int2  fourth  #     -1                      11   two's complement
        truncated uint4  fifth   #   +136               1000_1000   overflow, MSB truncated
        @sealed
    \end{minted}

    It can be seen that the bit layout is rather complicated because the field boundaries do not align with byte
    boundaries, which makes it a good case study.
    The resulting serialized byte sequence is shown below in the base-2 system:
    $$
        \underbrace{%
            \overbrace{%
                \underset{7}{\overset{7}{\hugett{1}}}%
                \underset{6}{\overset{6}{\hugett{1}}}%
                \underset{5}{\overset{5}{\hugett{0}}}%
                \underset{4}{\overset{4}{\hugett{1}}}%
                \underset{3}{\overset{3}{\hugett{1}}}%
                \underset{2}{\overset{2}{\hugett{0}}}%
                \underset{1}{\overset{1}{\hugett{1}}}%
                \underset{0}{\overset{0}{\hugett{0}}}%
            }^{\texttt{first}}%
        }_{\texttt{byte 0}}%
        \hugett{,}%
        \underbrace{%
            \overbrace{%
                \underset{7}{\overset{0}{\hugett{1}}}%
            }^{\texttt{third}}%
            \overbrace{%
                \underset{6}{\overset{2}{\hugett{1}}}%
                \underset{5}{\overset{1}{\hugett{1}}}%
                \underset{4}{\overset{0}{\hugett{1}}}%
            }^{\texttt{second}}%
            \overbrace{%
                \underset{3}{\overset{11}{\hugett{1}}}%
                \underset{2}{\overset{10}{\hugett{1}}}%
                \underset{1}{\overset{9}{\hugett{1}}}%
                \underset{0}{\overset{8}{\hugett{0}}}%
            }^{\texttt{first}}%
        }_{\texttt{byte 1}}%
        \hugett{,}%
        \underbrace{%
            \overbrace{%
                \underset{7}{\overset{2}{\hugett{0}}}%
                \underset{6}{\overset{1}{\hugett{0}}}%
                \underset{5}{\overset{0}{\hugett{0}}}%
            }^{\texttt{fifth}}%
            \overbrace{%
                \underset{4}{\overset{1}{\hugett{1}}}%
                \underset{3}{\overset{0}{\hugett{1}}}%
            }^{\texttt{fourth}}%
            \overbrace{%
                \underset{2}{\overset{3}{\hugett{1}}}%
                \underset{1}{\overset{2}{\hugett{0}}}%
                \underset{0}{\overset{1}{\hugett{1}}}%
            }^{\texttt{third}}%
        }_{\texttt{byte 2}}%
        \hugett{,}%
        \underbrace{%
            \overbrace{%
                \underset{7}{\overset{?}{\hugett{?}}}%
                \underset{6}{\overset{?}{\hugett{?}}}%
                \underset{5}{\overset{?}{\hugett{?}}}%
                \underset{4}{\overset{?}{\hugett{?}}}%
                \underset{3}{\overset{?}{\hugett{?}}}%
                \underset{2}{\overset{?}{\hugett{?}}}%
                \underset{1}{\overset{?}{\hugett{?}}}%
            }^{\substack{\text{Next object or} \\ \text{zero padding bits}}}
            \overbrace{%
                \underset{0}{\overset{3}{\hugett{1}}}%
            }^{\texttt{fifth}}%
        }_{\texttt{byte 3}}%
    $$

    Note that some of the complexity of the above illustration stems from the modern convention of representing
    numbers with the most significant components on the left moving to the least significant component of the
    number of the right. If you were to reverse this convention the bit sequences for each type in the composite
    would seem to be continuous as they crossed byte boundaries. Using this reversed representation, however, is
    not recommended because the convention is deeply ingrained in most readers, tools, and technologies.
\end{remark}

\subsubsection{Sealed tagged union}

Similar to variable-length arrays, a serialized representation of a sealed tagged union consists of two segments:
the implicit \emph{union tag} value followed by the selected field attribute value.

The implicit union tag is an unsigned integer value whose serialized representation
is implicitly injected in the beginning of the serialized representation of its tagged union.
The bit length of the implicit union tag is determined as follows:
$$b=\lceil{}\log_2 n\rceil{}$$
where $n$ is the number of field attributes in the union, $n \geq 2$ and $b$ is the minimum number of bits needed
to encode $n$ as an unsigned integer. An additional transformation of $b$ ensures byte alignment of this implicit
field when serialized\footnote{Future updates to the specification may allow this second step to be modified but
the default action will always be to byte-align the implicit length field.}:

$$2^{\lceil{}\log_2 (\text{max}(8, b))\rceil{}}$$

Each of the tagged union field attributes is assigned an index according to the order of their definition;
the order follows that of the DSDL statements (see section~\ref{sec:dsdl_grammar} on statement ordering).
The first defined field attribute is assigned the index 0 (zero),
the index of each following field attribute is incremented by one.

The index of the field attribute whose value is currently held by the tagged union is encoded
in the serialized representation of the implicit union tag as described in section
\ref{sec:dsdl_serialized_unsigned_integer}.
By definition, $i < n$, where $i$ is the index of the current field attribute;
therefore, bit sequences where the implicit union tag field contains values
that are greater than or equal $n$ do not belong to the set of serialized representations of the tagged union.

The serialized representation of the implicit union tag is immediately followed by
the serialized representation of the currently selected field attribute value\footnote{%
    Observe that the implicit union tag field, per its definition,
    is guaranteed to never break the alignment of the following field.
    There may be no padding between the implicit union tag field and the selected field.
}.

\begin{remark}
    Consider the following example:

    \begin{minted}{python}
        @sealed
        @union           # In this case, the implicit union tag is one byte wide
        uint16 FOO = 42  # A regular constant attribute
        uint16 a         # Field index 0
        uint8 b          # Field index 1
        uint32 BAR = 42  # Another regular constant attribute
        float64 c        # Field index 2
    \end{minted}

    In order to serialize the field \verb|b|, the implicit union tag shall be assigned the value 1.
    The following type will have an identical layout:

    \begin{minted}{python}
        @sealed
        uint8 implicit_union_tag  # Set to 1
        uint8 b                   # The actual value
    \end{minted}

    Suppose that the value of \verb|b| is 7.
    The resulting serialized representation is shown below in the base-2 system:
    $$%
    \overset{\text{byte 0}}{%
        \underbrace{\hugett{00000001}}_{\substack{\text{union} \\ \text{tag}}}%
    }%
    \hugett{,}%
    \overset{\text{byte 1}}{%
        \underbrace{\hugett{00000111}}_{\text{field }\texttt{b}}%
    }
    $$

\end{remark}

\begin{remark}
    Let the following data type be defined under the short name \verb|Empty| and version 1.0:

    \begin{minted}{python}
        # Empty. The only valid serialized representation is an empty bit sequence.
        @sealed
    \end{minted}

    Consider the following union:

    \begin{minted}{python}
        @sealed
        @union
        Empty.1.0 none
        AnyType.1.0 some
    \end{minted}

    The set of serialized representations of the union given above is equivalent to
    that of the following variable-length array:

    \begin{minted}{python}
        @sealed
        AnyType.1.0[<=1] maybe_some
    \end{minted}
\end{remark}

\subsubsection{Delimited types}\label{sec:dsdl_serialization_composite_non_sealed}

Objects of delimited (non-sealed) composite types that are nested inside other objects\footnote{%
    Of any type, not necessarily composite; e.g., arrays.
}
are serialized into opaque containers that consist of two parts:
the fixed-length \emph{delimiter header},
immediately followed by the serialized representation of the object as if it was of a sealed type.

Objects of delimited composite types that are \emph{not} nested inside other objects (i.e., top-level objects)
are serialized as if they were of a sealed type (without the delimiter header).
The delimiter header, therefore, logically belongs to the container object rather than the contained one.

\begin{remark}
    Top-level objects do not require the delimiter header because the change in their length does not necessarily
    affect the backward compatibility thanks to the implicit truncation rule
    (section \ref{sec:dsdl_serialization_implicit_truncation}) and the implicit zero extension rule
    (section \ref{sec:dsdl_serialization_implicit_zero_extension}).
\end{remark}

The delimiter header is an implicit field of type \verb|uint32| that encodes the length of the
serialized representation it precedes in bytes\footnote{%
    Remember that by virtue of the padding requirement (section \ref{sec:dsdl_composite_alignment_cumulative_bls}),
    the length of the serialized representation of a composite type is always an integer number of bytes.
}.
During deserialization, if the length of the serialized representation reported by its delimiter header
does not match the expectation of the deserializer,
the implicit truncation (section \ref{sec:dsdl_serialization_implicit_truncation})
and the implicit zero extension (section \ref{sec:dsdl_serialization_implicit_zero_extension})
rules apply.

The length encoded in a delimiter header cannot exceed the number of bytes remaining between the delimiter header
and the end of the serialized representation of the outer object.
Otherwise, the serialized representation of the outer object is invalid and is to be discarded
(section \ref{sec:dsdl_serialized_error}).

It is allowed for a sealed composite type to nest non-sealed composite types, and vice versa.
No special rules apply in such cases.

\begin{remark}
    The resulting serialized representation of a delimited composite is identical to \verb|uint8[<2**32]|
    (sans the higher alignment requirement).
    The implicit array length field is like the delimiter header,
    and the array content is the serialized representation of the composite as if it was sealed.

    The following illustrates why this is necessary for robust extensibility.
    Suppose that some composite $C$ contains two fields whose types are $A$ and $B$.
    The fields of $A$ are $a_0,\ a_1$;
    likewise, $B$ contains $b_0,\ b_1$.

    Suppose that $C^\prime$ is modified such that $A^\prime$ contains an extra field $a_2$.
    If $A$ (and $A^\prime$) were sealed, this would result in the breakage of compatibility between $C$ and $C^\prime$
    as illustrated in figure \ref{fig:dsdl_sealed_non_extensibility} because the positions of the fields of $B$
    (which is sealed) would be shifted by the size of $a_2$.

    The use of opaque containers allows the implicit truncation and the implicit zero extension rules to apply
    at any level of nesting, enabling agents expecting $C$ to truncate $a_2$ away,
    and enabling agents expecting $C^\prime$ to zero-extend $a_2$
    if it is not present, as shown in figure \ref{fig:dsdl_non_sealed_extensibility},
    where $H_A$ is the delimiter header of $A$.
    Observe that it is irrelevant whether $C$ (same as $C^\prime$) is sealed or not.

    \begin{figure}[H]
        \centering
        \begin{tabular}{r c c c c c}
            \cline{2-5}
            $C$ &
            \multicolumn{1}{|c|}{$a_0$} & \multicolumn{1}{c|}{$a_1$}
            &\multicolumn{1}{c|}{$b_0$} & \multicolumn{1}{c|}{$b_1$} &
            \\\cline{2-5}
            & $\checkmark$ & $\checkmark$ & $\times$ & $\times$ & $\times$ \\
            \cline{2-6}
            $C^\prime$ &
            \multicolumn{1}{|c|}{$a_0$} & \multicolumn{1}{c|}{$a_1$} & \multicolumn{1}{c|}{$a_2$}
            &\multicolumn{1}{c|}{$b_0$} & \multicolumn{1}{c|}{$b_1$}
            \\\cline{2-6}
        \end{tabular}
        \caption{Non-extensibility of sealed types}
        \label{fig:dsdl_sealed_non_extensibility}
    \end{figure}

    \begin{figure}[H]
        \centering
        \begin{tabular}{r c c c c c c}
            \cline{2-7}
            $C$ &
            \multicolumn{1}{|c|}{$H_A$} & \multicolumn{1}{c|}{$a_0$} & \multicolumn{1}{c|}{$a_1$}
            &\multicolumn{1}{c|}{\footnotesize{$\ldots$}}
            &\multicolumn{1}{c|}{$b_0$} & \multicolumn{1}{c|}{$b_1$}
            \\\cline{2-7}
            & $\checkmark$ & $\checkmark$ & $\checkmark$ & $\checkmark$ & $\checkmark$ & $\checkmark$ \\
            \cline{2-7}
            $C^\prime$ &
            \multicolumn{1}{|c|}{$H_A$} & \multicolumn{1}{c|}{$a_0$} & \multicolumn{1}{c|}{$a_1$} &
            \multicolumn{1}{c|}{$a_2$}
            &\multicolumn{1}{c|}{$b_0$} & \multicolumn{1}{c|}{$b_1$}
            \\\cline{2-7}
        \end{tabular}
        \caption{Extensibility of delimited types with the help of the delimiter header}
        \label{fig:dsdl_non_sealed_extensibility}
    \end{figure}

    This example also illustrates why the extent is necessary.
    Per the rules set forth in \ref{sec:dsdl_composite_extent_and_sealing},
    it is required that the extent (i.e., the buffer memory requirement) of $A$ shall be large enough to accommodate
    serialized representations of $A^\prime$, and, therefore,
    the extent of $C$ is large enough to accommodate serialized representations of $C^\prime$.
    If that were not the case, then an implementation expecting $C$ would be unable to correctly process $C^\prime$
    because the implicit truncation rule would have cut off $b_1$, which is unexpected.

    The design decision to make the delimiter header of a fixed width may not be obvious so it's worth explaining.
    There are two alternatives: making it variable-length and making the length a function of the extent
    (section \ref{sec:dsdl_composite_extent_and_sealing}).
    The first option does not align with the rest of the specification because DSDL does not make use of
    variable-length integers (unlike some other formats, like Google Protobuf, for example),
    and because a variable-length length {\footnotesize{(sic!)}} prefix would have somewhat complicated the
    bit length set computation.
    The second option would make nested hierarchies (composites that nest other composites) possibly highly fragile
    because the change of the extent of a deeply nested type may inadvertently move the delimiter header of an
    outer type into a different length category, which would be disastrous for compatibility and hard to spot.
    There is an in-depth discussion of this issue (and other related matters) on the forum.

    The fixed-length delimiter header may be considered large,
    but delimited types tend to also be complex, which makes the overhead comparatively insignificant,
    whereas sealed types that tend to be compact and overhead-sensitive do not contain the delimiter header.
\end{remark}

\begin{remark}
    In order to efficiently serialize an object of a delimited type,
    the implementation may need to perform a second pass to reach the delimiter header
    after the object is serialized, because before that, the value of the delimiter header cannot be known
    unless the object is of a fixed-size (i.e., the cardinality of the bit length set is one).

    Consider:
    \begin{minted}{python}
        uint8[<=4] x
    \end{minted}
    Let $\texttt{x} = \left[ 4, 2 \right]$,
    then the nested serialized representation would be constructed as:
    \begin{enumerate}
        \item Memorize the current memory address $M_\text{origin}$.
        \item Skip 32 bits.
        \item Encode the length: 2 elements.
        \item Encode $x_0 = 4$.
        \item Encode $x_1 = 2$.
        \item Memorize the current memory address $M_\text{current}$.
        \item Go back to $M_\text{origin}$.
        \item Encode a 32-bit wide value of $(M_\text{current} - M_\text{origin})$.
        \item Go back to $M_\text{current}$.
    \end{enumerate}

    However, if the object is known to be of a constant size, the above can be simplified,
    because there may be only one possible value of the delimiter header.
    Automatic code generation tools should take advantage of this knowledge.
\end{remark}

\section{Compatibility and versioning}\label{sec:dsdl_versioning}

\subsection{Rationale}

데이터 타입 정의는 어플리케이션의 니즈에 더 잘 맞도록 개선되면서 시간이 지날수록 진화하게 된다.
UAVCAN은 데이터 타입 디자이너가 하위 호환이나 기능 안정성을 보장하기 위해서 데이터 타입 정의를 수정하고 이용하도록 규칙들을 정의할 수 있다.

\subsection{Semantic compatibility}\label{sec:dsdl_semantic_compatibility}

데이터 타입 $A$은 데이터 타입 $B$와 \emph{semantically compatible}하다고 하는 경우는 만약에 관련 어플리케이션의 동작 속성이 항상 $A$ 를 $B$로 대체 가능한 경우이다.
의미 호환이 가능한 속성은 교환법칙이 성립된다.

\begin{remark}[breakable]
    The following two definitions are semantically compatible and can be used interchangeably:

    \begin{minted}{python}
        uint16 FLAG_A = 1
        uint16 FLAG_B = 256
        uint16 flags
        @extent 16
    \end{minted}

    \begin{minted}{python}
        uint8 FLAG_A = 1
        uint8 FLAG_B = 1
        uint8 flags_a
        uint8 flags_b
        @extent 16
    \end{minted}

    필드와 일정한 속성이 달라지므로 주의를 주는 것이 필요하다.
    제공된 정의로부터 자동생성된 소스 코드는 교체하기 위해서 어플리케이션에서 변경이 필요할 수 있다.;
    하지만 소스 코드 레벨 어플리케이션 호환성은 데이터 타입 호환성과 관련이 없다.


    다음 수퍼타입은 의미상 제거된 필드의 의미에 따라서 위에 있는 것과 의미상 호환이 될 수도 안될 수도 있다.:

    \begin{minted}{python}
        uint8 FLAG_A = 1
        uint8 flags_a
        @extent 16
    \end{minted}
\end{remark}

\begin{remark}
    Let node $A$ publish messages of the following type:

    \begin{minted}{python}
        float32 foo
        float64 bar
        @extent 128
    \end{minted}

    Let node $B$ subscribe to the same subject using the following data type definition:

    \begin{minted}{python}
        float32 foo
        float64 bar
        int16   baz  # Extra field; implicit zero extension rule applies.
        @extent 128
    \end{minted}

    Let node $C$ subscribe to the same subject using the following data type definition:

    \begin{minted}{python}
        float32 foo
        # The field 'bar' is missing; implicit truncation rule applies.
        @extent 128
    \end{minted}

    Provided that the semantics of the added and omitted fields allow it,
    the nodes will be able to interoperate successfully despite using different data type definitions.
\end{remark}

\subsection{Versioning}

\subsubsection{General assumptions}

버전닝의 개념은 컴포지트 데이터 타입에만 적용된다.
특별히 언급하지 않으면 이 섹션에 ``data type''에 대한 모든 레퍼런스는 컴포지트 데이터 타입을 의미한다.

데이터 타입은 전체 이름으로 식별되고 모든 루트 네임스페이스는 유일한 이름을 갖는다고 가정한다.
모든 데이터 타입의 1개 이상의 버전이 있다.

데이터 타입 정의는 전체 이름과 버전 넘버의 쌍으로 식별된다.
달리 말하면, 버전 넘버가 다른 데이터 타입의 다양한 정의가 있을 수 있다.

\subsubsection{Versioning principles}

모든 데이터 타입 정의는 버전 넘버의 쌍을 가진다. ---
메이저 버전 넘버와 마이너 버전 넘버은 의미있는 버전을 매기는 원칙을 따른다.

다음 정의의 목적을 위해서 데이터 타입 정의의 \emph{release}는 의도된 사용자나 일반 대중에게 데이터 타입의 정의를 공개하는 것을 의미하거나 제품 시스템에서 데이터 타입의 정의의 사용을 개시하는 목적이다.

어플리케이션 동작을 보장하고 데이터 타입 정의와 관련된 강건한 마이그레이션 경로를 보장하기 위해서 동일한 전체 이름과 동일한 메이저 버전 넘버를 공유하는 모든 데이터 타입 정의는 의미적으로 서로 호환된다.

버전닝을 매기는 원칙은 데이터 타입의 이름이 변경되는 지점의 시나리오로 확장되지 않는데,
왜냐하면 이렇게 하면 본질적으로 새로운 데이터 타입의 릴리즈로 해석되기 때문이다.
이는 모든 호환 요구사항으로부터 디자이너를 안심시킬 수 있다.
새로운 데이터 타입이 처음으로 릴리즈되는 때에,
처음 정의의 버전 넘버는 ``1.0'' (major 1, minor 0)이 할당된다.

UAVCAN을 활용하여 어플리케이션의 예측가능성과 기능정 안정성을 확보하기 위해서,
일단 데이터 타입 정의가 릴리즈시키는 것을 추천하고 DSDL 소스 텍스트, 이름, 버전 넘버, 고정 port-ID, 확장, 실링 및 기타 속성은 다음과 같은 예외를 제외하고는 수정이 되지 않는다.:
\begin{itemize}
    \item Whitespace changes of the DSDL source text are allowed,
          excepting string literals and other semantically sensitive contexts.

    \item Comment changes of the DSDL source text are allowed as long as such changes
          do not affect semantic compatibility of the definition.

    \item A deprecation marker directive (section~\ref{sec:dsdl_directives}) can be added or removed\footnote{%
              Removal is useful when a decision to deprecate a data type definition is withdrawn.
          }.
\end{itemize}
특정 버전의 데이터 타입 정의가 릴리즈 되고 난 이후에는 고정 port-id를 추가 혹은 제거는 허용되지 않는다.

따라서 큰 변화는 동일한 데이터 타입의 새로운 정의를 릴리즈하는 경우에만 적용된다.(예) 새로운 버전)
데이터 타입의 새로운 정의를 위해서 동일한 메이저 넘버를 유지하는 것이 바람직하고 가능하다면,
새로운 정의의 마이너 버전 넘버는 새로운 정의를 사용하기 전에 가장 최신의 기존 마이너 버전 넘버보다 큰 수로 정할 수 있다.
반면에 메이저 버전 넘버는 1씩 증가하고 마이너 버전은 0으로 설정된다.

메이저 버전 넘버가 0이 되면 위에 원칙에 대한 예외가 적용된다.
메이저 버전 넘버가 0인 데이터 타입 정의는 어떠한 호환성 요구사항도 적용되지 않는다.
메이저 버전 넘버가 0으로 릴리즈된 데이터 타입 정의는 호환성에 대한 고려없이 임의의 방식으로 변경될 수 있다.
하지만 불변성의 원칙을 따르기 위해서 가장 최신의 기존 정의보다 마이너 버전 넘버가 1이 증가된 차기 정의를 릴리즈하는 것을 추천한다.

어떤 데이터 타입에 대해서 버전마다 많아야 하나의 정의가 있다.
달리 말하면 정의는 데이터 타입 이름과 버전 넘버 쌍의 조합마다 하나이거나 없거나 둘 중에 하나이다.

동일한 이름 아래에서 모든 데이터 타입은 동일한 종류가 된다.
달리 말하면 데이터 타입의 첫음 릴리즈된 정의가 메시지 종류 중에 하나라면 모든 버전도 메시지 종류 중에 하나여야 한다.

동일한 이름과 메이저 버전 넘버 아래에서 모든 데이터 타입은 동일한 확장과 동일한 실링 상태를 공유해야만 한다.
따라서 다음과 같이 다음과 같이 권고한다:
\begin{itemize}
    \item Avoid marking types sealed, especially complex types,
    because it is likely to render their evolution impossible.

    \item When the first version is released, its extent should be sufficiently large
    to permit addition of new fields in the future.
    Since the value of extent does not affect the network traffic, it is safe to pick a large value
    without compromising the temporal properties of the system.
\end{itemize}

\subsubsection{Fixed port identifier assignment constraints}

The following constraints apply to fixed port-ID assignments:
\begin{align*}
    \exists P(x_{a.b})                          &\rightarrow \exists P(x_{a.c})
    &\mid&\ b < c;\ x \in (M \cup S)
    \\
    \exists P(x_{a.b})                          &\rightarrow         P(x_{a.b}) =    P(x_{a.c})
    &\mid&\ b < c;\ x \in (M \cup S)
    \\
    \exists P(x_{a.b}) \land \exists P(x_{c.d}) &\rightarrow         P(x_{a.b}) \neq P(x_{c.d})
    &\mid&\ a \neq c;\ x \in (M \cup S)
    \\
    \exists P(x_{a.b}) \land \exists P(y_{c.d}) &\rightarrow         P(x_{a.b}) \neq P(y_{c.d})
    &\mid&\ x \neq y;\ x \in T;\ y \in T;\ T = \left\{ M, S \right\}
\end{align*}
where $t_{a.b}$ denotes a data type $t$ version $a.b$ ($a$ major, $b$ minor);
$P(t)$ denotes the fixed port-ID (whose existence is optional) of data type $t$;
$M$ is the set of message types, and $S$ is the set of service types.

\subsubsection{Data type version selection}

DSDL 컴파일러는 모든 유효한 데이터 타입 버전을 따로따로 컴파일해야만 한다.
이렇게 해야 어플리케이션이 모든 메이저와 마이너 버전 조합으로부터 선택할 수 있다.

전송이 시작되면 데이터 타입의 메이저 버전은 어플리케이션의 재량으로 선택된다.
마이너 버전은 선택된 메이저 버전 중에 하나
When emitting a transfer, the major version of the data type is chosen at the discretion of the application.
The minor version should be the newest available one under the chosen major version.

When receiving a transfer, the node deduces which major version of the data type to use
from its port identifier (either fixed or non-fixed).
The minor version should be the newest available one under the deduced major version\footnote{%
    Such liberal minor version selection policy poses no compatibility risks since all definitions under the same
    major version are compatible with each other.
}.

It follows from the above two rules that when a node is responding to a service request,
the major data type version used for the response transfer shall be the same that is used for the request transfer.
The minor versions may differ, which is acceptable due to the major version compatibility requirements.

\begin{remark}[breakable]
    A simple usage example is provided in this intermission.

    Suppose a vendor named ``Sirius Cybernetics Corporation'' is contracted to design a
    cryopod management data bus for a colonial spaceship ``Golgafrincham B-Ark''.
    Having consulted with applicable specifications and standards, an engineer came up with the following
    definition of a cryopod status message type (named \verb|sirius_cyber_corp.b_ark.cryopod.Status|):

    \begin{minted}{python}
        # sirius_cyber_corp.b_ark.cryopod.Status.0.1

        float16 internal_temperature    # [kelvin]
        float16 coolant_temperature     # [kelvin]

        uint8 FLAG_COOLING_SYSTEM_A_ACTIVE = 1
        uint8 FLAG_COOLING_SYSTEM_B_ACTIVE = 2
        # Status flags in the lower bits.
        uint8 FLAG_PSU_MALFUNCTION = 32
        uint8 FLAG_OVERHEATING     = 64
        uint8 FLAG_CRYOBOX_BREACH  = 128
        # Error flags in the higher bits.
        uint8 flags  # Storage for the above defined flags (this is not the recommended practice).

        @extent 1024 * 8  # Pick a large extent to allow evolution. Does not affect network traffic.
    \end{minted}

    The definition is then deployed to the first prototype for initial laboratory testing.
    Since the definition is experimental, the major version number is set to zero in order to signify the
    tentative nature of the definition.
    Suppose that upon completion of the first trials it is identified that the units should track their
    power consumption in real time for each of the three redundant power supplies independently.

    It is easy to see that the amended definition shown below is not semantically compatible
    with the original definition; however, it shares the same major version number of zero, because the backward
    compatibility rules do not apply to zero-versioned data types to allow for low-overhead experimentation
    before the system is deployed and fielded.

    \begin{minted}{python}
        # sirius_cyber_corp.b_ark.cryopod.Status.0.2

        truncated float16 internal_temperature    # [kelvin]
        truncated float16 coolant_temperature     # [kelvin]

        saturated float32 power_consumption_0     # [watt] Power consumption by the redundant PSU 0
        saturated float32 power_consumption_1     # [watt] likewise for PSU 1
        saturated float32 power_consumption_2     # [watt] likewise for PSU 2
        # breaking compatibility with Status.0.1 is okay because the major version is 0

        uint8 FLAG_COOLING_SYSTEM_A_ACTIVE = 1
        uint8 FLAG_COOLING_SYSTEM_B_ACTIVE = 2
        # Status flags in the lower bits.
        uint8 FLAG_PSU_MALFUNCTION = 32
        uint8 FLAG_OVERHEATING     = 64
        uint8 FLAG_CRYOBOX_BREACH  = 128
        # Error flags in the higher bits.
        uint8 flags  # Storage for the above defined flags (this is not the recommended practice).

        @extent 512 * 8  # Extent can be changed freely because v0.x does not guarantee compatibility.
    \end{minted}

    The last definition is deemed sufficient and is deployed to the production system
    under the version number of 1.0: \verb|sirius_cyber_corp.b_ark.cryopod.Status.1.0|.

    Having collected empirical data from the fielded systems, the Sirius Cybernetics Corporation has
    identified a shortcoming in the v1.0 definition, which is corrected in an updated definition.
    Since the updated definition, which is shown below, is semantically compatible\footnote{%
        The topic of data serialization is explored in detail in section~\ref{sec:dsdl_data_serialization}.
    } with v1.0, the major version number is kept the same and the minor version number is incremented by one:

    \begin{minted}{python}
        # sirius_cyber_corp.b_ark.cryopod.Status.1.1

        saturated float16 internal_temperature    # [kelvin]
        saturated float16 coolant_temperature     # [kelvin]

        float32[3] power_consumption    # [watt] Power consumption by the PSU

        bool flag_cooling_system_a_active
        bool flag_cooling_system_b_active
        # Status flags (this is the recommended practice).

        void3   # Reserved for other flags

        bool flag_psu_malfunction
        bool flag_overheating
        bool flag_cryobox_breach
        # Error flags (this is the recommended practice).

        @extent 512 * 8  # Extent is to be kept unchanged now to avoid breaking compatibility.
    \end{minted}

    Since the definitions v1.0 and v1.1 are semantically compatible,
    UAVCAN nodes using either of them can successfully interoperate on the same bus.

    Suppose further that at some point a newer version of the cryopod module,
    equipped with better temperature sensors, is released.
    The definition is updated accordingly to use \verb|float32| for the temperature fields instead of \verb|float16|.
    Seeing as that change breaks the compatibility, the major version number has to be incremented by one,
    and the minor version number has to be reset back to zero:

    \begin{minted}{python}
        # sirius_cyber_corp.b_ark.cryopod.Status.2.0

        float32 internal_temperature    # [kelvin]
        float32 coolant_temperature     # [kelvin]

        float32[3] power_consumption    # [watt] Power consumption by the PSU

        bool flag_cooling_system_a_active
        bool flag_cooling_system_b_active
        void3
        bool flag_psu_malfunction
        bool flag_overheating
        bool flag_cryobox_breach

        @extent 768 * 8  # Since the major version number is different, extent can be changed.
    \end{minted}

    Imagine that later it was determined that the module should report additional status information
    relating to the coolant pump.
    Thanks to the implicit truncation (section \ref{sec:dsdl_serialization_implicit_truncation}),
    implicit zero extension (section \ref{sec:dsdl_serialization_implicit_zero_extension}),
    and the delimited serialization (section \ref{sec:dsdl_serialization_composite_non_sealed}),
    the new fields can be introduced in a semantically-compatible way without releasing
    a new major version of the data type:

    \begin{minted}{python}
        # sirius_cyber_corp.b_ark.cryopod.Status.2.1

        float32 internal_temperature    # [kelvin]
        float32 coolant_temperature     # [kelvin]

        float32[3] power_consumption    # [watt] Power consumption by the PSU

        bool flag_cooling_system_a_active
        bool flag_cooling_system_b_active
        void3
        bool flag_psu_malfunction
        bool flag_overheating
        bool flag_cryobox_breach

        float32 rotor_angular_velocity  # [radian/second] (usage of RPM would be non-compliant)
        float32 volumetric_flow_rate    # [meter^3/second]
        # Coolant pump fields (extension over v2.0; implicit truncation/extension rules apply)
        # If zero, assume that the values are unavailable.

        @extent 768 * 8
    \end{minted}

    It is also possible to add an optional field at the end wrapped into a variable-length
    array of up to one element, or a tagged union where the first field is empty
    and the second field is the wrapped value.
    In this way, the implicit truncation/extension rules would automatically make such optional field
    appear/disappear depending on whether it is supported by the receiving node.

    Nodes using v1.0, v1.1, v2.0, and v2.1 definitions can coexist on the same network,
    and they can interoperate successfully as long as they all support at least v1.x or v2.x.
    The correct version can be determined at runtime from the port identifier assignment as described in
    section~\ref{sec:basic_subjects_and_services}.

    In general, nodes that need to maximize their compatibility are likely to employ all existing major versions of
    each used data type.
    If there are more than one minor versions available, the highest minor version within the major version should
    be used in order to take advantage of the latest changes in the data type definition.
    It is also expected that in certain scenarios some nodes may resort to publishing the same message type
    using different major versions concurrently to circumvent compatibility issues
    (in the example reviewed here that would be v1.1 and v2.1).

    The examples shown above rely on the primitive scalar types for reasons of simplicity.
    Real applications should use the type-safe physical unit definitions available in the SI namespace instead.
    This is covered in section~\ref{sec:application_functions_si}.
\end{remark}

\section{Conventions and recommendations}

이번 섹션에서는 데이터 타입 설계자가 일관성 있는 스타일을 유지하는데 도움을 얻을 수 있는 추천 규칙과 일반적인 실수를 피하는 방법에 대해서 다룬다.
이 섹션에서 제공하는 모든 추천 규칙은 반드시 따라야하는 강제성은 없으면 선택할 수 있다.

\subsection{Naming recommendations}

일반 SW 개발 분야에서 널리 사용되는 DSDL 네이밍 추천은 아래와 같다.

\begin{itemize}
    \item Namespaces and field attributes should be named in the \verb|snake_case|.
    \item Constant attributes should be named in the \verb|SCREAMING_SNAKE_CASE|.
    \item Data types (excluding their namespaces) should be named in the \verb|PascalCase|.
    \item Names of message types should form a declarative phrase or a noun. For example,
          \verb|BatteryStatus| or \verb|OutgoingPacket|.
    \item Names of service types should form an imperative phrase or a verb. For example,
          \verb|GetInfo| or \verb|HandleIncomingPacket|.
    \item Short names, unnecessary abbreviations, and uncommon acronyms should be avoided.
\end{itemize}

\subsection{Comments}

모든 데이터 타입 정의 파일에서 시작은 해당 데이터 타입에 대한 상세한 설명, 목적, 사용 패턴, 관련 데어터 교환 패턴, 제약사항 등과 같이 데이터 타입을 사용하는데 유용한 정보를 제공하는 헤더 코멘트로 시작한다.

데이터 타입 정의에 대한 모든 속성과 각 필드의 특별한 속성은 상세한 설명, 목적, 사용 패턴, 관련 데어터 교환 패턴, 제약사항 등을 가지고 있다.
속성 중에 충분한 설명이 제공되고 이름이 의도를 명확하게 나타내는 경우에는 예외가 적용된다.

코멘트는 설명하려는 엔트티 뒤에 위치해야만 한다:
동일한 라인에 있던가 다음 라인에 위치한다.
이런 추천은 파일 헤더 코멘트에는 적용되지 않는다.

% Field comment placement https://forum.uavcan.org/t/dsdl-documentation-comments/407

\subsection{Optional value representation}

자료구조에는 항상 존재하지 않을 수 있는 옵션 필드 속성을 포함할 수도 있다.

옵션 필드 속성을 표현하는 추천 접근법은 하나의 엘리멘트로 가변 길이 배열을 사용하는 것이다.

다른 방법은 이런 1개짜리 엘리멘트 가변 길이 배열은 2개짜리 필드 unions로 대체가 가능하다. 여기서 첫번째 필드는 비어 있고 두번째 필드는 원하는 옵션 값을 포함한다.
원하는 레이아웃은 위에서 설명한 의미상 1개짜리 엘리멘트 배열과 호환되고 필드 속성이 스왑되지 않았다는 것을 제공한다.

Floating-point-typed 필드 속성은 IEEE 754 NaN의 값이 할당될 수 있다. 이것은 해당 값이 지정되지 않았다는 것을 나타낸다.:
하지만 이런 패턴은 해당 값이 존재하지 않더라도 버스 상에서 전송되어야 하는데 못하고 있으므로 사용하지 않는다. 그리고 이런 특별한 값은 타입 안전성을 헤친다.

\begin{remark}[breakable]
    Array-based optional field:

    \begin{minted}{python}
        MyType[<=1] optional_field
    \end{minted}

    Union-based optional field:

    \begin{minted}{python}
        @sealed                         # Sic!
        @union                          # The implicit tag is one byte long.
        uavcan.primitive.Empty none     # Represents lack of value, unpopulated field.
        MyType some                     # The field of interest; field ordering is important.
    \end{minted}

    The defined above union can be used as follows (suppose it is named \verb|MaybeMyType|):

    \begin{minted}{python}
        MaybeMyType optional_field
    \end{minted}

    The shown approaches are semantically compatible.
\end{remark}

\begin{remark}[breakable]
    The implicit truncation and the implicit zero extension rules allow one to freely add such optional fields
    at the end of a definition while retaining semantic compatibility.
    The implicit truncation rule will render them invisible to nodes that utilize older data type definitions
    which do not contain them, whereas nodes that utilize newer definitions will be able to correctly process
    objects serialized using older definitions because the implicit zero extension rule guarantees
    that the optional fields will appear unpopulated.

    For example, let the following be the old message definition:

    \begin{minted}{python}
        float64 foo
        float32 bar
    \end{minted}

    The new message definition with the new field is as follows:

    \begin{minted}{python}
        float64 foo
        float32 bar
        MyType[<=1] my_new_field
    \end{minted}

    Suppose that one node is publishing a message using the old definition,
    and another node is receiving it using the new definition.
    The implicit zero extension rule guarantees that the optional field array will
    appear empty to the receiving node because the implicit length field will be read as zero.
    Same is true if the message was nested inside another one, thanks to the delimiter header.
\end{remark}

\subsection{Bit flag representation}

비트 플래그들의 집합을 정의하는 추천 방법은 각각에 대해서 \verb|bool|-타입 필드 속성을 명시하는 것이다.
2의 자승의 정수 합을\footnote{Which are popular in programming.} 기반으로 하는 표현은 의도를 명확히 드러내는데 모호하므로 사용하지 않는다.

\begin{remark}
    Recommended approach:

    \begin{minted}{python}
        void5
        bool flag_foo
        bool flag_bar
        bool flag_baz
    \end{minted}

    Not recommended:

    \begin{minted}{python}
        uint8 flags             # Not recommended
        uint8 FLAG_BAZ = 1
        uint8 FLAG_BAR = 2
        uint8 FLAG_FOO = 4
    \end{minted}
\end{remark}


\input{transport/transport.tex}
\input{application/application.tex}
\chapter{List of standard data types}\label{sec:sdt}

이 챕터는 UAVCAN 스펙에서 정의한 표준 데이터 타입의 전체 목록을 포함한다.
여기서 제공하는 DSDL 데이터 타입 정의의 소스 텍스트는 공식 프로젝트 웹사이트인 \href{http://uavcan.org}{uavcan.org}를 통해 확인할 수 있다.

비표준 정의\footnote{%
    I.e., public definitions contributed by vendors and other users
    of the specification, as explained in section~\ref{sec:basic_data_type_regulation}.
}는 이 목록에 포함되지 않는다.

테이블에서 \emph{BLS}는 bit 길이 집합을 의미한다.
실드된 엔트티에 대해서는 크기가 보여지지 않는다. -- 실링이 의미하는 것은 크기와 최대 bit 길이 집합이 동일하기 때문에 중복이 될 수 있다.
서비스 타입에 대해서 request와 response에 속하는 파라미터는 따로 보여진다.

인덱스 테이블~\ref{table:dsdl:uavcan}은 네비게이션이 쉽도록 정의 전에 제공된다.

\clearpage\DSDL{uavcan.*}


\end{document}
