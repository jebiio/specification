\chapter{Introduction}\label{sec:introduction}

This is a non-normative chapter covering the basic concepts that govern development and maintenance of
the specification.

\section{Overview}

UAVCAN은 경량 프로토콜로서 publish-subscribe과 원격 프로시저 호출 방식의 높은 신뢰성을 가지는 통신을 제공하도록 설계되었다. 장치내에 있는 강건한 버스 네트워크를 통해서 항공기와 로보틱스 응용 장치에 적합하다.
차세대 지능 운송 장치의 온보드 상에 시스템과 컴포넌트 사이의 데이터 교환의 문제점을 해결하기 위해 고안되었다.: 유인/무인 비행장치, 우주선, 로봇, 차량


UAVCAN can be approximated as a highly deterministic decentralized object request broker
with a specialized interface description language and a highly efficient data serialization format
suitable for use in real-time safety-critical systems with optional modular redundancy.

UAVCAN의 \emph{Uncomplicated Application-level Vehicular Computing And Networking}를 의미한다.

UAVCAN은 모두에게 오픈된 표준이며 향후에도 계속 오픈된 상태를 유지할 것이다.
구현, 배포, 사용에 있어서 허가나 승인이 필요하지 않다.

UAVCAN 스펙의 개발 및 유지보수는 공개된 토론 포럼, 소프트웨어 저장소, 공식 사이트 \href{http://uavcan.org}{uavcan.org}를 통해서 이뤄진다.

UAVCAN을 활용을 모색하는 엔지니어는 공식 웹사이트에서 별도로 제공하는 \emph{The UAVCAN Guide} 를 참고하도록 한다.

\section{Document conventions}

프로토콜의 정의에 직접 들어가지 않는 비공식 내용, 예제, 추천, 퇴고는 footnotes\footnote{This is a footnote.}에 포함되거나 아래와 같이 강조 섹션에 포함된다.

\begin{remark}
    예제와 같이 비공식 내용은 여기와 같이 어두운 박스에 표현된다.
\end{remark}

코드는 아래와 같은 형태로 표현된다.
이런 형태의 모든 코드는 특별한 언급이 없다면 이 스펙과 동일한 라이센스로 배포된다.

\begin{minted}{rust}
    // This is a source code listing.
    fn main() {
        println!("Hello World!");
    }
\end{minted}

하나의 바이트(byte)는 8개 비트(bit)의 그룹이다.
A byte is a group of eight (8) bits.

Textual patterns are specified using the standard
POSIX Extended Regular Expression (ERE) syntax;
the character set is ASCII and patterns are case sensitive, unless explicitly specified otherwise.

Type parameterization expressions use subscript notation,
where the parameter is specified in the subscript enclosed in angle brackets:
$\texttt{type}_\texttt{<parameter>}$.

Numbers are represented in base-10 by default.
If a different base is used, it is specified after the number in the subscript\footnote{%
    E.g., $\text{BADC0FFEE}_{16} = 50159747054$, $10101_2 = 21$.
}.

DSDL definition examples provided in the document are illustrative and may be incomplete or invalid.
This is to ensure that the examples are not cluttered by irrelevant details.
For example, \verb|@extent| or \verb|@sealed| directives may be omitted if not relevant.

\section{Design principles}

\begin{description}
    \item[평등한 네트워크] --- 마스터 노드가 존재하지 않는다.
    네트워크 내에 있는 모든 노드들은 동일한 통신 권한을 가진다. 단일 장애로 네트워크 전체가 불능상태로 빠지지 않는다.

    \item[기능 안정성 촉진] --- A system designer relying on UAVCAN will have the necessary
    guarantees and tools at their disposal to analyze the system and ensure its correct behavior.

    \item[High-level 통신 추상화] --- The protocol will support publish/subscribe and remote procedure
    call communication semantics with statically defined and statically verified data types (schema).
    The data types used for communication will be defined in a clear, platform-agnostic way
    that can be easily understood by machines, including humans.  % I hope you are ok with this, my dear fellow robots.

    \item[벤더간 상호운용성 촉진] --- UAVCAN will be a common foundation that
    different vendors can build upon to maximize interoperability of their equipment.
    UAVCAN will provide a generic set of standard application-agnostic communication data types.

    \item[잘 정의된 일반 high-level 기능] --- UAVCAN will define standard services
    and messages for common high-level functions, such as network discovery, node configuration,
    node software update, node status monitoring, network-wide time synchronization, plug-and-play node support, etc.

    \item[Atomic data abstractions] --- Nodes shall be provided with a simple way of exchanging large
    data structures that exceed the capacity of a single transport frame\footnote{%
        A \emph{transport frame} is an atomic transmission unit defined by the underlying transport protocol.
        For example, a CAN frame.
    }.
    UAVCAN should perform automatic data decomposition and reassembly at the protocol level,
    hiding the related complexity from the application.

    \item[High throughput, low latency, determinism] --- UAVCAN will add a very low overhead to the underlying
    transport protocol, which will ensure high throughput and low latency, rendering the protocol well-suited
    for hard real-time applications.

    \item[Support for redundant interfaces and redundant nodes] --- UAVCAN shall be suitable for use in
    applications that require modular redundancy.

    \item[Simple logic, low computational requirements] --- UAVCAN targets a wide variety of embedded systems,
    from high-performance on-board computers to extremely resource-constrained microcontrollers.
    It will be inexpensive to support in terms of computing power and engineering hours,
    and advanced features can be implemented incrementally as needed.

    \item[Rich data type and interface abstractions] --- An interface description language will be a core part of
    the technology which will allow deeply embedded sub-systems to interface with higher-level systems directly and
    in a maintainable manner while enabling simulation and functional testing.

    \item[Support for various transport protocols] --- UAVCAN will be usable with different transports.
    The standard shall be capable of accommodating other transport protocols in the future.

    \item[API-agnostic standard] --- Unlike some other networking standards, UAVCAN will not attempt to describe
    the application program interface (API). Any details that do not affect the behavior of an implementation
    observable by other participants of the network will be outside of the scope of this specification.

    \item[Open specification and reference implementations] --- The UAVCAN specification will always be open and
    free to use for everyone; the reference implementations will be distributed under the terms of
    the permissive MIT License or released into the public domain.
\end{description}

\section{Capabilities}

논리 네트워크에서 최대 노드의 수는 사용하는 트랜스포트 프로토콜에 따라 다르지만 128보다는 많다는 것은 보장한다.

UAVCAN은 무한대의 컴포지트 데이터 타입을 지원하며 이런 데이터 타입에는 스펙에 정의하거나(\emph{표준 데이터 타입(standard data types)}) 다른 사람이 개별적인 사용이나 공개를 위해서(\emph{application-specific} 혹은 \emph{vendor-specific}) 정의할 수 있다.
데이터 타입의 256개 까지의 메이저 버전과 각 메이저 버전에는 최대 256개까지의 마이너 버전이 허용된다.

UAVCAN은 publish/subscribe 통신을 위해서 8192 메시지 서브젝트 식별자를 지원하고 원격 프로시저 호출 교환을 위해서는 512개 서비스 식별자를 지원한다.
이런 식별자의 일부는 핵심 표준과 벤더가 배포하는 타입을 위해서 예약되어 있다.(chapter~\ref{sec:application})

트랜스포트 프로토콜에 따라서 UAVCAN은 적어도 8개의 구별되는 통신 우선순위 레벨을 지원한다. (section~\ref{sec:transport_transfer_priority})

UAVCAN이 지원하는 트랜스포트 프로토콜의 목록은 chapter~\ref{sec:transport}를 참고하자.
비다중화, 이중화, 삼중화 트랜스포트를 지원한다.
추가적인 트랜스포트 계층은 향후 프로토콜 버전에 추가될 예정이다.

어플리케이션-레벨에서 할 수 있는 일(시간 동기화, 파일 전송, 노드 소프트웨어 업데이트, 진단, 스키마 없는 레지스터, PnP 노드 추가 등등)의 목록은 section~\ref{sec:application_functions}을 참고하자.

핵심 스펙에는 전송에서 물리적 계층에 대해서 명시적인 제약사항을 정의하지 않았다. 하지만 UAVCAN이 요구하는 속성은 물리적 네트워크에 대한 제약이나 최소 성능요구사항을 내포하고 있다. 이런 이유로 핵심 표준에서는 물리적 네트워크 상에 있는 노드끼리 트랜스포트 계층 아래 부분에서 호환에 대한 제약을 두지 않는다.(예) 하드웨어 제약이 적용되지 않는 가상 네트워크에 연결된 노드들 사이의 호환성을 걱정하지 않아도 된다.)
커넥터를 포함하여 물리 계층에 대한 추가적인 표준은 운송 시스템 내부 표준을 따라 사용할 수 있다.

프로토콜에서 정의한 기능들은 스펙의 메이저 버전내에서는 절대로 줄어들지 않으며 확장될 수는 있다.

\section{Management policy}

UAVCAN 유지보수자는 스펙과 public regulated data types\footnote{%
The related technical aspects are covered in chapters~\ref{sec:basic} and~\ref{sec:dsdl}.
}을 유지 관리하고 발전시키는 일을 담당하고 있다.
The UAVCAN maintainers are tasked with maintaining and advancing this specification and
the set of public regulated data types\footnote{%
    The related technical aspects are covered in chapters~\ref{sec:basic} and~\ref{sec:dsdl}.
} based on their research and the input from adopters.
The maintainers will be committed to ensuring long-term stability and backward compatibility of
existing and new deployments.
The maintainers will publish relevant announcements and solicit inputs from adopters
via the discussion forum whenever a decision that may potentially affect existing deployments is being made.

The set of standard data types is a subset of public regulated data types and is an integral part of the specification;
however, there is only a very small subset of required standard data types needed to implement the protocol.
A larger set of optional data types are defined to create a standardized data exchange environment
supporting the interoperability of COTS\footnote{Commercial off-the-shelf equipment.}
equipment manufactured by different vendors.
Adopters are invited to take part in the advancement and maintenance of the public regulated data types
under the management and coordination of the UAVCAN maintainers.

\section{Referenced sources}

The UAVCAN specification contains references to the following sources:

% Please keep the list sorted alphabetically.
\begin{itemize}
    \item CiA 103 --- Intrinsically safe capable physical layer.
    \item CiA 801 --- Application note --- Automatic bit rate detection.

    \item IEEE 754 --- Standard for binary floating-point arithmetic.
    \item IEEE Std 1003.1 --- IEEE Standard for Information Technology --
          Portable Operating System Interface (POSIX) Base Specifications.

    \item IETF RFC2119 --- Key words for use in RFCs to Indicate Requirement Levels.

    \item ISO 11898-1 --- Controller area network (CAN) --- Part 1: Data link layer and physical signaling.
    \item ISO 11898-2 --- Controller area network (CAN) --- Part 2: High-speed medium access unit.
    \item ISO/IEC 10646 --- Universal Coded Character Set (UCS).
    \item ISO/IEC 14882 --- Programming Language C++.

    \item \href{http://semver.org}{semver.org} --- Semantic versioning specification.

    \item ``A Passive Solution to the Sensor Synchronization Problem'', Edwin Olson.
    \item ``Implementing a Distributed High-Resolution Real-Time Clock using the CAN-Bus'', M. Gergeleit and H. Streich.
    \item ``In Search of an Understandable Consensus Algorithm (Extended Version)'', Diego Ongaro and John Ousterhout.
\end{itemize}

\section{Revision history}

\subsection{v1.0 -- work in progress}

\begin{itemize}
    \item The maximum data type name length has been increased from 50 to 255 characters.

    \item The default extent function has been removed (section \ref{sec:dsdl_composite_extent_and_sealing}).
    The extent now has to be specified explicitly always unless the data type is sealed.
\end{itemize}

\subsection{v1.0-beta -- Sep 2020}

Compared to v1.0-alpha, the differences are as follows (the motivation is provided on the forum):

\begin{itemize}
    \item The physical layer specification has been removed.
    It is now up to the domain-specific UAVCAN-based standards to define the physical layer.

    \item The subject-ID range reduced from $[0, 32767]$ down to $[0, 8191]$.
    This change may be reverted in a future edition of the standard, if found practical.

    \item Added support for delimited serialization; introduced related concepts of \emph{extent} and \emph{sealing}
    (section \ref{sec:dsdl_composite_extent_and_sealing}).
    This change enables one to easily evolve networked services in a backward-compatible way.

    \item Enabled the automatic runtime adjustment of the transfer-ID timeout on a per-subject basis
    as a function of the transfer reception rate (section \ref{sec:transport_transfer_reception}).
\end{itemize}

\subsection{v1.0-alpha -- Jan 2020}

문서의 초기 버전이다.
초기 버전을 위해 논의된 내용은 공개된 UAVCAN 포럼에서 확인할 수 있다.
